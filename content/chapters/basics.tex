\chapter{Theoretische Grundlagen}

In diesem Kapitel werden die für den Forschungshintergrund und für die Entwicklung des Prototyps wichtigen theoretischen Grundlagen vorgestellt.
Zunächst werden verschiedene Theorien zu Kommunikationsmodellen vorgestellt, auf deren Basis das Modell für diese Anwendung zugrunde liegt.
Im Anschluss daran werden die in der Ludologie beschriebenen Akteure vorgestellt, welche sich in den Probanden dieser Masterarbeitsstudie wiederfinden werden. Allgemein ist bekannt, dass Video- und Computerspiele drei verschiedene Modi haben können: Singleplayer, Multiplayer und Mischformen. Da für den Zweck dieser Studie ein Multiplayer-Spiel konzipiert und umgesetzt wurde, werden im weiteren Verlauf verschiedene Kategorien von Multiplayer-Spielen vorgestellt. Außerdem werden die damit einhergehenden Netzwerkinfrastrukturen vorgestellt, die relevant sind und für die weitere Entwicklung relevant sein könnten.

\section{Kommunikationsmodelle}
[Literatur suchen]
% In der Kommunikationswissenschaft wird die Kommunikation in 2 Arten unterteilt, die Massen- und Individualkommunikation.

\subsection{Nach Schulz von Thun}
\subsection{Nach Wazlawik}
\subsection{Nach Rogers}

% [Könnte besser in die Einleitung passen
% \section{Spiele als soziales Medium}
% \cite{depping_trust_2016}
% \cite{gerling_designing_2014}
% \cite{ducheneaut_alone_2006}
% ]

\section{Spielertypen}
Die Kommunikationswissenschaft umfasst zwar den Hauptteil dieser Arbeit, allerdings beinhaltet diese ebenfalls ludologische Aspekte. Dabei geht es um die Lehre über das Spiel (vgl. \cite{ludologie_spielforschung_nodate}). 

Im Hinblick auf die Konzeption und die Entwicklung eines Spiels ist es wichtig, die Eigenschaften des Spielsystems so zu gestalten, dass sich Begeisterung und Engagement bei der gewünschten Zielgruppe hervorrufen. Aus diesem Grund muss zunächst die Zielgruppe in verschiedene Typen eingeteilt werden. In der Ludologie gibt es dafür verschiedene Spielertypen. Zwar ist nicht jeder Mensch ein \say{Spielertyp}, grundsätzlich kann er jedoch über verschiedene Spielelemente angesprochen werden (vgl. \cite{ludologie_spielertypen_nodate}).

\subsection{Nach Bartle}
1996 beschäftigte sich Richard Bartle mit der Frage, welche Spielertypen es in der Ludologie gibt. Dabei ging es zunächst um die Klassifizierungen, welche Ansätze es beim Spielen von sogenannten \ac{MUD}s existieren (vgl. \cite{bartle_hearts_1996}). Diese Klassifizierungen werden noch heute für die Einteilung in Spielertypen genutzt.

Bartle unterscheidet bei der Einteilung der Spielertypen auf zwei unterschiedliche Grundinteressen (vgl. Abbildung \ref{fig:bartle-muds}):

\begin{figure}[ht]
\centering
\includegraphics[width=1\linewidth]{content/pictures/basic_interests.PNG}
\caption{Interessen Graph nach Bartle (vgl. \cite{bartle_hearts_1996})}
\label{fig:bartle-muds}
\end{figure}

In X-Achsen-Richtung wird unterschieden, ob der Spieler seine Spielerfahrung über das Verhalten der anderen Mitspieler (Players) oder der Spielwelt (World) bevorzugt. Auf der Y-Achsen-Richtung unterscheidet er, ob der Spieler bevorzugt, selbst einen Einfluss auf die Spielwelt zu geben und diese beeinflusst (Acting) oder ob er in tiefere Interaktion mit der Spielwelt eingehen will (Interacting).

Die daraus resultierenden Typen sind:
\paragraph{Achiever}
Sie sind daran interessiert, auf die Welt einzuwirken, um dadurch in sie eintauchen zu können. Sie wollen das Spiel meistern und dazu bringen, das zu tun, was sie wollen. Ihr Status im Spiel ist ihnen wichtig und die wenige Zeit, die sie dafür benötigt haben.

\paragraph{Explorer}
Sie wollen vom Spiel überrascht werden und mit der Spielwelt interagieren. Die virtuelle Welt löst ein Gefühl des Staunens aus, nach dem sie sich sehnen. Sie sind stolz auf das Wissen über das Spiel, das sie sammeln, und wollen dieses Wissen gerne an neue Spieler weitergeben.

\paragraph{Socialiser}
Sie wollen mit anderen Spielern interagieren. Zumeist erfolgt das über Gespräche, es kann aber auch ungewöhnliche Verhaltensweisen einschließen. Andere Menschen kennenzulernen und mehr über sie zu erfahren, ist für sie wertvoller als für andere. Die Spielwelt ist für sie nur eine Kulisse, für sie sind andere Charaktere fesselnder. Sie sind stolz auf Freundschaften, ihre Kontakte und ihren Einfluss.

\paragraph{Killer}
Sie sind daran interessiert, auf andere Spiele einzuwirken und mit ihnen Dinge zu machen. Im Allgemeinen erfolgt dies ohne das Einverständnis der anderen Spieler. Sie wollen ihre Überlegenheit gegenüber anderen Menschen demonstrierenie sind stolz auf ihren Ruf und oft geübte Kampffähigkeiten.

(vgl. \cite{bartle_hearts_1996}).

\subsection{Erweiterte Spielertypeneinteilungen}


% \cite{bartle_hearts_nodate}

% [erwähnen wurde aber rausgelassen, man kann erwähnen, dass es noch weitere klassifizierungen gibt
% \subsection{Das BrainHex-Model}
% \cite{nacke_brainhex_2014}
% ]

% [kommt zu wichtige Begriffe
% \section{Kooperative Gamedesign Pattern}
% \subsection{Was sind Game Pattern}
% \cite{bjork_patterns_2005}

% \subsection{Complementarity}

% \subsection{Synergies}

% \subsection{Abilities}

% \subsection{Shared Goals}

% \subsection{Synergies between goals}

% \subsection{Special Rules for Player of the same Team}

% \subsection{Camera Setting}

% \subsection{Interacting with the same object}

% \subsection{Shared puzzle}

% \subsection{Shared characters}

% \subsection{Special characters targetting lone wolf}

% \subsection{Vocalization}

% \subsection{Limited ressources}

% \subsection{Einflussnahme}
% \cite{emmerich_impact_2017}

% ]
\section{Multiplayerspiele}

\subsection{Klassifizierungen}

\subsubsection{Synchrone Multiplayer}

\subsubsection{Asynchrone Multiplayer}

\subsubsection{Symmetrische Multiplayer}

\subsubsection{Asymmetrische Multiplayer}

\subsection{Artverwandte Beispiele}
Hier kommen die analysierten Spiele rein, also die Auflistung der Spiele, die ich mir im Zuge angesehen habe

\section{Netzwerkinfrastrukturen}

enthält eine Liste von Möglichkeiten auf welcher Grundlage verschiedene Multiplayer Anwendungen gebaut werden können


\chapter{Verwandte Arbeiten}

\cite{harris_asymmetry_2019}
\cite{sajjadi_maze_2014}

hier würden Paper reinkommen die asymmetrische Multiplayer gemacht haben, welche aspekte da mitreinspielen, da kommen dann auch die wichtigen Begriffe dazu mitrein. Auch bereits umgesetzt asymetrische VR Spiele?


Auch Anna Lotz´ Thesis wäre hier relevant


\section{Wichtige Begriffe}

\subsection{Interdependence}
\cite{harris_leveraging_2016}
\cite{depping_cooperation_2017}

\subsection{Degrees of Interdependence}
\cite{beznosyk_effect_2012}

\subsection{Soziale Präsenz}

Vertrauen gibts in dem Kontext auch und wie man dan über Spiele aufbaut

\section{Untersuchungsschwerpunkte}