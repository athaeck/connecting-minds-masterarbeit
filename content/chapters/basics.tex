\chapter{Theoretische Grundlagen}\label{sec:basics}

In diesem Kapitel werden die für den Forschungshintergrund und die Entwicklung des Prototyps wichtigen theoretischen Grundlagen beschrieben.
Zunächst werden verschiedene Theorien zu Kommunikationsmodellen und kooperativer Kommunikation vorgestellt, die dem Modell dieser Anwendung zugrunde liegen.
Im Anschluss daran werden die in der Ludologie beschriebenen Akteurstypen vorgestellt, die sich in den Teilnehmenden dieser Masterarbeitsstudie widerspiegeln.
Allgemein ist bekannt, dass Video- und Computerspiele drei verschiedene Modi haben können: Singleplayer, Multiplayer und Mischformen. Da im Rahmen dieser Studie ein Multiplayer-Spiel konzipiert und umgesetzt wurde, werden im weiteren Verlauf verschiedene Kategorien von Multiplayer-Spielen vorgestellt. Außerdem werden die damit einhergehenden Netzwerkinfrastrukturen dargestellt, die für die aktuelle sowie eine potenzielle zukünftige Entwicklung relevant sind. Zum Schluss erfolgt ein Einblick in die \ac{AR}.

\section{Kommunikationsforschung}

Die Kommunikationsforschung umfasst innerhalb der Kommunikationswissenschaft viele Gebiete. Aus diesem Grund muss zunächst der Begriff \say{Kommunikation} definiert werden.

Kommunikation bezeichnet den Prozess des Sendens und Empfangens von Botschaften, den Austausch von Informationen sowie die Interaktion mit anderen. Sei es im direkten persönlichen Kontakt oder über eine Vielzahl etablierter und neuer Technologien (vgl. \citealp[S. 18]{krcmar_communication_2016}).
Bentele und Beck präzisieren diese Definition, indem sie ergänzen, dass die wechselseitige Bedeutungsübermittlung sowohl sprachlich-symbolisch als auch nonverbal erfolgen kann, etwa durch Blicke oder Gesten (vgl. \citealp[S. 20]{bentele_information_1994}).

\begin{quote}
\textit{\enquote{Kommunikation wird somit als soziale Interaktion zwischen zwei oder mehreren Menschen verstanden, die dem Austausch von Informationen, Gedanken, Erfahrungen etc. innerhalb einer aktuellen Situation dient.}} \cite[S. 19]{becker_praxishandbuch_2018}
\end{quote}

Sobald Menschen miteinander kommunizieren, geschieht dies in der Regel auf mehrere Ebenen gleichzeitig. Dabei wird zwischen verbaler, paraverbaler und nonverbaler Kommunikation unterschieden. 

Verbale Informationen können gelesen ebenfalls einen Sinn ergeben.
Nonverbale Informationen durch Mimik, Gestik und die Körpersprache können nur dann verstanden werden, wenn der Gegenüber sie sieht. 
Paraverbale Informationen werden durch die Betonung, Stimmlage, Pausenlängen etc. übermittelt. Sie können nur durch das Hören gedeutet werden (vgl. \citealp[S. 33]{ebert_formen_2018}).

Kommunikation ist durch gegenseitige Abhängigkeiten geprägt: Während der Sender das Geschehen aktiv gestaltet, nimmt der Empfänger Informationen auf und interpretiert sie. Dabei ist nicht jede Kommunikation bewusst beabsichtigt. Die Abgrenzung zwischen Kommunikation und Interaktion ist schwierig, da Kommunikation häufig als Bestandteil von Interaktion verstanden wird. Kommunikatives Verhalten wird sowohl durch bewusste Erfahrungen und Lernprozesse als auch durch unbewusste Verhaltensmuster beeinflusst (vgl. \citealp[S. 20]{becker_praxishandbuch_2018}.

Um besser zu verstehen, welche Prozesse innerhalb der Kommunikation ablaufen und wie sie erlebt und interpretiert wird, wurden im Laufe der Zeit zahlreiche Theorien und Modelle entwickelt (vgl. \citealp[S. 311]{schwarz_grundlagen_2019}).

Diese \say{Kommunikationsmodelle} berücksichtigen insbesondere den sozialen Kontext. Sie sind keine exakten Abbilder kommunikativer Realität, sondern vereinfachte Darstellungen realer Erscheinungsformen und Prozesse (vgl. \citealp[S. 56]{maletzke_kommunikationswissenschaft_1998}).

\subsection{Kommunikationsmodell nach Shannon und Weaver}

\begin{figure}[ht]
\centering
\includegraphics[width=1\linewidth]{content/pictures/shannon-weaver.PNG}
\caption{Kommunikationsmodell von Shannon und Weaver (Quelle: \citealp[S. 2]{shannon_mathematical_1948})}
\label{fig:shannon-weaver-modell}
\end{figure}

Abbildung \ref{fig:shannon-weaver-modell} zeigt das wohl bekannteste Klassifikationsmodell von Shannon und Weaver. Es beschreibt jedoch keine soziale Kommunikation, sondern einen technischen Signaltransfer zwischen Sender und Empfänger. Im Fokus steht die physikalische Informationsübertragung, beispielhaft in der Form eines Telefongesprächs (vgl. \citealp[S. 92]{scheufele_kommunikationstheorien_2004}). 

Dabei geht das Modell von einer Informationsquelle aus, die eine Nachricht an ein Ziel übermitteln möchte. Die Nachricht wird zunächst von einem Transmitter in ein analoges Signal umgewandelt und anschließend an ein Empfangsgerät gesendet. Während der Übertragung kann es zu Störungen kommen, durch externe Geräusche oder technische Einflüsse, die das Signal verfälschen. Diese Störsignale werden gemeinsam mit der ursprünglichen Nachricht übermittelt. Am Ende des Prozesses empfängt das Empfangsgerät die Nachricht, sodass der Empfämger die Nachricht, einschließlich möglicher Störungen, bspw. als Audiosignal wahrnehmen kann.

\subsection{Kommunikationsmodell nach Osgood und Schramm}
Während das Modell von Shannon und Weaver die technischen Aspekte der Informationsübertragung in den Mittelpunkt stellt, stößt es bei der Beschreibung sozialer Kommunikation an seine Grenzen. Insbesondere der interaktive und dynamische Charakter zwischenmenschlicher Verständigung bleibt unberücksichtigt.

\begin{figure}[ht]
\centering
\includegraphics[width=1\linewidth]{content/pictures/osgood-schramm.jpg}
\caption{Kommunikationsmodell von Osgood und Schramm (Quelle: \citealp{wrench_24_2021})}
\label{fig:osgood-schramm-modell}
\end{figure}

Das in Abbildung \ref{fig:osgood-schramm-modell} gezeigte Modell unterscheidet sich grundlegend von früheren linearen Modellen wie dem von Shannon und Weaver. Es betont stattdessen die wechselseitige Struktur und Zirkularität von Kommunikation. Kommunikation ist hier kein einmaliger Signaltransfer, sondern ein kontinuierlicher Prozess, bei dem beide Gesprächspartner die Bedeutung einer Botschaft gemeinsam aushandeln. Feedback ist dabei kein optionales, sondern ein zentrales Element. Es ermöglicht es, Missverständnisse zu klären, Reaktionen einzuholen und aussagen zu präzisieren (vgl. \citealp{noauthor_osgood_2024}). 

Besonders an dem Modell ist, dass persönliche Merkmale der Kommunikationsteilnehmer, wie individuelle Erfahrungen, kulturelle Prägungen, Bildungshintergründe und Erwartungen, in den Gesprächszyklus einfließen. Auf Grundlage dieser Faktoren erfolgt ein Codoierungs- und Decodierungsprozess in sprachliche und nicht-sprachliche Zeichen, die vom Empfänger interpretiert werden (vgl. \citealp{noauthor_osgood_2024}). 

\subsection{Kommunikationsmodell nach Badura}
Als Erweiterung des Modells von Osgood und Schramm kann das Modell von Badura gesehen werden (vgl. Abbildung \ref{fig:badura-modell}).

\begin{figure}[ht]
\centering
\includegraphics[width=1\linewidth]{content/pictures/badura.PNG}
\caption{Kommunikationsmodell von Badura (Quelle: \cite{badura_kommunikation_1992}, \cite[S. 93]{scheufele_kommunikation_2007})}
\label{fig:badura-modell}
\end{figure}

Das Kommunikationsmodell unterscheidet zwischen semantischen, syntaktischen und pragmatischen Aspekten der Sprache bzw. kommunikativen Botschaften. Der Rahmen, in dem Kommunikation stattfindet, umfasst in diesem Modell vier zentrale Aspekte der Kommunikationssituation: das Informationsniveau der Teilnehmer, ihrem emotionalen Erlebnishorizont in den jeweiligen Situationen sowie ihre Interessen und Ziele (vgl. \citealp[S. 93]{scheufele_kommunikation_2007}).

\subsection{Das Kommunikationsquadrat nach Schultz von Thun}
Während die zuvor genannten Modelle den Ablauf und die Zirkularität von Kommunikation beschreiben, rückt das Kommunikationsquadrat die inhaltliche Vielschichtigkeit jeder Botschaft in den Mittelpunkt.

\begin{figure}[ht]
\centering
\includegraphics[width=1\linewidth]{content/pictures/Kommunikationsquadrat.PNG}
\caption{4 - Ohren Modell von Schulz von Thun (Quelle: \citealp{noauthor_kommunikationsquadrat_nodate})}
\label{fig:four-ears}
\end{figure}

Das in Abbildung \ref{fig:four-ears} gezeigte Modell, auch \say{Nachrichtenquadrat} oder \say{Kommunikationsquadrat} genannt, zeigt dass jede Nachricht vier Botschaften gleichzeitig übermittelt. Diese sind die Sachinformation,die darüber Auskunft gibt, worum es inhaltlich geht. Die Selbstkundgabe, bei der der Sender Informationen über sich selbst preisgibt. Dem Beziehungshinweise, der erkennen lässt, in welchem Verhältnis der Sender zum Empfänger steht. Sowie der Appell, durch den der Sender versucht, beim Empfänger eine bestimmte Reaktion oder Handlung auszulösen (vgl. \citealp{noauthor_kommunikationsquadrat_nodate}).

Die gewählten Äußerungen entstammen dabei aus den sog. \say{vier Schnäbeln} des Senders, die auf die \say{vier Ohren} des Empfängers treffen (vgl. \citealp{noauthor_kommunikationsquadrat_nodate}). Schulz von Thun geht dabei davon aus, dass jede Nachricht mit vier Ohren empfangen werden (vgl \citealp[S. 23]{becker_praxishandbuch_2018}). 

Das Modell ist im Kern ein Ausdrucksmodell der Kommunikation. Es soll einerseits dabei helfen, die Ursachen möglicher Missverständnisse besser zu verstehen. Andererseits wird jedoch kritisiert, dass es sowohl den Charakter der Kommunikation als gemeinschaftliche Handlung als auch den steuernden Aspekt des Sprechens vernachlässigt (vgl \citealp[S. 23]{becker_praxishandbuch_2018}). Es berücksichtigt somit nicht die Möglichkeit des Nachfragens beim Sender. Stattdessen legt das Modell nahe, dass jeder Äußerung ein einziger \say{wahrer} Bedeutungskern innewohne (vgl \citealp[S. 23]{becker_praxishandbuch_2018}). 

\subsection{Konversationsmaximen von Grice}
Grice erweitert die o. g. Theorien insofern, das Kommunikation nicht nur über das Gesagte, sondern auch über das Implizierte funktioniert (vgl. \citealp[S. 43f]{grice1975logic}). 

Ausgangspunkt seiner Überlegungen sind die Bedingungen, unter deren Konversationen stattfinden. In der Regel bestehen sie nicht aus zufälligen und unzusammenhängenden Äußerungen, sondern folgen einer bestimmten Struktur. Daraus leitet Grice das sogenannte Kooperationsprinzip ab. Dieses besagt, dass ein Gesprächsbeitrag so gestaltet sein soll, wie es der Zweck oder die Richtung des Gesprächs im jeweiligen Stadium erfordert (vgl. \citealp[S. 45]{grice1975logic}).

Aus diesem Prinzip ergeben sich vier Maximen der Kommunikation. 

Die Maxime der Quantität (Informationsmenge) besagt, dass der gesprochene Beitrag so informativ wie erforderlich, jedpch nicht informativer als nötig gestaltet sein soll. Die Maxime der Qualität (Wahrheit) betrifft die Aufrichtigkeit der Nachricht. Es soll nichts gesagt werden, was der Sprecher für falsch hält oder wofür keine ausreichenden Belege vorliegen. Die Maxime der Relevanz fordert, dass der Beitrag sachbezogen und zum Gesprächsziel passend ist. Schließlich umfasst die Maxime der Modalität (Art und Weise des Gesagten) die Vermeidung von Unklarheiten und Mehrdeutigkeiten sowie das Vermeiden von Ausschweifungen und die Forderung nach Ordnung in der Darstellung (vgl. \citealp[S. 45f]{grice1975logic}).

Wird gegen eine der Maximen verstoßen, kann der Gesprächspartner davon ausgehen, dass dies nicht zufällig, sondern absichtlich geschieht. Somit wird eine zusätzliche Bedeutung impliziert. In solchen Fällen entsteht eine implizite Bedeutung, die über das wörtlich Gesagte hinausgeht. Doe Maximen dienen somit als Interpretationshilfe, um solche kommunikativen Hinweise richtig zu deuten (vgl. \citealp[S. 49f]{grice1975logic}).

\subsection{Axiome der Kommunikationen nach Watzlawik et. al}
Ergänzend zu den davor dargestellten Modellen betrachten Watzlawik, Beavin und Jackson zwischenmenschliche Kommunikation aus einer systematisch-konstruktivistischen Perspektive. Im Mittelpunkt steht dabei Kommunikation als wechselseitiges Verhalten innerhalb sozialer Kontexte (vgl. \citealp{Watzlawick2016-km}). 

Zur Beschreibung dieses Verständnisses definieren die Autoren fünf Grundprinzipien (Axiome) mit denen sich menschliche Kommunikation systematisch erklären lässt (vgl. \citealp{Watzlawick2016-km}):
\begin{itemize}
    \item \say{Man kann nicht nicht kommunizieren}, da jedes Verhalten kommunikativen Charakter besitzt, ist Kommunikation unausweichlich (vgl. \citealp[S. 53]{Watzlawick2016-km}).
    \item \say{Jede Kommunikation hat einen Inhalts- und einen Beziehungsaspekt, derart, dass letzterer den ersteren bestimmt und daher eine Metakommunikation ist.} \cite[S. 64]{Watzlawick2016-km}
    \item \say{Die Natur einer Beziehung ist durch die Interpunktion der Kommunikationsabläufe seitens der Partner bedingt.} Die Kommunikation verläuft zirkular. (vgl. \citealp[S. 69]{Watzlawick2016-km})
    \item Menschliche Kommunikation nutzt sowohl digitale als analoge Ausdrucksformen. Digitale Kommunikation ist logisch strukturiert, aber in Beziehungssituationen bedeutungsarm. Analoge Kommunikation hingegen transportiert Beziehungsinhalte wirkungsvoll, ist jedoch weniger eindeutig in der Bedeutung (vgl. \citealp[S. 77]{Watzlawick2016-km}).
    \item \say{Zwischenmenschliche Kommunikationsabläufe sind entweder symmetrisch oder komplementär, je nachdem, ob die Beziehung zwischen den Partnern auf Gleichheit oder Ungleichheit beruht} \cite[S. 80]{Watzlawick2016-km}
\end{itemize}

\subsection{Gelingende Kommunikation nach Carl Rogers}
Dieses Modell geht einen Schritt weiter, als die zuvor behandelten Modelle. Es untersucht, wie Kommunikation zwischen den Teilnehmenden erfolgreich gelingen kann.

Rogers identifiziert drei zentrale Prinzipien, die gelingende Kommunikation ausmachen: Echtheit (Kongruenz), Empathie und Wertschätzung (bedingungslose Akzeptanz) (vgl. \citealp{jesse_carl_2025}).

Die Echtzeit beschreibt die Offenheit der Gesprächspartner und die Vermeidung von künstlichem Verhalten. Dadurch soll ein Gefühl von Sicherheit und Vertrauen entstehen. 

Die Empathie meint das Einfühlen in die Lage des Gegenübers, um dessen Gefühle und Gedanken aus dessen Perspektive zu verstehen. Dieses Verständnis förder eine tiefere emotionale Verbindung zwischen den Gesprächspartnern.

Die Wertschätzung umfasst das uneingeschränkte Akzeptieren und Wertschätzen der anderen Person, unabhängig von deren Verhalten oder Meinung. Aus dieser Wertschätzung heraus sollen sich die Beteiligten sicher und geschützt fühlen.

\section{Spielertypen}
Die kommunikationswissenschaftlichen Aspekte bilden zwar den Schwerpunkt dieser Arbeit, doch wurden auch ludologische Gesichtspunkte Berücksichtigt. Dabei handelt es sich um die wissenschaftliche Auseinandersetzung mit dem Spielen selbst (vgl. \citealp{institut_fur_ludologie_spielforschung_nodate}). 

Im Hinblick auf die Konzeption und Entwicklung eines Spiels ist es wichtig, die Eigenschaften des Spielsystems so zu gestalten, dass sie Begeisterung und Engagement bei der gewünschten Zielgruppe hervorrufen. Deshalb muss zunächst die Zielgruppe in verschiedene Typen eingeteilt werden. Die Ludologie unterscheidet hierfür verschiedene Spielertypen. Zwar ist nicht jeder Mensch ein eindeutiger \say{Spielertyp}, dennoch lassen sich unterschiedliche Spielertypen grundsätzlich über spezifische Spielelemente ansprechen (vgl. \citealp{institut_fur_ludologie_spielertypen_nodate}).

\subsection{Nach Bartle}
1996 beschäftigte sich Richard Bartle mit der Frage, welche Spielertypen in der Ludologie unterschieden werden können. Dabei ging es zunächst um Klassifizierungen von Ansätzen, die beim Spielen von sogenannten \ac{MUD}s existieren (vgl. \cite{bartle_hearts_1996}). Diese Klassifizierungen werden noch heute für die Einteilung in Spielertypen herangezogen.

Bartle unterscheidet bei der Einteilung der Spielertypen zwei grundlegende Interessen (vgl. Abbildung \ref{fig:bartle-muds}):

\begin{figure}[ht]
\centering
\includegraphics[width=1\linewidth]{content/pictures/basic_interests.PNG}
\caption{Interessen Graph nach Bartle (Quelle: \cite{bartle_hearts_1996})}
\label{fig:bartle-muds}
\end{figure}

Auf der X-Achse wird unterschieden, ob Spieler ihre Spielerfahrung über das Verhalten der anderen Mitspieler (Players) oder über die Spielwelt (World) definieren. Entlang der Y-Achse wird unterschieden, ob Spieler eher selbst aktiv Einfluss auf die Spielwelt nehmen möchten (Acting) oder ob sie in eine tiefere Interaktion mit ihr eingehen wollen (Interacting).

Die daraus resultierenden Typen sind:
\paragraph{Achiever}
Sie sind daran interessiert, auf die Spielwelt einzuwirken und alle ihnen gestellten Aufgaben mit Erfolg zu absolvieren. Ihr Status im Spiel ist ihnen wichtig - ebenso wie die Effizienz, mit der sie Fortschritte erzielen.

\paragraph{Explorer}
Sie wollen vom Spiel überrascht werden und intensiv mit der Spielwelt interagieren. Die virtuelle Welt löst ein Gefühl des Staunens aus, nach dem sie aktiv suchen. Sie sind stolz auf das Wissen, das sie im Spiel sammeln. Das erlangte Wissen möchten sie gerne an neue Spieler weitergeben.

\paragraph{Socialiser}
Sie wollen mit anderen Spielern interagieren, meist über Gespräche, aber auch durch ungewöhnliche oder kreative Verhaltensweisen. Andere Menschen kennenzulernen und mehr über sie zu erfahren, ist für sie wertvoller als für andere. Die Spielwelt dient dabei lediglich als Kulisse - entscheidend sind für sie die Begegnungen mit anderen Charaktere. Sie sind stolz auf Freundschaften, ihre Kontakte und ihren Einfluss.

\paragraph{Killer}
Sie sind daran interessiert, auf andere Spiele einzuwirken und mit ihnen zu interagieren - häufig ohne deren Einverständnis. Sie wollen ihre Überlegenheit gegenüber anderen Menschen demonstrieren und sind stolz auf ihren Ruf sowie ihre oft geübten Kampffähigkeiten.

(vgl. \cite{bartle_hearts_1996}).

\subsection{Erweiterte Einteilungen}
Bartle ist nicht der Einzige, der sich mit Spielertypen auseinandergesetzt hat. Seine Forschung bildet jedoch ein grundlegendes Fundament, das in der weiteren wissenschaftlichen Auseinandersetzung intensive Diskussionen innerhalb der Forschungs- und Game-Design-Community ausgelöst hat. 
\begin{quote}
    \textit{
        \enquote{Player types are not a defined concept and any categorization of players or users needs to occur within the context of a particular application or domain. Play-personas are suggested as a useful tool that can be used to put player type research into practice as part of the design process of gamified systems.}
    } 
    (\cite{dixon_player_nodate})
\end{quote}

\paragraph{Dixon} 
stellt Spieler-Personae vor, die analog zum \ac{UCD}-Prozess eingesetzt werden können. Dadurch muss im Designprozess nicht strikt zwischen Motivation, Verhalten und Vorlieben unterschieden werden, da Personae als ausführliche, erzählerische Darstellung gedacht sind (vgl. \cite{dixon_player_nodate}).

\paragraph{Bateman und Boon}
entwickelten in ihrer 2005 erschienenen Studie zur Bestimmung des ersten Modells des Demographic Game Design (DGD1) vier Spielstile, die sie durch die Einbeziehung des \ac{MBTI} ableiteten (vgl. \cite{noauthor_mbti_nodate}; \cite{bateman_21st_2005}).
Die vier Spielstile lauteten: Conquerer (Eroberer), Manager (Manager) Wanderer (Wanderer) und Participant (Teilnehmer).

In einer zweiten Studie wurden vier hypothetische Spielstile entwickelt, die auf einer Untersuchung von \cite{berens_understanding_2000} basierten (vgl. \cite{bateman_player_2012}). Die daraus resultierenden Stile lauteten: Logistical, Tactical, Strategic und Diplomatic.

Im Kern sind diese Modelle Weiterentwicklungen bzw. Ableitungen von Bartles ursprünglicher Typologie (vgl. \cite{ludologie_spielertypen_nodate}).

\paragraph{Yee}
Nick Yee entwickelte ein empirisch fundiertes Modell zur Beschreibung von Spielmotivationen in Online-Spielen, das bis heute einen bedeutenden Einfluss auf die Ludologie hat. Anhand eines faktorenanalytischen Ansatzes untersuchte er eine Vielzahl an Daten aus Online-Umfragen und identifizierte dabei zehn spezifische Motivationsgruppen, die sich in drei übergeordnete Hauptkategorien einordnen lassen (vgl. Abbildung \ref{fig:nick_yee_motivations}):

\begin{figure}[ht]
\centering
\includegraphics[width=1\linewidth]{content/pictures/nick_yee_categorizations.PNG}
\caption{Motivationsgruppen nach Nick Yee (Quelle: \cite{yee_motivations_2006})}
\label{fig:nick_yee_motivations}
\end{figure}

Die Achievement-Komponente umfasst den Fortschritt im Spiel, sowie das damit einhergehende Verlangen Macht zu erlangen, schnell voranzukommen und Symbole für Reichtum oder Status zu erwerben. Zudem besteht ein Interesse daran, die Spielmechanik zu analysieren, die Regeln und Systeme zu verstehen um die Leistung der Spielfigur zu optimieren. Auch ist der Wettbewerb spielt eine zentrale Rolle: Es besteht der Wunsch, sich mit anderen zu messen und gegen sie anzutreten.

Die soziale Komponente beschreibt das Bedürfnis nach Sozialisierung. Spieler haben Interesse daran, anderen zu helfen und sich mit ihnen zu unterhalten. Daraus entstehen Beziehungen, bei denen der Wunsch besteht, langfristige und bedeutungsvolle Bindungen zu anderen aufzubauen. Teamarbeit ist dabei gewünscht, um gemeinsame Ziele zu erreichen oder sich im Wettbewerb zu behaupten.

Die Immersion-Komponente beschreibt das Entdecken der Spielwelt und das damit verbundene Finden von Objekten sowie das Erlangen von Wissen, das den meisten anderen Spielern unbekannt ist. Rollenspiel-Elemente sind besonders wichtig, um den Spielfiguren Hintergrundgeschichten zu geben und gemeinsam improvisierte Erzählungen zu entwickeln. Der Spielavatar sollte zudem anpassbar sein, damit persönliche Vorlieben und der individueller Stil der Spieler zum Ausdruck kommen können. Die Spielwelt dient auch als Mittel um dem Alltag zu entfliehen und den Problemen der realen Welt zu entkommen.

\paragraph{weitere Modelle}
Im Zuge der fortschreitenden Forschungen entstanden weitere Modelle wie zum Beispiel das Gamer Motivation Model, das auf Basis der Forschung von Nick Yee entwickelt wurde (vgl. \cite{ludologie_spielertypen_nodate}):

\begin{figure}[ht]
\centering
\includegraphics[width=1\linewidth]{content/pictures/gamer_motivations_model.png}
\caption{Gamer Motivation Model der QUANTIC FOUNDRY (Quelle: \cite{noauthor_quantic_nodate})}
\label{fig:gamer_motivation_model}
\end{figure}

Ein weiteres Modell, das in der Arbeit von Bateman genannt wird, ist das BRAINHEX-Model, bei dem die verschiedenen Spielertypen in hexagonaler Anordnung platziert sind (vgl. Abbildung: \ref{fig:brain-hex}):

\begin{figure}[ht]
\centering
\includegraphics[width=1\linewidth]{content/pictures/brainhex-classes.png}
\caption{Brainhex-Model Darstellung von \cite{noauthor_i_nodate} nach \cite{nacke_brainhex_2013}}
\label{fig:brain-hex}
\end{figure}

\section{Multiplayer-Spiele}
Im Vergleich zu Einzelspieler-Spielen existieren bei Multiplayer-Spielen nicht nur Unterschiede im Genre, sondern auch in den Spielrollen (Symmetrie / Asymmetrie) sowie in den Spielzeitpunkten, zu denen die Spielteilnehmer an ihrem Spielfortschritt weiterarbeiten (Synchron / Asynchron). [Hier wäre eine Quelle noch gut]. Auf dem Spielemarkt existieren außerdem Multiplayer-Spiele, die unterschiedliche Medientechniken verwenden. Teilweise dienen diese Medientechniken dazu, Cross-Plattform Funktionalität zu gewährleisten (vgl. \cite{noauthor_baldurs_nodate}), oder sie sind integrale Bestandteil des Gamedesigns (vgl. \cite{noauthor_keep_nodate}).

Da im Kontext von \say{Connecting-Minds} die Spieler zeitgleich in einer Sitzung gemeinsam spielen, wird im Folgenden auf die Symmetrie und Asymmetrie von Computer- und Videospielen eingegangen.

\subsection{Symmetrische Multiplayer}
Symmetrische Spiele sind solche, bei denen alle Spieler die gleichen Spielregeln haben und das gleiche Spielziel verfolgen. Viele traditionelle Spiele wie Schach sowie Computer- und Videospiele wie \say{Mario Kart} oder \say{Minecraft} sind symmetrische Multiplayer-Spiele, bei denen für jeden Spieler das gleiche Ziel gilt (vgl. \cite[S. 12]{adams_fundamentals_2013}); (vgl. \cite{nintendo_mario_1992}); (vgl. \cite{noauthor_willkommen_2009}). 


\subsection{Asymmetrische Multiplayer}
Asymmetrische Spiele hingegen können unterschiedlichen Spielern unterschiedliche Regeln zugestehen und ggf. verfolgen die Spieler unterschiedliche Ziele (vgl. \cite[S. 12]{adams_fundamentals_2013}). Sie sind sowohl in kooperativen als auch kompetitiven Spielen weit verbreitet und werden bspw. in Form verschiedener \say{Helden} oder \say{Klassen} umgesetzt. So gibt es z.B. in \say{Overwatch} oder \say{League of Legends}  unterschiedliche \say{Support}-Charaktere, deren Aufgabe es ist das Team zu heilen (vgl. \cite{smilovitch_birdquestvr_2019}); (vgl. \cite{noauthor_league_2025}); (vgl. \cite{noauthor_overwatch_nodate}). 
Außerdem ermöglichen asymmetrische Spiele, dass Spieler mit unterschiedlichen Fähigkeiten und Fähigkeitsniveaus gemeinsam spielen können. Ein asymmetrisches Design kann zudem die Inklusivität in Spielen fördern (vgl. \cite{smilovitch_birdquestvr_2019}).

\subsection{Hybride Multiplayer}
Wie Lotz in ihrer Bachelor-Arbeit beschrieben hat, unterscheiden sich Multiplayer auch in der verwendeten Medientechnik (vgl. \cite[S. 6f]{lotz_konzeption_2021}). Sogenannte hybride Spiele wie \say{New Super Mario Bros U} (vgl. \cite{noauthor_mario_nodate-1})

\begin{figure}[ht]
\centering
\includegraphics[width=1\linewidth]{content/pictures/lotz_hybrid_multiplayer.PNG}
\caption{Unterscheidung Multiplayertypen (Quelle: \cite[S.6]{lotz_konzeption_2021})}
\label{fig:lotz_multiplayer_types}
\end{figure}

Wie Abbildung \ref{fig:lotz_multiplayer_types} zeigt, können Multiplayer-Spiele hinsichtlich ihrer Medientechnik in drei Kategorien eingeteilt werden.
Spiele wie \say{Mario Kart} oder \say{Minecraft} können nur auf der gleiche Plattform gespielt werden. Bei Spielen wie \say{Among Us} oder \say{Fortnite} ist die Plattform, auf der gespielt wird, nicht relevant, da eine Cross-Play-Funktionalität gegeben ist. Jeder kann mit Spielern auf der Plattform spielen, die er zu zuhause hat. Die dritte Kategorie umfasst Spiele, bei denen die Spieler gezwungen werden, unterschiedliche Plattformen zu nutzen. In \say{Keep talking and nobody explodes} ist dies der Kern des Gamedesigns.

\section{Spielweisen von Multiplayer-Spielen}
Nachdem die unterschiedlichen Strukturen und technischen Formen von Multiplayer-Spielen behandelt wurden, ist es nun wichtig, die verschiedenen Spielweisen zu betrachten. Multiplayer-Spiele können dabei in drei Hauptspielformen unterteilen:
In kompetitive, kollaborative und kooperative Spielweisen (vgl. \cite[S. 25f]{zagal_collaborative_2006}).

\subsection{Kompetitiv}
Kompetitive Spiele sind solche, bei denen Spieler oder Teams gegeneinander antreten, um ein bestimmtes Ziel zu erreichen, wobei der Erfolg des einen oft den Misserfolg des anderen bedeutet. In diesen Spielen ist das Spiel selbst neutral und agiert nicht aktiv im Wettbewerb (vgl. \cite{noauthor_game_2014}), (vgl. \cite[S. 25]{zagal_collaborative_2006}).

\subsection{Kollaborativ}
Kollaborative Spiele sind solche, bei denen alle Spieler - ähnlich wie in Kooperationsspielen - gemeinsam gegen das Spiel verlieren können. Allerdings können sie nicht gemeinsam gewinnen. Diese Spiele sind im Kern meist kompetitiv, beinhalten jedoch die Möglichkeit einer kollektiven Niederlage. Die Spieler müssen zu einem gewissen Maß zusammenarbeiten, um nicht zu verlieren (vgl. \cite{noauthor_game_2014}), (vgl. \cite[S. 25]{zagal_collaborative_2006}).

\subsection{Kooperativ}
Bei Kooperationsspielen ist es möglich, dass alle Spieler gemeinsam gegen das Spiel verlieren oder gemeinsam gewinnen können. Ein Sieg wird erreicht, wenn das Spiel gemeinsam \say{besiegt} wird oder dadurch dass ein festgelegtes Ziel kollektiv oder individuell erreicht werden kann (vgl. \cite{noauthor_game_2014}), (vgl. \cite[S. 25]{zagal_collaborative_2006}).


\section{Netzwerkinfrastrukturen}\label{sec:basics-network-structures}
Um ein Multiplayer-Spiel entwickeln zu können, muss zunächst geklärt werden, wie die Netzwerkinfrastruktur der Anwendung aufgebaut sein soll. Es existieren zahlreiche Ansätze, die jeweils für verschiedene Anwendungszwecke konzipiert sind.

\subsection{Distributed Authority}
Bei einer \say{Distributed Authority}-Netzwerktopologie übernimmt jeder im Netzwerk verbundene Spielclient gemeinsam jeweils die Verantwortung für das Erstellen und Verwalten von Objekten im Netzwerk. Jeder Client simuliert dabei seinen Teil der Spielwelt selbst und steuert Objekte über die er Autorität besitzt.
Damit Positionen und andere relevanten Daten an alle anderen Clients im Netzwerk weitergeleitet werden können, wird ein zentraler, leichtgewichtiger Statusdienst verwendet, der ausschließlich für die Verteilung der notwendigen Informationen zuständig ist, ohne selbst die Anwendung zu simulieren (vgl. \cite{noauthor_distributed_2025}).

\begin{figure}[ht]
\centering
\includegraphics[width=1\linewidth]{content/pictures/distributed-authority-service.jpg}
\caption{Netzwerktopologie der Distributed Authority (Quelle: \cite{noauthor_distributed_2025})}
\label{fig:distributed_authority_topology}
\end{figure}

Spiele wie \say{Journey}, \say{God of War: Ascension}, \say{Mercenaries 2}, \say{GTA: Online}, \say{Dark Souls} und \say{Destiny} nutzen diese Netzwerkinfrastruktur. Häufig kommt diese Topologie zum Einsatz, wenn ein bestehendes Single-Player um eine Multiplayer-Komponente erweitert werden soll (Journey, GTA und Dark Souls), ohne den Kern des Quellcodes grundlegend umzustrukturieren. Diese Architektur erfordert keinen dedizierten Server, eignet sich für Spiele mit großen, offenen Spielwelten (Dark Souls, GTA) und kommt häufig zum Einsatz, wenn keine deterministische Physik benötigt wird bzw. kein vollständig deterministisches Spielkonzept vorliegt. Sie eignet sich zudem besonders für Spiele, bei denen die Prozessorleistung (z.B. durch Physiksimulationen) stark beansprucht wird. Für Spiele mit kooperativen Spielmechaniken, leichten kompetitiven Elementen oder innovativen Multiplayer-Ideen ist diese Infrastruktur eine sinnvolle Wahl (vgl. \cite{noauthor_choosing_2024}).

\subsection{Pure Client/Server}
Bei der Client-Server-Architektur übernimmt ein zentraler Server die Hauptsimulation und verwaltet alle wesentlichen Aspekte des Spiels. Dazu gehören unter anderem die Physiksimulation, das Erzeugen und Entfernen von Objekten sowie die Autorisierung von Anfragen der Clients. Aus Sicht der Clients besitzen diese lediglich die Anwendung, über die sie sich mit dem Server verbinden. Sie erhalten über diese Verbindung die Darstellung des Spiels (vgl. \cite{noauthor_client-server_2024}):
\paragraph{Ein dedizierter Server} bildet eine eigenständige Instanz, die ausschließlich dem Spielbetrieb dient (vgl. Abbildung \ref{fig:dedicated_server}).

\begin{figure}[ht]
\centering
\includegraphics[width=1\linewidth]{content/pictures/ded_server-d5369721966357b9b4d5e1fa96b05b22.png}
\caption{Client-Server-Architektur mit dediziertem Server (Quelle: \cite{noauthor_network_2024})}
\label{fig:dedicated_server}
\end{figure}

\paragraph{Ein Client gehosteter Server} läuft auf demselben Gerät wie die dazugehörige Client-Anwendung (vgl. Abbildung \ref{fig:client_server}).

\begin{figure}[ht]
\centering
\includegraphics[width=1\linewidth]{content/pictures/client-hosted-16be0b1c9b5020f21325b1e6a7beca73.png}
\caption{Client hosted Server (Quelle: \cite{noauthor_network_2024})}
\label{fig:client_server}
\end{figure}

\subsection{Peer-to-Peer}
Das \ac{P2P}-Architektur-Modell verbindet jeden Spieler direkt mit allen anderen. Über diese Verbindungen werden Daten zu Spielzuständen und Ereignissen ausgetauscht. Im \say{reinen} \ac{P2P}-System gibt es keinen zentralen \say{Host}. Stattdessen ist jeder Client dafür verantwortlich, seinen eigenen Avatar (oder seine Einheiten) zu verwalten und erhält gleichzeitig Updates von den anderen Clients (vgl. \cite{mygames_unity_2024}). Abbildung \ref{fig:p-2-p} zeigt die entsprechende Topologie.

\begin{figure}[ht]
\centering
\includegraphics[width=1\linewidth]{content/pictures/0_poGQC2fWQ3tPWPwT.png}
\caption{Peer-to-Peer Infrastruktur (Quelle: \cite{mygames_unity_2024})}
\label{fig:p-2-p}
\end{figure}


\subsection{Relay-Server}
Der Relay-Dienst ermöglicht Multiplayer-Unterstützung ohne die Notwenigkeit eines dedizierten Spielserver. Dabei wird die Kommunikation zwischen den Spielern über sogenannte Relay-Server weitergeleitet. Nachrichten werden mithilfe einer latenzarmen Datagramm-Übertragung übermittelt, sodass keine direkte Verbindung zwischen den einzelnen Spielern erforderlich ist (vgl. \cite{noauthor_relay_nodate}).

\begin{figure}[ht]
\centering
\includegraphics[width=1\linewidth]{content/pictures/0_o7LJU1ImxPHIM5Ej.png}
\caption{Relay-Server Infrastruktur (Quelle: \cite{mygames_unity_2024})}
\label{fig:relay-server}
\end{figure}

\section{Augmented Reality}

\ac{AR} stellt eine Form virtueller Umgebungen (\ac{VE}) dar, bei der im Gegensatz zur vollständigen Immersion in rein virtuelle Welten, wie es in \ac{VR} der Fall ist, die Umgebung weiterhin sichtbar bleibt. Virtuelle Objekte werden dabei über die physische Welt gelegt und mit ihr kombiniert, sodass eine erweiterte Wahrnehmung entsteht. Es gewährleistet außerdem eine Interaktion mit den räumlich (\ac{3D}) registrierten virtuellen Inhalten in Echtzeit. Die technische Umsetzung kann dabei über monitorbasierte Schnittstellen, monokulare Systeme oder durchsichtige \ac{HMD}s erfolgen (vgl. \cite[S. 2f]{azuma_survey_1997}).

Im Vergleich zur \ac{VR}, das eine voll-immersive Umgebung erstellt, erweitert die \ac{AR} die eigene Realität und bietet dadurch die Möglichkeit, dass sich die Nutzer frei in der Welt bewegen können und auch immer noch mit der echten Welt interagieren können (vgl. \citealp[S. 79]{billinghurst_survey_2015}; \citealp[S. 1]{stefanidi_meaningful_2024}).

\subsection{Abgrenzung der Begriffe}
In der Vergangenheit und aktuell wird in der wissenschaftlichen Forschung nach Konzepten und Möglichkeiten der Integration und Verbesserung von \ac{AR} und \ac{VR} gesucht. Damit zwischen diesen Begriffen und den weiteren in der Forschung angewandten Begriffen \ac{MR} oder \ac{XR} unterschieden werden kann, wurde von  \cite{milgram_taxonomy_1994} das Konzept des \say{\ac{VC}s} entworfen (vgl. Abbildung \ref{fig:virtuality-continuum}).

\begin{figure}[ht]
\centering
\includegraphics[width=1\linewidth]{content/pictures/virtuality-continuum_upscaled.PNG}
\caption{Vereinfachte Darstellung eines "Virtuality Continuums" (Quelle: \citealp[S. 9]{knoll_augmented_2022}; Modifiziert nach \citealp[S. 3]{milgram_taxonomy_1994})}
\label{fig:virtuality-continuum}
\end{figure}

Auf der linken Seite des \ac{VC}s befindet sich das reale Umfeld (\ac{PR}). Auf der entgegenfliegenden rechten Seite liegt die virtuelle Umgebung (\ac{VR}). Je weiter von der Seite der \ac{VR} nach links in die Richtung der \ac{PR} entgegen der Richtung der Virtualität gegangenen wird, desto häufiger werden virtuelle Elemente über die reale Welt gelegt. Die \ac{PR} besteht dabei ausschließlich aus realen Objekten. Durch die \ac{AR} können nun virtuelle Objekte über die reale Welt überlagert und erweitert werden. Bei der \ac{AV} befindet sich der Benutzer in einer virtuellen Umgebung, bei der etwas aus der realen Welt hinzugefügt wird. In der \ac{VR} befindet sich der Nutzer in einer vollständig künstlich geschaffenen Umgehung, in der er mit der Umgebung interagieren kann (vgl. \citealp[S. 3]{milgram_taxonomy_1994}; \citealp[S. 8f]{knoll_augmented_2022}; \citealp[S. 3]{zuniga_gonzalez_making_2021}). 

Die \ac{XR} umfasst alle Ausprägungen, bei denen virtuelle Elemente eingesetzt werden. Die \ac{MR} umfasst die Bereiche, in denen die virtuelle Welt die physische Welt überlagern, oder die physische Welt in der virtuellen Welt sichtbar ist. 
% Die \ac{MR} umfasst nun die Gebiete der \ac{AR} und \ac{AV}. 

\subsection{Technologische Umsetzungen}
% Die virtuelle Welt kann über verschiedene Technologien 
Die Einbettung der virtuellen Elemente in die physische Umgebung kann über verschiedene Techniken erfolgen. Die bekanntesten davon sind mitunter mobile Displays (Smartphone-Bildschirme oder Tablets), \ac{HMD}s oder Projektionssysteme. Im folgenden werden diese unterschiedlichen Technologien vorgestellt.
\begin{figure}[ht]
\centering
\includegraphics[width=1\linewidth]{content/pictures/devices.PNG}
\caption{Klassifizierungen von Augmented Reality Displays (Quelle: \citealp[S. 315]{leins_comparing_2024})}
\label{fig:ar-classes}
\end{figure}
\paragraph{Handheld Augmented Reality}
% In einer Hand haltbare Geräte eignen  sich gut für \ac{AR}-Anwendungen. Die Derzeitige Hardware haben 
\begin{figure}[ht]
\centering
\includegraphics[width=0.5\linewidth]{content/pictures/handheld-ar.PNG}
\caption{Handheld-AR Display (Quelle: \citealp[S. 318]{leins_comparing_2024})}
\label{fig:handheld-ar}
\end{figure}

\ac{HAR} erfolgt über Handheld-Displays, die der Benutzer in der Hand halten kann (vgl. Abbildung \ref{fig:handheld-ar}). Sie verwenden Video see-through Techniken, um virtuelle Objekte über die reale Umgebung zu legen. Um dies zu ermöglichen, nutzen sie Sensoren, wie digitale Kompasse und GPS-Geräte ein um ihre Ausrichtung und Pose zu bestimmen (vgl. \citealp[S. 347]{carmigniani_augmented_2011}).

\paragraph{Head-Mounted-Devices}

\begin{figure}[ht]
\centering
\includegraphics[width=0.5\linewidth]{content/pictures/hmd-ar.PNG}
\caption{Head-Mounted-Display (Quelle: \citealp[S. 4]{reitmayr_location_2003})}
\label{fig:hmd-ar}
\end{figure}

Bei einem \ac{HMD} trägt der Anwender ein Displaygerät auf dem Kopf oder als Teil eines Helmes (vgl. Abbildung \ref{fig:hmd-ar}), das ihm sowohl Bilder der realen als auch virtuellen Umgebung über das Sichtfeld legt. Sie können entweder video-see-through oder optisches see-through oder über monokulare- bzw. binokulare Displayoptiken verfügen (vgl. \citealp[S. 346]{carmigniani_augmented_2011}).

Video see-through Systeme setzen voraus, dass der Nutzer zwei Kameras auf dem Kopf trägt, und die Bilder aus beiden Perspektiven verarbeitet werden. Dabei werden die Bilder der Kameras (der \say{reale Teil}) zusammen mit der virtuellen Szene und der enthaltenen virtuellen Objekte verrechnet. Dieser Ansatz benötigt viel Rechenleistung (bsp.: \cite{noauthor_vive_nodate}). Optische see-through Systeme hingegen benötigen nicht so viel Rechenleistung, da eine Halbspiegeltechnologie verwendet wird, um die Sicht auf die reale Umgebung zu ermöglichen und virtuelle Elemente grafisch zu überlagern (vgl. \citealp[S. 346f]{carmigniani_augmented_2011}).

\paragraph{Spatial Augmented Reality}
\begin{figure}[ht]
\centering
\includegraphics[width=0.5\linewidth]{content/pictures/spatial-ar.PNG}
\caption{Spatial Augmented Reality (Quelle: \citealp[S. 7]{jin_bim-based_2020})}
\label{fig:spatial-ar}
\end{figure}

\ac{SAR} nutzt Videoprojektoren und weitere Tracking-Technologien um grafische Informationen direkt auf physische Objekte zu projizieren, ohne dass der Benutzer ein Display tragen oder führen muss (vgl. Abbildung \ref{fig:spatial-ar}). Außerdem trennen sie den Großteil der Technologie vom Benutzer und integrieren sie in die Umgebung (vgl. \citealp[S. 348]{carmigniani_augmented_2011}).

Es gibt drei verschiedene \ac{SAR}-Ansätze. Beim Bildschirmbasiertem Video see-through wird kostengünstige handelsübliche Hardware benutzt. Allerdings ist dieser Ansatz nur für stationäre Anwendungen geeignet. Räumliche optische see-through Bildschirme nutzen optische Komponenten wie Spiegelstrahler oder transparente Bildschirme, um Bilder direkt im Raum auszurichten. Sie sind jedoch ebenfalls nicht mobil einsetzbar. Projektorbasierte Displays projizieren Bilder direkt auf reale Oberflächen und ermöglichen so eine nahtlose Integration virtueller Inhalte in die physische Umgebung (vgl. \citealp[S. 348]{carmigniani_augmented_2011}).

\subsection{Platzierungsmethoden virtueller Objekte}
Damit die virtuellen Objekte in ihrer virtuellen Umgebung die reale Umgebung erweitern können, müssen diese an vorgegebenen oder gewünschten Positionen platziert werden. Um das zu erreichen gibt es verschiedene Platzierungsmethoden (vgl. Abbildung \ref{fig:placement-ar}).

\begin{figure}[ht]
\centering
\includegraphics[width=1\linewidth]{content/pictures/placement-methods.PNG}
\caption{Verschiedene Platzierungsmethoden in Augmented Reality (Quelle: \citealp[S. 3]{el_barhoumi_assessment_2022})}
\label{fig:placement-ar}
\end{figure}

Die gängigste Methode ist die Markerbasierte, bei dem Anhand eines bestimmten Markers virtuelle Objekte platziert werden. Dabei wird zwischen den Hyperlink Methoden, bei denen physische Objekte über grafische Tags oder automatischen Identifikationstechnologien mit webbasierten Inhalten verknüpft werden, und den Kamerabasierten Methoden, bei denen über die Auswertung der Kamerabilder gedruckte Muster oder reale Objekte erkannt werden und anhand dieser virtuelle Objekte platziert werden können (vgl. \citealp[S. 3f]{el_barhoumi_assessment_2022}). 

Der Markerlose Ansatz lässt sich über die sensorbasierten und kamerabasierten Methoden realisieren. Beim sensorbasierten Ansatz werden die intrinsischen Sensoren des Gerätes verwendet um die Position und Orientierung der Kamera zu bestimmen. Dabei kommen der Gyro und Beschleunigungssensor, sowie das GPS zum Einsatz. Der modellbasierte Kamera-Ansatz benötigt meist ein \ac{3D}-Modell der Umgebung. Über das Kanten- und Featurebasierte Tracking werden dabei einzelne Abschnitte, Kanten und Merkmale aus den Kamerabildern mit der Vorlage verglichen. Beim Template-Vergleich werden kleinere Bildausschnitte mit Datenbankbildern abgeglichen und beim Tiefenbildverfahren werden zusätzlich zu den Farbinformationen aus Bildern Tiefendaten verwendet, um Absände des Raumes zu erfassen.
Beim nicht modellbasiertem Ansatz wird kein vordefiniertes Modell benötigt, stattdessen erstellt das Verfahren gleichzeitig ein \ac{3D}-Rekonstruktion der Umgebung und verfolgt die Kamerabewegung (vgl. \citealp[S. 4f]{el_barhoumi_assessment_2022}).

