\chapter{Konzeption und Aufbau des Prototyps}
% In diesem Kapitel wird die Konzeption des Prototyps vorgestellt.
% % Sie enthält zunächst das methodische Vorgehen des Abschnitts, im Anschluss folgt eine Analyse der Spiele, die im Fokus der Arbeit stehen und aus welchen Aspekte in die Konzeption eingeflossen sind. Der Hauptteil der Konzeption umfassen das Game-Design des Prototyps.

Nachdem durch die Literaturrecherche in Kapitel \ref{sec:related-works} wichtige Funktionalitäten im Design identifiziert werden konnten und vergleichbare Spiele in Kapitel \ref{sec:analysis} auf ihr Game- und Rätseldesign analysiert wurden, kann aus den Ergebnissen \say{Connecting-Minds} nun vollständig konzipiert werden.

Die Konzeption dieses Spiels folgt einem methodischen Vorgehen: Zunächst wird die übergeordnete Designvision und Zielsetzung erläutert. Anschließend dient das \ac{MDA}-Framework (vgl. \cite{hunicke_mda_2004}) als zentrales Analyse- und Strukturierungsmittel, um das Spielerlebnis gezielt zu gestalten. Auf Basis des \ac{MDA}-Frameworks werden die grundlegenden Spielmechaniken, Rollenverteilungen und dynamischen Abläufe beschrieben.

\section{Designziele und Zielgruppe}
\say{Connecting-Minds} verfolgt das Ziel, kooperative Kommunikation unter ungleichen Perspektiven in einem Escape-Room-ähnlichen Szenario zu verbessern. Dabei existieren zwei verschiedene Rollen - Player und Watcher - die gemeinsam die Rätsel lösen und Hindernisse überwinden müssen. Die Wahrnehmung der Spielwelt und Handlungsmöglichkeiten ist in beiden Anwendungen unterschiedlich, gleichzeitig ergänzt sie sich aber. 

Ziel ist es, durch asymmetrische Informationsverteilung eine Spannung zwischen Orientierung und Vertrauen aufzubauen und gleichzeitig den gemeinsamen Fortschritt in den Mittelpunkt stellen.

% \section{Genre}

% \section{Spielmechanik}

% \section{Spielablauf}

% \subsection{Spielablauf des Spiels}

% \subsection{Levelablauf}

% \section{Session}

% % \section{Belohnungen}

% \section{Spielerrollen}

% \subsection{Player}

% % \subsubsection{Interaktion mit Gegenständen}

% % \subsubsection{Tragen von Gegenständen}

% \subsection{Watcher}

% % \subsubsection{Platzieren von Gegenständen}

% % \subsubsection{Entfernen von Gegenständen}

% % \subsubsection{Previewen von Gegenständen}

% \section{Gegenstände}

% \subsection{Leichte Gegenstände}

% \subsection{Schwere Gegenstände}

% \subsection{Hinweise}

% \section{Leveldesign}

% \subsection{Gegenstände}

% \subsection{Hinweise}

% \subsection{Hindernisse}

% \section{Informationen für den Spieler}

% \section{Sounddesign}

% \chapter{Visuelles Design des Prototyps}

% \section{Moodboard}

% \section{Art-Stil}

% \section{Avatar des Players}

% \section{User Interface}

% \section{Führung durch das Level}

% \section{Gegenstände}

% \section{Menü}

% \section{Leveldesign}