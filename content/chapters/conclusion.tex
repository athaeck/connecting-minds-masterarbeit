\chapter{Fazit}

Ziel dieser Masterarbeit war es, mit Connecting-Minds einen Spiel-basierten Prototypen zu entwickeln, der als Versuchsumgebung zur Untersuchung kommunikativer Prozesse in asymmetrischen Multiplayer-Szenarien dient. Die zentrale Forschungsfrage lautet, inwiefern durch ein solches Spielkonzept die Kommunikation zwischen zwei Personen, insbesondere im Umgang mit zunächst fremden Personen, verbessert werden kann.

Die Analyse der quantitativen Daten zeigt, dass der Prototyp Potenzial besitzt, um soziale Nähe zwischen Spielpartnern zu fördern. Zwar konnten aufgrund der geringen Stichprobengröße keine signifikanten Veränderungen im Kommunikationsverhalten nachgewiesen werden, jedoch deuten einzelne Effekte, wie etwa die mittleren Korrelationen bei Gesprächsinitiativen oder der Pausenzeiten, auf erste positive Tendenzen hin. Darüber hinaus wurde das Spielkonzept von den Probanden überwiegend positiv bewertet, insbesondere in Bezug auf auf seine Originalität und motivierende Wirkung.

Gleichzeitig wurden auch Schwächen offensichtlich. Vor allem die Bedienbarkeit der Watcher-Anwendung wurde vielfach kritisch bewertet. Eine klarere Trennung der Gestensteuerung (Zoom vs. Rotation) sowie eine konsistente UI-Gestaltung stellen zentrale Ansatzpunkte für zukünftige Iterationen dar. Auch konzeptionell wurden einige Rätsel als nicht vollständig durchdacht empfunden und sollten überarbeitet werden, um das das interdependente Zusammenspiel der beiden Rollen stärker herauszuarbeiten.

\section{Ausblick}\label{sec:prospect}

Diese Arbeit legt die Grundlage für weitere Forschungen im Bereich kooperativer Kommunikation durch asymmetrische Multiplayer Adventure-Spiele.

Aufbauend auf den gewonnenen Erkenntnissen sollten zukünftige Studien eine größere Stichprobenzahl einbeziehen, um eine größere Anzahl statistisch signifikanten Aussagen treffen zu können. Darüber hinaus wäre es sinnvoll, Zwischenmessungen zu integrieren, um Entwicklungsverläufe differenzierter erfassen und interpretieren zu können. Ergänzend könnten noch weitere Erhebungsinstrumente, etwa Fragebögen zur wahrgenommenen Interdependenz oder der qualitativen Usability, eingesetzt werden, um den Prototyp gezielter weiterentwickeln zu können. Auch das bestehende Rätsel- und Interface-Design bietet Potenzial für iterative Weiterentwicklungen, insbesondere im Hinblick auf Benutzerfreundlichkeit, Rollenbalance und die spielmechanische Unterstützung kooperativer Zusammenarbeit.

Das Konzept Connecting-Minds bietet nicht nur eine spannende Möglichkeit das Kommunikationsverhalten von Dyaden zu verbessern, sondern auch einen allgemeinen Spielspaß mit einer neuartigen Spielmechanik. Wenn es gelingt, die gebrauchstauglichen Aspekte zu verbessern und die Konzeption final umzusetzen, steht dem Spiel nichts im Wege, es der breiten Öffentlichkeit zugänglich zu machen.