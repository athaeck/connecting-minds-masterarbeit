\chapter{Einleitung}
% Einleitung über Themen von multiplayern, zusammen spielen, vl corona noch, dann darüber dass es wenig asymmetrische multiplayer gibt, irgendwie steam lib durchgehen oder andere Listen noch, verweise auf crossplattform noch geben da es ähnlich wäre.

% Einsamkeitsstudien von corona zeigen und auflisten von studien or Papern bei denen es darum geht, dass die Spiele spaßig sind und kommunikationsfördernd und social engagement steigernd sind um sowas wie corona entgegenzuwirken.

% Kurzer Überblick über den Beginn der Arbeit; enthält den Interaktionsdesignworkshop
% Titel des Projektes

[TODO: Wissenschaftliche Einleitung zu Themen wie Corona, bestehenden Online Games usw geben]

Ein erstes Konzept, um auf diese Probleme einzugehen, wurde im Rahmen der Veranstaltung Interaktionsdesign im 1. Mastersemester der Studiengänge \ac{MIM} und \ac{DIM} im Wintersemester 2023/2024 entwickelt und umgesetzt. 
Aufgrund der interessanten Rätselmechanik, des positiven Feedbacks bei den in der Projektausarbeitung enthaltenen Probandentests und des interessanten Forschungsgebiets wurde beschlossen, dieses Projekt weiterzuführen und die konzeptionellen Grundideen zu implementieren.

Der Titel des Projektes lautet \say{\emph{Connecting-Minds}}.

\section{Motivation und Aufgabenstellung}
Das Ziel dieser Masterarbeit ist es, den bisherigen Prototyp, der ein Mindestmaß an Funktionen des Konzepts enthielt, auf technischer Ebene neu zu entwickeln. Zusätzlich dazu soll der Prototyp als Versuchsumgebung dienen, um Effekte auf das Kommunikationsverhalten der Spieler zu erforschen. 

Die folgenden Forschungsfragen bilden das Grundgerüst dieser Abschlussarbeit:


\begin{itemize}
    \item \textbf{Kann eine spielbasierte Umgebung für die Untersuchung und Verbesserung von Kommunikation zwischen zwei oder mehreren Personen realisiert werden?}
    \item \textbf{Welche spezifischen Eigenschaften muss eine solche Umgebung aufweisen und welche Kommunikationsparameter werden dabei angesprochen?}
    \item \textbf{Welche Verbesserungen in der Kommunikation zwischen den Anwendern können durch ein asynchrones Multiplayer-Spiel mit zwei verschiedenen Spielerklassen beobachtet werden?}
    \item \textbf{Welche Unterschiede können in der Art das Kommunikationsverhalten bei der Verwendung von zwei unterschiedlichen Anwendungen (AR und 3D) (festgestellt/beobachtet) werden}
    \item \textbf{Wie stehen die Nutzer zu einem spielerischen Ansatz und zur Verbesserung der Kommunikation, insbesondere auch im Umgang mit Fremden?}
\end{itemize}

\section{Struktur der Arbeit}

[TODO: Zusammenfassen in welchen Kapitel sich was befindet]