\chapter{Stand der Forschung}\label{sec:related-works}

Im Rahmen der \ac{CSCW} sowie der \ac{HCI} bzw. \ac{CHI}, existiert eine Vielzahl an Arbeiten, die sich mit Multiplayer-Spielen, deren Einfluss auf das soziale Miteinander sowie den relevanten Gestaltungsaspekten im Gamedesign beschäftigen.

Im Mittelpunkt steht dabei häufig die Frage, wie Spiele soziale Interaktionen fördern oder hemmen können - sowohl in kompetitiven als auch kollaborativen Kontexten.

Video- und Computerspiele im Allgemeinen können einen positiven Einfluss auf das Miteinander haben. So untersuchte \cite{mason_friends_2013} wie wichtig Freundschaften für den Erfolg von Einzelpersonen und Teams in komplexen kollaborativen Umgebungen sind. Sie fanden heraus, dass Freundschaften einen großen Einfluss auf die verbesserte individuelle- und Teamleistungen haben. Spieler richten sich dabei nach sozialen Gelegenheiten aus, sodass verborgene Freundschaftsbeziehungen direkt abgeleitet werden konnten. Kern der Studie war dabei der Online Multiplayer First-Person-Shooter \say{Halo: Reach} bei dem Spieler des Spiels eine anonyme Online Umfrage ausfüllen mussten. 

Doch soziale Dynamiken verlaufen nicht immer so positiv wie erhofft. Andere Untersuchungen zeigen ein differenzierteres Bild des Zusammenspiels in Onlinewelten.

So argumentiert \cite{ducheneaut_alone_2006} anhand einer Langzeitstudie zu \say{World of Warcraft}, dass soziale Aktivitäten in \ac{MMOG}s, oft überschätzt werden. Die meisten Spieler sind zwar von anderen umgeben, interagieren jedoch nur selten aktiv miteinander. Sie spielen häufig \say{allein zusammen}. Vor allem in den Quests am Anfang des Spiels ist das oft der Fall. Erst durch langfristige soziale Strukturen wie Gilden entstehen nachhaltige Bindungen und echte Zusammenarbeit.

Damit jedoch solche sozialen Beziehungen überhaupt entstehen können, ist es essenziell, dass Spiele die Aufmerksamkeit und das Interesse der Spielenden wecken. Ein Aspekt, der unter dem Begriff \say{Player Engagement} intensiv erforscht wird.

\cite{rashed_review_2025} fassen in ihrer Überblickarbeit unterschiedliche Methoden zur Schätzung des Spieler-Engagements zusammen. Ihr Ziel war es, über verschiedene Messmethoden wie EEG, Mimik, Eye Tracking und Spieler-Verhalten hinreichend eindeutige Daten zu sammeln um darüber eine Aussage über das Engagement treffen zu können. Die Validierung der Ergebnisse (da das Engagement subjektiv), ist schwer um objektiv eine \say{Ground Truth} Aussage treffen zu können. \cite{yu_video_2023} verfolgten einen anderen Ansatz. Sie versuchten nicht nur auf das Engagement der Spieler einzugehen, sondern erforschten direkt im Bereich der Entwicklung von Zusammenarbeit- und Kooperationsfähigkeiten. Sie untersuchten kommerzielle Multiplayer-Spiele um Konzepte und Spielmechaniken zu identifizieren, die von Game-Designern zur Förderung von kooperativen Spielen genutzt werden können. Im Zuge der Forschung entwickelten sie kleine Prototypen und führten mit ihnen kleine Studien durch. 

Einige Studien gehen noch einen Schritt weiter. Sie untersuchten nicht nur Engagement, sondern auch die spezifischen Bedingungen erfolgreicher Kooperation in Spielen. Insbesondere durch das Design asymmetrischer Rollenverteilungen.

So zeigen die Arbeiten von \cite{harris_beam_2014}, \cite{harris_leveraging_2016} und \cite{harris_asymmetry_2019}, dass asymmetrische Spielkonzepte, bei denen sich Rollen, Fähigkeiten und Ziele der Spielenden unterschieden, einen positiven Einfluss auf die Zusammenarbeit haben. Untersucht werden dabei die Faktoren \say{Interdependence}, \say{Degrees of Interdepencene} sowie Mechaniken der Asymmetrie und Abhängigkeiten der Anwendungen. Ein asymmetrisches Spielkonzept ermöglicht außerdem eine Integration bzw. Inklusion von Spielergruppen mit eingeschränkten Fähigkeiten (vgl. \citealp{goncalves_exploring_2021}). Für die Entwicklung von Spielen, die für die gesamte Familie gedacht sind, eignen sie sich ebenfalls (vgl. \citealp{pais_promoting_2024}).

Die Arbeiten von Harris et. al. dienen als Grundlage für die weitere Entwicklung des Game Designs für asymmetrische Multiplayer-Spiele. So identifizierte \cite{guimaraes_rocha_game_2008} verschiedene kooperative Design-Pattern, die in der weiteren Forschung und deren Spielumsetzung anklang fanden. In der Arbeit von \cite{emmerich_impact_2017} werden drei der definierten Pattern verwendet um eine Aussage darüber treffen zu können, wie sich Interaktionen im Spiel gezielt gestalten lassen. Die Ergebnisse der Studie zeigen, dass eine hohe Spielerinterdependenz mit mehr Kommunikation und weniger Frustration einhergeht. Geteilte Kontrolle führte jedoch zu einem geringeren Erleben von Kompetenz und Autonomie.

Diese gestalterischen Grundlagen bilden einen Ausgangspunkt für eine weiterführende Forschung, die sich nun den sozialen, psychologischen und metastrukturellen Wirkungen dieser Spielkonzepte widmet.

Die Arbeit von \cite{depping_trust_2016} beschäftigte sich mit dem zwischenmenschlichen Vertrauen innerhalb einer zusammenarbeitenden Gruppe. Der Fokus lag dabei auf der Problematik, dass im Online-Umfeld bewährte Methodiken zum Teambildung nur schwer umsetzbar sind. Bestimmte Situationen müssen einfacher simuliert werden. Daher wurde durch Einsatz eines sozialen Spiels bestimmte Situation wie Risikosituationen und gegenseitige Abhängigkeiten simuliert. Das Zusammenarbeiten im Team kann auch eine Quelle von Konflikten oder Veränderungen sein. \cite{velez_ingroup_2014} zeigen den Fall, dass wenn eine (neue) fremde Person zu einer bestehenden Gruppe hinzu kommt, Spannungen entstehen können. Ihre Studie belegt,  dass kooperative Spiele nicht nur das Helferverhalten steigert, sondern auch das Aggressionsverhalten gegenüber Mitgliedern einer Fremdgruppe verringern kann.

In der Forschung von \ac{VR}-Spielen entstanden einige Interessante Arbeiten bezüglich des Game-Designs aber auch der enthaltenen Forschung.

\cite{karaosmanoglu_playing_2023} untersuchten die Vertrautheit von Zweierteams, die aus Fremden oder befreundeten Personen bestanden, im Zusammenhang mit sozialen und spielerischen Erfahrungen sowie ihrer Spielleistung. Die Studie ergab, dass es keine signifikanten Unterschiede zwischen den Freundeteams und Fremdenteams gab. Um Zusammenarbeit ging es ebenfalls in der Anwendung von \cite{sajjadi_maze_2014}. Die Ergebnisse der Studie zeigen, dass das konzipierte Spielkonzept bei den Spieler-Rollen mit den Sifteo Cubes und der VR Anwendung für die Oculus Rift eine positive Bewertung sowohl des Spielerlebnisses als auch der Zusammenarbeit ergab. Ebenfalls mit dem Bezug auf die Zusammenarbeit beschäftigte sich die Arbeit von \cite{smilovitch_birdquestvr_2019}, bei der es darüber hinaus um das Ausschöpfen der Möglichkeiten von \ac{VR} ging.

Im Kerngebiet der Kommunikation beschäftigte sich \cite{nasir_cooperative_2013} und \cite{nasir_effect_2015} zunächst mit der Entwicklung eines \say{ice-breaking} Spiels, das in Form eines 2D-\ac{RPG} konzipiert und entwickelt wurde. Der Sinn des Spiels ist dabei, die Zusammenarbeit in einer folgenden Gruppenarbeit zu verbessern. In der Studie wurden dabei drei unterschiedliche Gruppen miteinander verglichen (eine Gruppe hat das konzipierte Spiel gespielt, eine weitere hatte ein generisches ice-breaking Spiel gespielt und die dritte Gruppe keins). Die Gruppen, die das konzipierte Spiel gespielt hatten, zeigten eine erhöhte Interaktion. Die fortführende Studie untersuchte, ob das aus der ersten Studie umgesetzte Spiel die Zusammenarbeit in realen Teams verbessern kann. Es wurden dabei Gruppen verglichen, die vor der Arbeitsaufgabe das konzipierte ice-breaking gespielt hatten, mit denen, die es nicht gespielt hatten. Es wurde festgestellt, dass die Gruppen, die das ice-breaking Spiel spielten, in der anschließenden Arbeitsaufgabe eine erhöhte Zusammenarbeit zeigten.

\section{Wichtige Begriffe}
In den vorangegangenen Arbeiten beschäftigten sich die Autoren mit einigen Begrifflichkeiten, die Grundlage in der Konzeption und Entwicklung dieses Prototyps sowie der Forschung dieser Arbeit sind. 

In den folgenden Kapiteln werden diese Begriffe erklärt.

\subsection{Interdependence}
Der Begriff \say{Interdependence} stammt aus dem psychologischen Rahmenwerk für soziale und gruppenbezogene Interaktionen. Die Interdependence wird über das Ausmaß, in dem Gruppenmitglieder aufeinander angewiesen sind, um ihre Aufgabe effektiv zu erfüllen, definiert (vgl. \citealp[S. 451]{depping_cooperation_2017}; \citealp{saavedra_complex_1993}; \citealp[S. 197:4]{holly_asymmetric_2023}). Auf Video- und Computerspiele bezogen, können Aufgaben als das Spielziel bezeichnet werden (vgl. \citealp[S. 451]{depping_cooperation_2017}). 
In \citealp[S. 52]{van_der_vegt_patterns_2001} werden unterschiedliche Formen der Interdependence vorgestellt:

\paragraph{Task interdependence} beschreibt die Abhängigkeit von Teammitgliedern in ihren Aufgaben, die sie zu tun haben. Der Grad der Abhängigkeit nimmt zu, je komplexer die Aufgabe wird.
\paragraph{Goal interdependence} beschreibt die quantitativen und qualitativen Leistungen, die von den Gruppenmitgliedern gemeinsam erreicht werden müssen, um das Gruppenziel zu erreichen.

\subsection{Degrees of Interdependence}
In der Arbeit von \citealp[S. 7]{harris_asymmetry_2019} werden unterschiedliche Grade der Interdependence untersucht. Unter Grad der interdependence versteht man das Ausmaß in dem die Handlungen der Spieler voneinander abhängig sind, um das Spielziel erfolgreich zu erreichen. Je höher der Grad der Interdependence, desto stärker sind die Spieler darauf angewiesen, dass ihre Handlungen sinnvoll aufeinander abgestimmt sind (vgl. \citealp[S. 7]{harris_asymmetry_2019}).

\subsection{Soziale Präsenz}
Die soziale Präsenz beschreibt \say{das Gefühl, mit einem anderen zusammen zu sein} [eigene Übersetzung] \cite[S. 1]{biocca_towards_2003}. \say{Das andere} kann dabei entweder ein anderer Mensch oder eine künstliche Intelligenz sein. Innerhalb der \ac{HCI} untersucht die Theorie der sozialen Präsenz, wie das \say{Gefühl, mit einem anderen anderen zusammen zu sein} [eigene Übersetzung] \cite[S. 1]{biocca_towards_2003} durch Schnittstellen gestaltet und beeinflusst wird (vgl. \citealp[S. 1]{biocca_towards_2003}). Sie wird im Einzelnen durch die Wahrnehmung der physischen Repräsentation des anderen Spielers sowie durch psychologische Beteiligung und Verhaltensabhängigkeiten gekennzeichnet. Soziale Präsenz kann somit als das Ergebnis eines komplexen Zusammenspiels von wechselseitigem Verhalten, Kommunikation und sozialen Kontextmerkmalen gesehen werden. Die Voraussetzung hierfür ist, dass ein Spieler die Kopräsenz einer anderen sozialen Einheit wahrnimmt (vgl. \citealp[S. 1]{emmerich_game_2016}).


\section{Forschungsbeitrag}

Diese Arbeit knüpft an die grundlegenden Untersuchungen von Nasir et al. (\citeyear{nasir_cooperative_2013,nasir_effect_2015}) an und erweitert deren Forschungsansatz um Vor- und Nachtest innerhalb derselben Versuchsgruppe. Ziel ist es, nachzuweisen, dass durch den Einsatz des entwickelten Prototyps gezielt eine Verbesserung der gemeinsamen Kommunikation in bestehenden Gruppen erzielt werden kann. Im Gegensatz zu einer stilisierten Anwendung mit dem Zweck des Ice-Breakings dient der entwickelte Prototyp darüber hinaus als vollwertiges Multiplayer-Spiel, das auch unabhängig vom Kontext wissenschaftlicher Experimente in der Freizeit genutzt werden kann.

Ein zentrales Merkmal des Prototyps ist seine Realisierung als Cross-Plattform-Multiplayer-Spiel (vgl. Abbildung \ref{fig:lotz_multiplayer_types}). Dabei kommen unterschiedliche Endgeräte zum Einsatz, um die damit verbundenen Effekte auf die Interaktion und Kommunikation der Nutzer gezielt untersuchen zu können. Im Fokus der Untersuchung steht insbesondere die Integration von \ac{AR} sowie die Touchsteuerung beider Anwendungskomponenten, welche im Rahmen des Prototyps umgesetzt werden.

\chapter{Stand der Technik} \label{sec:sota}
Nachdem die einzelnen Charakteristiken von Multiplayer-Spielen im Kapitel \emph{\nameref{sec:basics}} vorgestellt wurden, werden nun Spiele vorgestellt, welche im Rahmen dieser Arbeit näher betrachtet wurden.

Die Spielreihe \say{\textbf{We were here}} des niederländischen Entwicklerstudios Total Mayhem Games umfasst eine Reihe asymmetrischer kooperativer Multiplayer-Spiele, in denen zwei Spieler gemeinsam Rätsel lösen und Hindernisse überwinden müssen, um aus einer Umgebung zu entkommen, in der die Spielfiguren gefangen sind. Die Kommunikation erfolgt ausschließlich über ein im Spiel integriertes Walkie-Talkie-System. In der Regel sieht sich eine Person mut einem Rätsel oder einer Aufgabe konfrontiert, die sie dem Mitspieler beschreiben muss, damit dieser auf Grundlage der erhaltenen Information die korrekte Lösung übermitteln oder selbständig umsetzen kann. Die beiden Spieler befinden sich dabei in voneinander getrennten Bereichen der Spielwelt, was eine präzise und kooperative Kommunikation erforderlich macht (vgl. \citealp{total_mayhem_games_we_2017,total_mayhem_games_we_2018}).  

Auch die vom schwedischen Entwicklerstudio Hazelight Studios veröffentlichten Titel \say{\textbf{A Way Out}}, \say{\textbf{It takes two}} und  \say{\textbf{Split Fiction}} sind kooperative Action-Adventure-Spiele, die im lokalen oder Online-Splitscreen-Modus zu zweit gespielt werden. Dabei übernehmen die Spieler jeweils die Rolle eines der beiden Protagonisten. Das Gameplay ist asymmetrisch angelegt, was sich in den differenzierten Aufgabenverteilungen innerhalb der einzelnen Spielabschnitte widerspiegelt. Jedoch müssen die Spieler oft koordinativ zusammenarbeiten. Um im Spielverlauf voranzukommen, müssen die Spieler gemeinsam Rätsel lösen, Hindernisse überwinden und aufeinander abgestimmt der Geschichte folgen (vgl. \citealp{hazelight_studios_way_2018,hazelight_studios_it_2021,hazelight_studios_split_2025}).

Ein weiteres Beispiel stellt das Spiel \say{\textbf{The past within}} des niederländischen Entwicklerstudios Rusty Lake dar. Es handelt sich um ein asymmetrisches kooperatives Multiplayer-Spiel, bei dem zwei Spieler gemeinsam innerhalb einer Sitzung in zwei Zeitebenen (Vergangenheit und Zukunft) Rätsel lösen müssen, um der Protagonistin und ihrem Vater zu helfen. Dabei befindet sich eine Person in einer \say{2D}-Anwendung, während die andere in einer \ac{3D}-Umgebung spielt. Besonders hervorzuheben ist die plattformübergreifende Spielbarkit, die eine Kooperation unabhängig vom verwendeten Endgerät ermöglicht (vgl. \citealp{rusty_lake_past_2022}). 

Das bereits in Kapitel \ref{sec:hybrid-multiplayer} thematisierte Spiel \say{\textbf{Keep Talking and Nobody Explodes}} stellt ein weiteres Beispiel für asymmetrisches kooperatives Spieldesign dar. Eine Person benötigt das Spiel, um es gemeinsam im Team spielen zu können. Das Spiel zeichnet sich durch seine Cross-Plattform-Funktionalität aus, bei der eine Person die Rolle des Bombenentschärfer übernimmt und sich dabei innerhalb des Spiels befindet. Die übrigen Spielteilnehmer die als \say{Experten} fungieren, verfügen über ausgedruckte Handbücher mit Anleitungen zur Bombenentschärfung. Sie sehem jedoch die Bombe nicht selbst, sondern müssen anhand der mündlichen Beschreibung durch den Bombenentschärfer die passenden Informationen aus dem Handbuch ableiten. Jedoch hat das Team nur eine begrenzte Länge an Zeit, da die Bombe nach ablaufen des verbauten Timers explodiert (vgl. \citealp{steel_crate_games_keep_2015}).

Ein aktuelles Beispiel ist das im März 2025 erschienene Spiel \say{\textbf{Myrmidon}} des französischen Entwicklerstudios Studio Popot. Auch hierbei handelt ers sich um ein asymmetrisches kooperatives Multiplayer-Spiel, in dem zwei Spieler gemeinsam, jedoch in unterschiedlichen Rollen, agieren. Eine Person steuert eine Stop-Motion-Puppe, die sich in einer filmisch inszenierten Welt bewegt und dabei Hindernisse überwinden sowie Plattformen erreichen muss. Die zweite Person übernimmt die Rolle des Animators, der für die Bedienung der Kulisse und die Aktivierung der Umgebungselemente zuständig ist. Nur durch koordinierte Zusammenarbeit können die Spieler die Herausforderungen bewältigen (vgl. \citealp{studio_popote_myrmidon_2025}).
