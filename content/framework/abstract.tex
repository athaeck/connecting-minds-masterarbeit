\chapter*{Abstract\markboth{Abstract}{}}
\addcontentsline{toc}{chapter}{Abstract}
Diese Masterarbeit widmet sich der Konzeption und Entwicklung eines asymmetrischen AR-/3D-Adventure Multiplayer Anwendung zur Beobachtung und Förderung der Kommunikation zwischen Individuen. Vor dem Hintergrund zunehmender sozialer Isolation, insbesondere bei jungen Erwachsenen durch die COVID-19-Pandemie, untersucht die Arbeit, inwiefern spielerische Interaktionssysteme gezielt soziale Kommunikation anregen und verbessern können. Die Arbeit verfolgt das Ziel, ein digitales Spielsystem zu entwickeln, in dem zwei Spielerrollen (Player und Watcher) kooperativ Rätsel lösen und dabei unterschiedliche Perspektiven und Informationslagen einnehmen. Aufbauend auf kognitionswissenschaftlichen Modellen und spieltheoretischen Konzepten wurde ein funktionaler Prototyp umgesetzt, getestet und im Rahmen einer Nutzerstudie evaluiert. Die Untersuchung analysiert quantitative Veränderungen im Kommunikationsverhalten der Teilnehmenden und zeigt, dass asymmetrisches Gameplay hinreichenden Auswirkungen auf Gesprächsführung, Rollenverteilung, Empathie und Engagement haben. Die Ergebnisse deuten das Potenzial von Mixed-Plattform-Spielen zur Verbesserung zwischenmenschlicher Kommunikation an, und liefern praxisnahe Empfehlungen für zukünftige interaktive Systeme in sozialen und pädagogischen Kontexten.


% [TODO: Zusammenfassung der Arbeit schreiben]
% \say{\emph{A Fraction of Time}} basiert auf der Idee, den Spieler verschiedene parallele Zeitlinien von sich selbst erzeugen zu lassen. Der Spieler muss in diesen verschiedenen Zeitlinien, mithilfe der diversen Zeitlinien von sich selbst, Rätsel lösen und mit seinen Zeitlinien zusammenarbeiten. Grundlage dieser Weiterentwicklung ist der digitale Prototyp, der bereits im Gamedesign Workshop im Wintersemester 2021/ 2022 entwickelt wurde. 

% Die Zielsetzung dieser Abschlussarbeit ist es, das bisher implementierte System auf technischer Seite soweit zu verbessern, dass es für den Spieler offensichtlich fehlerfrei funktioniert. Außerdem soll ein Rahmen geschaffen werden, in dem es möglich ist, Erweiterungen zu integrieren. Hinzu werden weitere Inhalte, wie weitere Level und Spielwelten konzipiert und umgesetzt. Der Spieler soll dabei eine Variation an Spielwelten und Rätsel erhalten, welche er lösen muss, um das Level abzuschließen. Hierbei soll ihm auch das Spielprinzip so vermittelt werden, dass er durch das Verständnis der Mechanik jedes Rätsel lösen kann. Validiert werden die Inhalte durch User-Tests, bei denen das Verständnis der Spielmechanik auf inhaltlicher und technischer Ebene abgefragt wird.