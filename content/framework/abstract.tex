\chapter*{Abstract\markboth{Abstract}{}}
\addcontentsline{toc}{chapter}{Abstract}
Diese Masterarbeit widmet sich der Konzeption und Entwicklung eines asymmetrischen AR-/3D-Adventure Multiplayer Anwendung zur Beobachtung und Förderung der Kommunikation zwischen Individuen. Vor dem Hintergrund zunehmender sozialer Isolation, insbesondere bei jungen Erwachsenen durch die COVID-19-Pandemie, untersucht die Arbeit, inwiefern spielerische Interaktionssysteme gezielt soziale Kommunikation anregen und verbessern können. Die Arbeit verfolgt das Ziel, ein digitales Spielsystem zu entwickeln, in dem zwei Spielerrollen (Player und Watcher) kooperativ Rätsel lösen und dabei unterschiedliche Perspektiven und Informationslagen einnehmen. Aufbauend auf kognitionswissenschaftlichen Modellen und spieltheoretischen Konzepten wurde ein funktionaler Prototyp umgesetzt, getestet und im Rahmen einer Nutzerstudie evaluiert. Die Untersuchung analysiert quantitative Veränderungen im Kommunikationsverhalten der Teilnehmenden und zeigt, dass asymmetrisches Gameplay hinreichenden Auswirkungen auf Gesprächsführung, Rollenverteilung, Empathie und Engagement haben. Die Ergebnisse deuten das Potenzial von Mixed-Plattform-Spielen zur Verbesserung zwischenmenschlicher Kommunikation an, und liefern praxisnahe Empfehlungen für zukünftige interaktive Systeme in sozialen und pädagogischen Kontexten.
