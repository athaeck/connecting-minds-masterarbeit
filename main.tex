\documentclass[
	12pt,
	a4paper,
	bibtotoc,
	cleardoubleempty, 
	idxtotoc,
	ngerman,
	openright
	final,
	listof=nochaptergap,
	]{scrbook}
\usepackage{cmap}
\usepackage[T1]{fontenc}
\usepackage[utf8]{inputenc}
\usepackage{pdfpages}

\usepackage{nameref}
\usepackage{dirtytalk}
\usepackage{natbib}

\include{preamble}

\begin{document}

\setcounter{secnumdepth}{3}

% Titelblatt
\include{content/framework/title}
\cleardoubleemptypage

\frontmatter
\pagenumbering{Roman}

% Abstract
\chapter*{Abstract\markboth{Abstract}{}}
\addcontentsline{toc}{chapter}{Abstract}
Diese Masterarbeit widmet sich der Konzeption und Entwicklung eines asymmetrischen AR-/3D-Adventure Multiplayer Anwendung zur Beobachtung und Förderung der Kommunikation zwischen Individuen. Vor dem Hintergrund zunehmender sozialer Isolation, insbesondere bei jungen Erwachsenen durch die COVID-19-Pandemie, untersucht die Arbeit, inwiefern spielerische Interaktionssysteme gezielt soziale Kommunikation anregen und verbessern können. Die Arbeit verfolgt das Ziel, ein digitales Spielsystem zu entwickeln, in dem zwei Spielerrollen (Player und Watcher) kooperativ Rätsel lösen und dabei unterschiedliche Perspektiven und Informationslagen einnehmen. Aufbauend auf kognitionswissenschaftlichen Modellen und spieltheoretischen Konzepten wurde ein funktionaler Prototyp umgesetzt, getestet und im Rahmen einer Nutzerstudie evaluiert. Die Untersuchung analysiert quantitative Veränderungen im Kommunikationsverhalten der Teilnehmenden und zeigt, dass asymmetrisches Gameplay hinreichenden Auswirkungen auf Gesprächsführung, Rollenverteilung, Empathie und Engagement haben. Die Ergebnisse deuten das Potenzial von Mixed-Plattform-Spielen zur Verbesserung zwischenmenschlicher Kommunikation an, und liefern praxisnahe Empfehlungen für zukünftige interaktive Systeme in sozialen und pädagogischen Kontexten.

\cleardoubleemptypage

\chapter*{Gender-Hinweis\markboth{Gender-Hinweis}{}}
\addcontentsline{toc}{chapter}{Gender-Hinweis}

Zur besseren Lesbarkeit wird in dieser Hausarbeit das generische Maskulinum verwendet. Die in dieser Arbeit verwendeten Personenbezeichnungen beziehen sich, sofern nicht anders kenntlich gemacht, auf alle Geschlechter.
\cleardoubleemptypage

% Inhaltsverzeichnis
\phantomsection
\addcontentsline{toc}{chapter}{Inhaltsverzeichnis}
\tableofcontents
\cleardoubleemptypage

% Abbildungsverzeichnis einbinden und ins Inhaltsverzeichnis
% WORKAROUND: tocloft und KOMA funktionieren zusammen nicht
% korrekt\phantomsection
\phantomsection 
\addcontentsline{toc}{chapter}{\listfigurename} 
\listoffigures
\cleardoubleemptypage

% Tabellenverzeichnis einbinden und ins Inhaltsverzeichnis
% WORKAROUND: tocloft und KOMA funktionieren zusammen nicht
% korrekt\phantomsection
% \phantomsection
% \addcontentsline{toc}{chapter}{\listtablename}
% \listoftables
% \cleardoubleemptypage

% Quellcodeverzeichnis einbinden und ins Inhaltsverzeichnis
% \phantomsection
% \addcontentsline{toc}{chapter}{Quellcodeverzeichnis}

% %Define listing
% \makeatletter
% \begingroup\let\newcounter\@gobble\let\setcounter\@gobbletwo
%   \globaldefs\@ne \let\c@loldepth\@ne
%   \newlistof{listings}{lol}{\lstlistlistingname}
% \endgroup
% \let\l@lstlisting\l@listings
% \makeatother
% \setlength{\cftlistingsindent}{0em}
% \renewcommand{\cftlistingsafterpnum}{\vskip0pt} %Spacing between entries
% \renewcommand*{\cftlistingspresnum}{\lstlistingname~}
% \settowidth{\cftlistingsnumwidth}{\cftlistingspresnum}
% \renewcommand{\lstlistlistingname}{Quellcodeverzeichnis}
% % Tabellenverzeichnis anpassen
% \renewcommand{\lstlistingname}{Codeauschnitt}
% \renewcommand{\cftlistingsaftersnum}{:}
% % Breite des Nummerierungsbereiches [Codeauschnitt 1:]
% \newlength{\codeLength}
% \settowidth{\codeLength}{\bfseries\lstlistingname\cftlistingsaftersnum}
% \addtolength{\codeLength}{5mm}
% \setlength{\cftlistingsnumwidth}{\codeLength}
% \lstlistoflistings
% \cleardoubleemptypage

% Abkürzungsverzeichnis
\chapter*{Abkürzungsverzeichnis\markboth{Abkürzungsverzeichnis}{}}
\addcontentsline{toc}{chapter}{Abkürzungsverzeichnis}

\begin{acronym}
\acro{2D}{zweidimensional}
\acro{3D}{dreidimensional}
\acro{ACK}{Acknowledgement}
\acro{AI}{Artificial Intelligence}
\acro{AR}{Augmented Reality}
\acro{AV}{Augmented Virtuality}
\acro{CF}{Collaborative Floor Holding}
\acro{CHI}{Conference on Human Factors in Computing Systems}
\acro{CMP}{Cooperative Performance Metrics}
\acro{CSCW}{Conference on Computer-Supported Cooperative Work \& Social Computing}
\acro{DM}{Digitale Medien}
\acro{DIM}{Design Interaktiver Medien}
\acro{FBX}{Filebox}
\acro{GEQ}{Game Experience Questionnaire}
\acro{HAR}{Handheld Augmented Reality}
\acro{HCI}{Human-Computer Interaction}
\acro{HFU}{Hochschule Furtwangen University}
\acro{HMD}{Head-Mounted-Display}
\acro{HTTP}{Hypertext Transfer Protocol}
\acro{HUD}{Head-up-Display}
\acro{IEQ}{Immersive Experience Questionnaire}
\acro{IIIUS}{Institut für Intelligente Interaktive Ubiquitäre Systeme}
\acro{IMI}{Intrinsic Motivation Inventory}
\acro{IOS}{Inclusion of the Other in the Self}
\acro{ID}{Identifikationsnummer}
\acro{IT}{Information Technology}
\acro{LTS}{Long-Term-Support}
\acro{M}{Mittelwert}
\acro{MBTI}{Myer-Briggs Type Indicator}
\acro{MDA}{Mechanics, Dynamics and Aesthetics}
\acro{MIM}{Medieninformatik Master}
\acro{MMOG}{Massively Multiplayer Online Game}
\acro{MR}{Mixed-Reality}
\acro{MUD}{Multi-User-Dungeon}
\acro{MVC}{Model-View-Controller}
\acro{MVP}{Minimum Viable Product}
\acro{NASA-TLX}{NASA Task Load Index}
\acro{NPM}{Node Package Manager}
\acro{OBJ}{Object}
\acro{OSI}{Open Systems Interconnection}
\acro{P2P}{Peer-to-Peer}
\acro{PR}{Physcial Reality}
\acro{QCAE}{Questionnaire of Cognitive and Affective Empathy}
\acro{Q-Q}{Quantil-Quantil}
\acro{RPG}{Role Playing Game}
\acro{RTS}{Real-Time-Strategy}
\acro{SAM}{Self-Assessment Manikin}
\acro{SAR}{Spatial Augmented Reality}
\acro{Sci-Fi}{Science-Fiction}
\acro{SD}{Standardabweichung}
\acro{SF}{Single Floor Holding}
\acro{SUS}{System Usability Scale}
\acro{SYN}{Synchronize}
\acro{TCP}{Transmission Control Protocol}
\acro{UDP}{User Datagram Protocol}
\acro{UCD}{User Centered Design}
\acro{UI}{User Interface}
\acro{URP}{Universal Render Pipeline}
\acro{URL}{Uniform Resource Locator}
\acro{VC}{Virtuality Continuum}
\acro{VE}{Virtual Environments}
\acro{VR}{Virtual Reality}
\acro{XR}{Extended Reality}
\end{acronym}

\mainmatter

\chapter{Einleitung}
% Einleitung über Themen von multiplayern, zusammen spielen, vl corona noch, dann darüber dass es wenig asymmetrische multiplayer gibt, irgendwie steam lib durchgehen oder andere Listen noch, verweise auf crossplattform noch geben da es ähnlich wäre.

% Einsamkeitsstudien von corona zeigen und auflisten von studien or Papern bei denen es darum geht, dass die Spiele spaßig sind und kommunikationsfördernd und social engagement steigernd sind um sowas wie corona entgegenzuwirken.

% Kurzer Überblick über den Beginn der Arbeit; enthält den Interaktionsdesignworkshop
% Titel des Projektes

[TODO: Wissenschaftliche Einleitung zu Themen wie Corona, bestehenden Online Games usw geben]

Ein erstes Konzept, um auf diese Probleme einzugehen, wurde im Rahmen der Veranstaltung Interaktionsdesign im 1. Mastersemester der Studiengänge \ac{MIM} und \ac{DIM} im Wintersemester 2023/2024 entwickelt und umgesetzt. 
Aufgrund der interessanten Rätselmechanik, des positiven Feedbacks bei den in der Projektausarbeitung enthaltenen Probandentests und des interessanten Forschungsgebiets wurde beschlossen, dieses Projekt weiterzuführen und die konzeptionellen Grundideen zu implementieren.

Der Titel des Projektes lautet \say{\emph{Connecting-Minds}}.

\section{Motivation und Aufgabenstellung}
Das Ziel dieser Masterarbeit ist es, den bisherigen Prototyp, der ein Mindestmaß an Funktionen des Konzepts enthielt, auf technischer Ebene neu zu entwickeln. Zusätzlich dazu soll der Prototyp als Versuchsumgebung dienen, um Effekte auf das Kommunikationsverhalten der Spieler zu erforschen. 

Die folgenden Forschungsfragen bilden das Grundgerüst dieser Abschlussarbeit:


\begin{itemize}
    \item \textbf{Kann eine spielbasierte Umgebung für die Untersuchung und Verbesserung von Kommunikation zwischen zwei oder mehreren Personen realisiert werden?}
    \item \textbf{Welche spezifischen Eigenschaften muss eine solche Umgebung aufweisen und welche Kommunikationsparameter werden dabei angesprochen?}
    \item \textbf{Welche Verbesserungen in der Kommunikation zwischen den Anwendern können durch ein asynchrones Multiplayer-Spiel mit zwei verschiedenen Spielerklassen beobachtet werden?}
    \item \textbf{Welche Unterschiede können in der Art das Kommunikationsverhalten bei der Verwendung von zwei unterschiedlichen Anwendungen (AR und 3D) (festgestellt/beobachtet) werden}
    \item \textbf{Wie stehen die Nutzer zu einem spielerischen Ansatz und zur Verbesserung der Kommunikation, insbesondere auch im Umgang mit Fremden?}
\end{itemize}

\section{Struktur der Arbeit}

[TODO: Zusammenfassen in welchen Kapitel sich was befindet]
\chapter{Theoretische Grundlagen}

In diesem Kapitel werden die für den Forschungshintergrund und für die Entwicklung des Prototyps wichtigen theoretischen Grundlagen vorgestellt.
Zunächst werden verschiedene Theorien zu Kommunikationsmodellen vorgestellt, auf deren Basis das Modell für diese Anwendung zugrunde liegt.
Im Anschluss daran werden die in der Ludologie beschriebenen Akteure vorgestellt, welche sich in den Probanden dieser Masterarbeitsstudie wiederfinden werden. Allgemein ist bekannt, dass Video- und Computerspiele drei verschiedene Modi haben können: Singleplayer, Multiplayer und Mischformen. Da für den Zweck dieser Studie ein Multiplayer-Spiel konzipiert und umgesetzt wurde, werden im weiteren Verlauf verschiedene Kategorien von Multiplayer-Spielen vorgestellt. Außerdem werden die damit einhergehenden Netzwerkinfrastrukturen vorgestellt, die relevant sind und für die weitere Entwicklung relevant sein könnten.

\section{Kommunikationsmodelle}
[Literatur suchen]
% In der Kommunikationswissenschaft wird die Kommunikation in 2 Arten unterteilt, die Massen- und Individualkommunikation.

\subsection{Nach Schulz von Thun}
\subsection{Nach Wazlawik}
\subsection{Nach Rogers}

% [Könnte besser in die Einleitung passen
% \section{Spiele als soziales Medium}
% \cite{depping_trust_2016}
% \cite{gerling_designing_2014}
% \cite{ducheneaut_alone_2006}
% ]

\section{Spielertypen}
Die Kommunikationswissenschaft umfasst zwar den Hauptteil dieser Arbeit, allerdings beinhaltet diese ebenfalls ludologische Aspekte. Dabei geht es um die Lehre über das Spiel (vgl. \cite{ludologie_spielforschung_nodate}). 

Im Hinblick auf die Konzeption und die Entwicklung eines Spiels ist es wichtig, die Eigenschaften des Spielsystems so zu gestalten, dass sich Begeisterung und Engagement bei der gewünschten Zielgruppe hervorrufen. Aus diesem Grund muss zunächst die Zielgruppe in verschiedene Typen eingeteilt werden. In der Ludologie gibt es dafür verschiedene Spielertypen. Zwar ist nicht jeder Mensch ein \say{Spielertyp}, grundsätzlich kann er jedoch über verschiedene Spielelemente angesprochen werden (vgl. \cite{ludologie_spielertypen_nodate}).

\subsection{Nach Bartle}
1996 beschäftigte sich Richard Bartle mit der Frage, welche Spielertypen es in der Ludologie gibt. Dabei ging es zunächst um die Klassifizierungen, welche Ansätze es beim Spielen von sogenannten \ac{MUD}s existieren (vgl. \cite{bartle_hearts_1996}). Diese Klassifizierungen werden noch heute für die Einteilung in Spielertypen genutzt.

Bartle unterscheidet bei der Einteilung der Spielertypen auf zwei unterschiedliche Grundinteressen (vgl. Abbildung \ref{fig:bartle-muds}):

\begin{figure}[ht]
\centering
\includegraphics[width=1\linewidth]{content/pictures/basic_interests.PNG}
\caption{Interessen Graph nach Bartle (vgl. \cite{bartle_hearts_1996})}
\label{fig:bartle-muds}
\end{figure}

In X-Achsen-Richtung wird unterschieden, ob der Spieler seine Spielerfahrung über das Verhalten der anderen Mitspieler (Players) oder der Spielwelt (World) bevorzugt. Auf der Y-Achsen-Richtung unterscheidet er, ob der Spieler bevorzugt, selbst einen Einfluss auf die Spielwelt zu geben und diese beeinflusst (Acting) oder ob er in tiefere Interaktion mit der Spielwelt eingehen will (Interacting).

Die daraus resultierenden Typen sind:
\paragraph{Achiever}
Sie sind daran interessiert, auf die Welt einzuwirken, um dadurch in sie eintauchen zu können. Sie wollen das Spiel meistern und dazu bringen, das zu tun, was sie wollen. Ihr Status im Spiel ist ihnen wichtig und die wenige Zeit, die sie dafür benötigt haben.

\paragraph{Explorer}
Sie wollen vom Spiel überrascht werden und mit der Spielwelt interagieren. Die virtuelle Welt löst ein Gefühl des Staunens aus, nach dem sie sich sehnen. Sie sind stolz auf das Wissen über das Spiel, das sie sammeln, und wollen dieses Wissen gerne an neue Spieler weitergeben.

\paragraph{Socialiser}
Sie wollen mit anderen Spielern interagieren. Zumeist erfolgt das über Gespräche, es kann aber auch ungewöhnliche Verhaltensweisen einschließen. Andere Menschen kennenzulernen und mehr über sie zu erfahren, ist für sie wertvoller als für andere. Die Spielwelt ist für sie nur eine Kulisse, für sie sind andere Charaktere fesselnder. Sie sind stolz auf Freundschaften, ihre Kontakte und ihren Einfluss.

\paragraph{Killer}
Sie sind daran interessiert, auf andere Spiele einzuwirken und mit ihnen Dinge zu machen. Im Allgemeinen erfolgt dies ohne das Einverständnis der anderen Spieler. Sie wollen ihre Überlegenheit gegenüber anderen Menschen demonstrieren. Sie sind stolz auf ihren Ruf und oft geübten Kampffähigkeiten.

(vgl. \cite{bartle_hearts_1996}).

\subsection{Erweiterte Einteilungen}
Bartle ist nicht der Einzige, der sich mit Spielertypen auseinandergesetzt hat. Seine Forschung gilt als Fundament, welches in der weiteren Forschung für Diskussionen in der Forschungs- und Game-Design-Community gesorgt hat. 
\begin{quote}
    \textit{
        \enquote{Player types are not a defined concept and any categorization of players or users needs to occur within the context of a particular application or domain. Play-personas are suggested as a useful tool that can be used to put player type research into practice as part of the design process of gamified systems.}
    } 
    (\cite{dixon_player_nodate})
\end{quote}

\paragraph{Dixon} 
stellt Spieler-Personae vor, die wie im \ac{UCD}-Prozess verwendet werden können. Dadurch muss im Designprozess nicht zu sehr zwischen Motivation, Verhalten oder Vorlieben unterschieden werden, da Personae als reichhaltige und erzählerische Darstellung gedacht sind (vgl. \cite{dixon_player_nodate}).

\paragraph{Bateman und Boon}
benutzen in ihrer 2005 erschienen Studie zur Bestimmung des ersten Modells des demografischen Game Designs (DGD1) vier Spielstile, welche sie durch die Hinzunahme der Myers-Briggs Type Indicator (vgl. \cite{noauthor_mbti_nodate}) ableiteten (vgl. \cite{bateman_21st_2005}).
Conquerer (Eroberer), Manager, Wanderer (Wanderer) und Participant (Teilnehmer) waren dabei die vier Spielstile.

In einer zweiten Studie wurden vier hypothetische Spielstile erstellt, welche von einer Studie von Berens 2000 (vgl. \cite{berens_understanding_2000}) abgeleitet wurden (vgl. \cite{bateman_player_2012}). Die resultierenden Stile sind folgende: Logistical, Tactical, Strategic und Diplomatic.

Im Kern sind diese Modelle Ableitungen von Bartles ursprünglicher Metrik (vgl. \cite{ludologie_spielertypen_nodate}).

\paragraph{Yee}
Nick Yee entwickelte empirisch fundiertes Modell zur Beschreibung von Spielmotivationen in Online-Spielen, das bis heute einen großen Einfluss auf die Ludologie hat. Mittels eines faktorenanalytischen Ansatzes untersuchte er eine Vielzahl an Daten aus Online-Umfragen und identifizierte dabei 10 spezifische Motivationsgruppen, die in drei übergeordnete Hauptkategorien gegliedert werden (vgl. Abbildung \ref{fig:nick_yee_motivations}):

\begin{figure}[ht]
\centering
\includegraphics[width=1\linewidth]{content/pictures/nick_yee_categorizations.PNG}
\caption{Motivationsgruppen nach Nick Yee (vgl. \cite{yee_motivations_nodate})}
\label{fig:nick_yee_motivations}
\end{figure}

Die Achievement-Komponente umfasst den Fortschritt im Spiel, sowie das damit einhergehende Verlangen Macht zu erlangen, schnell voranzukommen und Symbole von Reichtum oder Status im Spiel zu erlangen. Außerdem existiert ein Interesse daran, die Mechanik des Spiels zu analysieren, die Regeln und Systeme zu verstehen um die Leistung der Spielfigur zu optimieren. Außerdem ist der Wettbewerb wichtig. Es besteht der Wunsch danach sich mit anderen zu messen und gegen sie anzutreten.

Die soziale Komponente beschreibt dabei die Sozialisierung der Spieler, bei denen sie Interesse daran haben anderen Spielern zu helfen und sich mit ihnen zu Unterhalten. Daraus entstehen Beziehungen, bei denen der Wunsch nahe liegt, dass langfristige und bedeutungsvolle Beziehungen zu anderen aufgebaut werden können. Außerdem ist Teamarbeit gewünscht, um sich gegen andere zu messen und gegen sie anzutreten.

Die Immersions-Komponnete beschreibt das Entdecken in der Spielwelt und dem damit einhergehende finden von Dingen, Wissen zu erlangen, welches den meisten anderen Spielern unbekannt ist. Rollenspiel-Elemente sind dabei besonders wichtig, um den Spielfiguren eine Hintergrundgeschichte zu geben und gemeinsam eine improvisierte Geschichte zu entwickeln. Der Spielavatar sollte auch anpassbar sein, damit der individuelle Geschmack der Spieler in das Spiel einfließen kann. Die Spiel-Welt wird genutzt um von den Problemen der realen-Welt zu entkommen.

\paragraph{weitere Modelle}
Im Zuge der fortschreitenden Forschungen entstanden weitere Modelle wie das Gamer Motivation Model, das auf Basis der Forschung von Nick Yee entwicjelt wurde (vgl. \cite{ludologie_spielertypen_nodate}):

\begin{figure}[ht]
\centering
\includegraphics[width=1\linewidth]{content/pictures/gamer_motivations_model.png}
\caption{Gamer Motivation Model der QUANTIC FOUNDRY (vgl. \cite{noauthor_quantic_nodate})}
\label{fig:gamer_motivation_model}
\end{figure}

Ein weiteres Modell, das in der Arbeit von Bateman genannt wird, ist das BRAINHEX-Model, bei dem die verschiedenen Spielertypen in Hexagonaler Anordnung platziert werden (vgl. Abbildung: \ref{fig:brain-hex}):

\begin{figure}[ht]
\centering
\includegraphics[width=1\linewidth]{content/pictures/brainhex-classes.png}
\caption{Brainhex-Model Darstellung von \cite{noauthor_i_nodate} nach \cite{nacke_brainhex_2013}}
\label{fig:brain-hex}
\end{figure}

% \cite{bartle_hearts_nodate}

% [erwähnen wurde aber rausgelassen, man kann erwähnen, dass es noch weitere klassifizierungen gibt
% \subsection{Das BrainHex-Model}
% \cite{nacke_brainhex_2014}
% ]

% [kommt zu wichtige Begriffe
% \section{Kooperative Gamedesign Pattern}
% \subsection{Was sind Game Pattern}
% \cite{bjork_patterns_2005}

% \subsection{Complementarity}

% \subsection{Synergies}

% \subsection{Abilities}

% \subsection{Shared Goals}

% \subsection{Synergies between goals}

% \subsection{Special Rules for Player of the same Team}

% \subsection{Camera Setting}

% \subsection{Interacting with the same object}

% \subsection{Shared puzzle}

% \subsection{Shared characters}

% \subsection{Special characters targetting lone wolf}

% \subsection{Vocalization}

% \subsection{Limited ressources}

% \subsection{Einflussnahme}
% \cite{emmerich_impact_2017}

% ]
\section{Multiplayer-Spiele}
Im Vergleich zu Einzelspieler-Spielen existieren bei Multiplayer-Spielen nicht nur Unterscheidungen im Genre des Spiels, sondern auch in den Spielrollen (Symmetrie / Asymmetrie) sondern auch in den Spielzeitpunkten (Synchron / Asynchron) wann die Spielteilnehmer an ihrem Sportfortschritt weiter arbeiten. 

In den folgenden Kapiteln werden die jeweiligen Eigenschaften der unterschiedlichen Ausprägungen von Multiplayer-Spielen aufgezählt.

\subsection{Synchrone Multiplayer}
Synchrone Multiplayer-Spiele sind solche, bei denen die Spieler i. d. R. zum selben Zeitpunkt, bzw. zur selben Zeit gemeinsam miteinander oder gegeneinander Spielen. [Quelle suchen]. Weit verbreitet sind hier vorallem Ego-Shooter wie die \say{Call of Duty}-Reihe, bei denen die Spieler innerhalb einer Sitzung gegeneinander im \say{Einzel} oder als \say{Team} gegeneinander Spielen (vgl. \cite{noauthor_call_nodate}).

\subsection{Asynchrone Multiplayer}
Asynchrone Multiplayer-Spiele werden zeitversetzt gespielt. [Quelle und beispiele suchen]

\subsection{Symmetrische Multiplayer}
Symmetrische Spiele sind die, bei denen alle Spieler die selben Spielregeln haben und das gleiche Spielziel verfolgen. Viele traditionelle Spiele wie Schach oder Computer- und Videospiele wie \say{Counter-Strike: Global Offensive (CS:GO)} sind Symmetrische Multiplayer-Spiele, bei denen für jeden Spieler das gleiche Ziel gilt (vgl. \cite[S. 12]{adams_fundamentals_2013}), (vgl. \cite{noauthor_counter-strike_nodate}). 


\subsection{Asymmetrische Multiplayer}
Asymmetrische Spiele hingegen können unterschiedliche Spieler unterschiedliche Regeln haben und versuchen ebenfalls unterschiedliche Ziele zu erreichen (vgl. \cite[S. 12]{adams_fundamentals_2013}). Sie sind in kooperativen und kompetitiven Spielen weit verbreitet und sind bspw. in Form von verschiedenen \say{Helden} oder \say{Klassen} umgesetzt. So gibt es z.B. in \say{Overwatch} oder \say{League of Legends}  unterschiedliche \say{Support}-Charaktere, deren Aufgabe es ist das Team zu heilen (vgl. \cite{smilovitch_birdquestvr_2019}), (vgl. \cite{noauthor_league_2025}), (vgl. \cite{noauthor_overwatch_nodate}). 
Außerdem ermöglichen sie, dass Spieler mit unterschiedlichen Fähigkeiten und Fähigkeitsniveaus gemeinsam spielen können. Ein asymmetrisches Design kann zudem die Inklusivität in Spielen fördern (vgl. \cite{smilovitch_birdquestvr_2019}).

\subsection{Artverwandte Beispiele}
Hier kommen die analysierten Spiele rein, also die Auflistung der Spiele, die ich mir im Zuge angesehen habe

\section{Netzwerkinfrastrukturen}

enthält eine Liste von Möglichkeiten auf welcher Grundlage verschiedene Multiplayer Anwendungen gebaut werden können


\chapter{Verwandte Arbeiten}

\cite{harris_asymmetry_2019}
\cite{sajjadi_maze_2014}

hier würden Paper reinkommen die asymmetrische Multiplayer gemacht haben, welche aspekte da mitreinspielen, da kommen dann auch die wichtigen Begriffe dazu mitrein. Auch bereits umgesetzt asymetrische VR Spiele?


Auch Anna Lotz´ Thesis wäre hier relevant


\section{Wichtige Begriffe}

\subsection{Interdependence}
\cite{harris_leveraging_2016}
\cite{depping_cooperation_2017}

\subsection{Degrees of Interdependence}
\cite{beznosyk_effect_2012}

\subsection{Soziale Präsenz}

Vertrauen gibts in dem Kontext auch und wie man dan über Spiele aufbaut

\section{Untersuchungsschwerpunkte}
\chapter{Stand der Forschung}\label{sec:related-works}

% neuer plan: zunächst die beiden referenz paper zusammenfassen, dann die methodik für die litertaur recherche nennen; dann die ergebnisse als related works für die herleitung auch der forschungsfragern beantworten und dann noch die anderen paper die so gefunden wurden noch nennen; das für den ersten teil bei der forschungsfragen beanwtortet werden ,was alles relevant für die weitere entwicklung ist

% \section{Verwandte Arbeiten}\label{sec:related-works}
In der \ac{CSCW} und \ac{HCI} bzw. \ac{CHI} existieren bereits diverse Arbeiten zu Multiplayer-Spielen, Effekt von Computerspiele auf das soziale Beisammensein der Spieler und welche Aspekte aus dem Gamedesign dafür verantwortlich sind.

Im Zentrum steht dabei häufig die Frage, wie Spiele soziale Interaktionen fördern oder hemmen, sowohl im kompeteiven als auch im kollaborativen Kontexten.

Video- und Computerspiele im Allgemeinen können einen positiven Einfluss auf das Miteinander haben. So untersuchte \cite{mason_friends_2013} wie wichtig Freundschaften für den Erfolg von Einzelpersonen und Teams in komplexen kollaborativen Umgebungen sind. Sie fanden heraus, dass Freundschaften einen großen Einfluss auf die verbesserte individuelle- und Teamleistung haben. Spieler richten sich dabei nach sozialen Gelegenheiten aus, sodass verborgene Freundschaftsbeziehungen direkt abgeleitet werden konnten. Kern der Studie war dabei der Online Multiplayer First-Person-Shooter \say{Halo: Reach} bei dem Spieler des Spiels eine anonyme Online Umfrage Fragen ausfüllen mussten. 

Doch soziale Dynamiken verlaufen nicht immer so positiv wie erhofft. Andere Untersuchungen zeigen ein differenzierteres Bild des Zusammenspiels in Onlinewelten.

So argumentiert \cite{ducheneaut_alone_2006} anhand einer Langzeitstudie zu \say{World of Warcraft}, dass soziale Aktivitäten in \ac{MMOG}s, oft überschätzt werden. Die meisten Spieler sind zwar von anderen umgeben, interagieren jedoch nur selten aktiv miteinander. Sie spielen häufig \say{allein zusammen}. Vor allem in den Quests zum Anfang ist das oft der Fall. Erst durch langfristige soziale Strukturen wie Gilden entstehen nachhaltige Bindungen und echte Zusammenarbeit.

Damit jedoch solche sozialen Beziehungen überhaupt entstehen können, ist es essenziell, dass Spiele die Aufmerksamkeit und das Interesse der Spielenden wecken - ein Aspekt, der unter dem Begriff \say{Player Engagement} intensiv erforscht wird.

\cite{rashed_review_2025} fassen in ihrer Überblickasarbeit unterschiedliche Methoden zur Schätzung des Spieler-Engagements zusammen. Ihr Ziel war es, über verschiedene Messmethoden wie EEG, Mimik, Eye Tracking und Spieler-Verhalten hinreichend eindeutige Daten zu sammeln um darüber eine Aussage über das Engagement treffen zu können. Die Validierung der Ergebnisse, da das Engagement subjektiv ist, ist schwer um objektiv eine \say{Ground Truth} Aussage treffen zu können.  \cite{yu_video_2023} verfolgten einen anderen Ansatz. Sie versuchten nicht nur auf das Engagement der Spieler einzugehen, sondern erforschten direkt im Bereich Zusammenarbeit und Kollaborative Fähigkeiten. Sie untersuchten kommerzielle Multiplayer-Spiele um Konzepte und Spielmechaniken zu identifizieren, die von Game-Designern zur Förderung von kooperativen Spielen genutzt werden können. Im Zuge der Forschung entwickelten sie kleine Prototypen und führten mit ihnen kleine Studien durch. 

Einige Studien gehen noch einen Schritt weiter und untersuchten nicht nur Engagement, sondern die spezifischen Bedingungen erfolgreicher Kooperation in Spielen - insbesondere durch das Design asymmetrischer Rollenverteilungen.

So zeigen die Arbeiten von \cite{harris_beam_2014}, \cite{harris_leveraging_2016} und \cite{harris_asymmetry_2019}, dass asymmetrische Spielkonzepte - bei denen sich Rollen, Fähigkeiten und Ziele der Spielenden unterschieden - einen positiven Einfluss auf die Zusammenarbeit hat. Untersucht werden dabei die Faktoren \say{Interdependence}, \say{Degrees of Interdepencene} sowie Mechaniken der Asymmetrie und Abhängigkeiten der Anwendungen. Ein asymmetrisches Spielkonzept ermöglicht außerdem eine Integration bzw. Inklusion von Spielergruppen mit eingeschränkten Fähigkeiten (vgl. \cite{goncalves_exploring_2021}). Für die Entwicklung von Spielen, die für die gesamte Familie gedacht sind, eignet es sich ebenfalls (vgl. \cite{pais_promoting_2024}).

Die Arbeiten von Harris et. al. dienen als Grundlage für die weitere Entwicklung des Game Designs für asymmetrische Multiplayer-Spiele. So identifizierte \cite{guimaraes_rocha_game_2008} verschiedene kooperative Design-Pattern, die in der weiteren Forschung und deren Spielumsetzung anklang fanden. In der Arbeit von \cite{emmerich_impact_2017} werden drei der definierten Pattern verwendet um eine Aussage darüber treffen zu können, wie sich Interaktionen im Spiel gezielt gestalten lassen. Die Ergebnisse der Studie zeigen, dass eine hohe Spielerinterdependenz mit mehr Kommunikation und weniger Frustration einhergeht. Geteilte Kontrolle führte jedoch zu einem geringeren Erleben von Kompetenz und Autonomie.

Diese gestalterischen Grundlagen bilden einen Ausgangspunkt für eine weiterführende Forschung, die sich nun den sozialen, psychologischen und metastrukturellen Wirkungen dieser Spielkonzepte widmet.

Die Arbeit von \cite{depping_trust_2016} beschäftigte sich mit dem zwischenmenschlichen Vertrauen innerhalb einer zusammenarbeitenden Gruppe. Der Fokus lag dabei auf der Problematik, dass im Online-Umfeld bewährte Methodiken zum Teambildung nur schwer umsetzbar sind und bestimmte Situationen einfacher simuliert werden müssen. Daher wurde durch Einsatz eines sozialen Spiels bestimmte Situation wie Risikosituationen und gegenseitige Abhängigkeiten simuliert. Das Zusammenarbeiten im Team kann auch eine Quelle von Konflikten oder Veränderungen sein. \cite{velez_ingroup_2014} zeigen den Fall, dass eine (neue) fremde Person zu einer bestehenden Gruppe Spannungen erzeugen kann. Ihre Studie belegtr, dass kooperative Spiele nicht nur das Helferverhalten steigert, sondern auch das Aggressionsverhalten gegenüber Mitgliedern einer Fremdgruppe verringern kann.

In der Forschung von \ac{VR}-Spielen entstanden einige Interessante Arbeiten bezüglich des Game-Designs aber auch der enthaltenen Forschung.

\cite{karaosmanoglu_playing_2023} untersuchten die Vertrautheit von Zweierteams, die aus sich Fremden oder befreundeten Personen bestanden, im Zusammenhang mit sozialen und spielerischen Erfahrungen sowie ihrer Spielleistung. Die Studie ergab, dass es keine signifikanten Unterschiede zwischen den Freundeteams und Fremdenteams gab. Um Zusammenarbeit ging es ebenfalls in der Anwendung von \cite{sajjadi_maze_2014}. Die Ergebnisse der Studie zeigen, dass das konzipierte Spielkonzept bei den Spieler-Rollen mit den Sifteo Cubes und der VR Anwendung für die Oculus Rift eine positive Bewertung sowohl des Spielerlebnisses als auch der Zusammenarbeit ergab. Ebenfalls mit dem Bezug auf die Zusammenarbeit beschäftigte sich die Arbeit von \cite{smilovitch_birdquestvr_2019}, bei der es darüber hinaus um das Ausschöpfen der Möglichkeiten von \ac{VR} ging.

Im Kerngebiet der Kommunikation beschäftigte sich \cite{nasir_cooperative_2013} und \cite{nasir_effect_2015} zunächst mit der Entwicklung eines \say{ice-breaking} Spiels, das in Form eines 2D-\ac{RPG} konzipiert und entwickelt wurde. Der Sinn des Spiels ist dabei, die Zusammenarbeit in einer folgenden Gruppenarbeit zu verbessern. In der Studie wurden dabei drei unterschiedliche Gruppen miteinander verglichen (eine Gruppe hat das konzipierte Spiel gespielt, eine weitere hatte ein generisches ice-breaking Spiel gespielt und die dritte Gruppe keins). Die Gruppen, die das konzipierte Spiel gespielt hatten, zeigten eine erhöhte Interaktion. Die fortführende Studie untersuchte, ob das aus der ersten Studie umgesetztee Spiel die Zusammenarbeit in realen Teams verbessern kann. Es wurden dabei Gruppen verglichen, die vor der Arbeitsaufgabe das konzipierte ice-breaking gespielt hatten, mit denen, die es nicht gespielt hatten. Es wurde festgestellt, dass die Gruppen, die das ice-breaking Spiel spielten, in der anschließenden Arbeitsaufgabe eine erhöhte Zusammenarbeit zeigten.
% zunächst mit der Wirkung eines kooperativen \say{ice-breaking} Spiel (Kennenlern-Spiel), das als Instrument dienen soll die Zusammenarbeit zu verbessern. Die Ergebnisse der Studie zeigten, dass die Gruppe, die das Kennenlern-Spiel gespielt hatten, mit erhöhter Interaktion an der Folgeaufgabe teilgenommen hatten, als die Vergleichsgruppe. In der darauffolgenden Arbeit 

% Die \ac{AR}-Forschung beschäftigte sich ebenfalls mit 

% [weiterführen dann mit den papern von den design pattern, da die hier dazu kommen, dann beispiele bringen die das als grundlage nehmen im übertragenene sinne]

% Eine der Arbeiten im Themengebiet dieser Arbeit wurde von \cite{nasir_effect_2015}, bei der es darum geht eine \say{Icebreaking}-Anwendung in Form eines Computer- und Videospiels zu haben um erste Hürden im Kennenlernen und effektiven gemeinsamen Arbeiten zu überwinden. Es zeigte sich, dass die Gruppe, die das Icebreaking-Videospiel gespielt hatte, eine erhöhte Zusammenarbeit. [Fortsezen mit inhalt des spiels und dann die forschung beschreiben]

% \cite{harris_asymmetry_2019}
% \cite{sajjadi_maze_2014}

% hier würden Paper reinkommen die asymmetrische Multiplayer gemacht haben, welche aspekte da mitreinspielen, da kommen dann auch die wichtigen Begriffe dazu mitrein. Auch bereits umgesetzt asymetrische VR Spiele?


% Auch Anna Lotz´ Thesis wäre hier relevant


\section{Wichtige Begriffe}
In den vorangegangenen Arbeiten beschäftigten sich die Autoren mit einigen Begrifflichkeiten, die Grundlage in der Konzeption und Entwicklung dieses Prototyps sowie der Forschung dieser Arbeit sind. 

In den folgenden Kapiteln werden diese Begriffe erklärt.

\subsection{Interdependence}
Der Begriff \say{Interdependence} stammt aus dem psychologischen Rahmenwerk für soziale und gruppenbezogene Interaktionen. Die Interdependence wird über das Ausmaß, in dem Gruppenmitglieder aufeinander angewiesen sind, um ihre Aufgabe effektiv zu erfüllen, definiert \cite[S. 451]{depping_cooperation_2017}, \cite{saavedra_complex_1993}, \cite[S. 197:4]{holly_asymmetric_2023}. Auf Video- und Computerspiele bezogen, können Aufgaben als das Spielziel bezeichnet werden \cite[S. 451]{depping_cooperation_2017}. 
In \cite[S. 52]{van_der_vegt_patterns_2001} werden unterschiedliche Formen der Interdependence vorgestellt:
\paragraph{Task interdependence} beschreibt die Abhängigkeit von Teammitgliedern in ihren Aufgaben, die sie zu tun haben. Der Grad der Abhängigkeit nimmt zu, je komplexer die Aufgabe wird.
\paragraph{Goal interdependence} beschreibt die quantitativen und qualitativen Leistungen, die von den Gruppenmitgliedern gemeinsam erreicht werden müssen, um das Gruppenziel zu erreichen.

[Hier schauen ob noch in den anderen Papern was dazu steht]

% \begin{itemize}
% \item \textbf{Task interdependence}: 
%     % \item \textbf{Kann eine spielbasierte Umgebung für die Untersuchung und Verbesserung von Kommunikation zwischen zwei oder mehreren Personen realisiert werden?}
%     % \item \textbf{Welche spezifischen Eigenschaften muss eine solche Umgebung aufweisen und welche Kommunikationsparameter werden dabei angesprochen?}
%     % \item \textbf{Welche Verbesserungen in der Kommunikation zwischen den Anwendern können durch ein asynchrones Multiplayer-Spiel mit zwei verschiedenen Spielerklassen beobachtet werden?}
%     % \item \textbf{Welche Unterschiede können in der Art das Kommunikationsverhalten bei der Verwendung von zwei unterschiedlichen Anwendungen (AR und 3D) (festgestellt/beobachtet) werden}
%     % \item \textbf{Wie stehen die Nutzer zu einem spielerischen Ansatz und zur Verbesserung der Kommunikation, insbesondere auch im Umgang mit Fremden?}
% \end{itemize}
% \cite{harris_leveraging_2016}
% \cite{depping_cooperation_2017}

\subsection{Degrees of Interdependence}
In der Arbeit von \cite[S. 7]{harris_asymmetry_2019} werden unterschiedliche Grade der Interdependence untersucht. Unter Grad der interdependence versteht man das Ausmaß in dem die Handlungen der Spieler voneinander abhängig sind, um das Spielziel erfolgreich zu erreichen. Je höher der Grad der Interdependence, desto stärker sind die Spieler darauf angewiesen, sich gegenseitig abzustimmen zusammenzuarbeiten um ihre Handlungen aufeinander abzustimmen (vgl. \cite[S. 7]{harris_asymmetry_2019}.

[Hier schauen ob noch in den anderen Papern was dazu steht]

% \cite{beznosyk_effect_2012}

\subsection{Soziale Präsenz}
Die soziale Präsenz beschreibt \say{das Gefühl, mit einem anderen zusammen zu sein} [eigene Übersetzung] \cite[S. 1]{biocca_towards_2003}. \say{Das andere} kann dabei entweder ein anderer Mensch oder eine künstliche Intelligenz sein. Innerhalb der \ac{HCI} untersucht die Theorie der sozialen Präsenz, wie das \say{Gefühl, mit einem anderen anderen zusammen zu sein} [eigene Übersetzung] \cite[S. 1]{biocca_towards_2003} durch Schnittstellen gestaltet und beeinflusst wird (vgl. \cite[S. 1]{biocca_towards_2003}). Sie wird im Einzelnen durch die Wahrnehmung der physischen Repräsentation des anderen Spielers sowie durch psychologische Beteiligung und Verhaltensabhängigkeiten gekennzeichnet. Soziale Präsenz kann somit als das Ergebnis eines komplexen Zusammenspiels von wechselseitigem Verhalten, Kommunikation und sozialen Kontextmerkmalen gesehen werden. Die Voraussetzung hierfür ist, dass ein Spieler die Kopräsenz einer anderen sozialen Einheit wahrnimmt (vgl. \cite[S. 1]{emmerich_game_2016}).

[Hier schauen ob noch in den anderen Papern was dazu steht]

% \cite{emmerich_impact_2017}

% Vertrauen gibts in dem Kontext auch und wie man dan über Spiele aufbaut

% \section{}#

\section{Forschungsbeitrag}

% In diesem Kapitel wird der Untersuchungsschwerpunkt dieser Arbeit vorgestellt. Dabei wird Bezug auf die bestehende Forschung aus dem Kapitel \emph{\nameref{sec:related-works}} genommen.

Diese Arbeit greift die grundlegende Forschung der Arbeiten von Nassir in \cite{nasir_cooperative_2013} und \cite{nasir_effect_2015} auf und erweitert diese durch Vor- und Nachtests derselben Versuchsgruppe, um zu beweisen, dass eine bestehende Gruppe durch die Anwendung des erstellten Prototyps gezielt in der gemeinsamen Kommunikation verbessert werden kann. Der Prototyp dient zudem nicht als stilisierte Anwendung für den Zweck des Ice-Breakings, sondern auch als Multiplayer-Spiel, das in der Freizeit gespielt werden kann.

Hinzu kommt, dass die Anwendung ein \say{Cross-Plattform} Multiplayer (vgl. Abbildung: \ref{fig:lotz_multiplayer_types}) ist, bei dem unterschiedliche Plattformen genutzt werden müssen und die damit einhergehende Wirkung mit untersucht werden soll. Im Fokus steht dabei die \ac{AR}-Integration einer Anwendung und die Touchsteuerung beider Anwendungen, die im Prototyp umgesetzt werden sollen.


\chapter{Stand der Technik} \label{sec:sota}

% \section{Artverwandte Spiele}
Nachdem die einzelnen Charakteristiken von Multiplayer-Spielen im Kapitel \emph{\nameref{sec:basics}} wurden, werden nun Spiele vorgestellt, welche im Rahmen dieser Arbeit näher betrachtet wurden.

Die Spielreihe \say{\textbf{We were here}} vom niederländischen Entwicklerstudio Total Mayhem Games beinhaltet asymmetrische Kooperative-Multiplayer-Spiele, bei denen zu zweit Rätsel und Hindernisse in der Spielwelt gelöst werden müssen um aus der Umgebung, in denen die Avatare der Spieler gefangen sind, zu entkommen. Dabei können die Spieler über ein \say{In-Game}-Walki-Talki miteinander kommunizieren. Zumeist ist es so, dass ein Spieler verschiedene Rätsel oder Hindernisse für sich hat, die er seinem Mitspieler beschreiben muss, damit dieser die passenden Antworten übermitteln oder Rätsel lösen kann. Die beiden Spieler befinden sich dabei in abgetrennten Räumen oder Gebieten innerhalb der Spielwelt (vgl. \cite{noauthor_we_nodate-1}; \cite{noauthor_total_nodate}).  

[hier it takes two und split fiction erwähnen]

Das Spiel \say{\textbf{The past within}} vom ebenfalls aus den Niederlanden kommenden Entwicklerstudio Rusty Lake ist ein asymmetrisches kooperatives Multiplayer-Spiel bei dem zwei Spieler gemeinsam in einer Sitzung sowohl in der Vergangenheit als auch in der Zukunft gemeinsam Rätsel lösen müssen um der Protagonistin und ihrem Vater zu helfen. Jeweils ein Spieler befindet sich dabei in einer 2D-Amnwendung, der andere in einer 3D-Anwendung. Es existiert die Möglichkeit, dass das Spiel von verschiedenen Plattformen aus gespielt werden kann (Cross-Plattform Spielbarkeit) (vgl. \cite{noauthor_past_nodate}). 

Das bereits in den vorangegangenen Kapiteln [Kapitel einbinden] erwähnte \say{\textbf{Keep Talking and Nobody Explodes}} ist ein asymmetrisches kooperatives Multiplayer-Spiel bei dem eine Person das Spiel besitzen muss damit es im Team gespielt werden kann. Das Spiel hat eine Besonderheit, da es ein Cross-Plattform Spiel ist, bei dem ein Teilnehmer (der Bombenentschärfen) eine Bombe entschärfen muss und die anderen Spielteilnehmer (die Experten) verschiedene Anleitungen von Bomben vorliegen haben. Die Aufgabe besteht darin die richtige Anleitung für die entsprechende Bombe zu finden und die Bombe innerhalb der vorgegebenen Zeit zu entschärfen. Im Spiel befindet sich jedoch nur der Bombenentschärfer, während die Experten die Anleitungen ausgedruckt durchschauen können (vgl. \cite{noauthor_keep_nodate}).

Im März 2025 erschien das Spiel \say{\textbf{Myrmidon}} vom Studio Studio Popot, welches ein asymmetrischer kooperativer Multiplayer ist, bei dem zwei Spieler zusammen, in zwei verschiedenen Rollen, miteinander spielen können. Eine Rolle ist dabei die Stop-Motion Puppe, welche in einer Stop-Motion Welt Hindernisse überqueren und über verschiedene Plattformen springen muss um ans Ziel zu kommen. Unterstützt wird die Puppe dabei vom Animator, der die Kulisse des Stop-Motion Films bedienen muss, damit die Puppe an ihr Ziel gelangt (vgl. \cite{noauthor_myrmidon_2024}).


% \section{Kooperative Gamedesign Pattern}
% FÜr die methodiische arbeit wichtig
% \subsection{Game Design Patterns}
% \cite{bjork_patterns_2005}

% \paragraph{Complementarity}

% \paragraph{Synergies}

% \paragraph{Abilities}

% \paragraph{Shared Goals}

% \paragraph{Synergies between goals}

% \paragraph{Special Rules for Player of the same Team}

% \paragraph{Camera Setting}

% \paragraph{Interacting with the same object}

% \paragraph{Shared puzzle}

% \paragraph{Shared characters}

% \paragraph{Special characters targetting lone wolf}

% \paragraph{Vocalization}

% \paragraph{Limited ressources}

% \paragraph{Einflussnahme}


% taxonomien Towards a Unified Taxonomy for Analog and Digital Escape Room Games hier vcrstellen; generell wie in der konzeption vorgegnagen wird
% \include{content/chapters/method}
\chapter{Analyse von artverwandten Spielen zur Konzeptentwicklung}
\say{Connecting-Minds} besitzt ein grundlegendes Spielkonzept, das durch die Analyse von artverwandten Spielen aus dem Kapitel \emph{\nameref{sec:sota}} und weiteren Spiele, die durch ihr Spieldesign als Vorlage für die Anwendungen von \say{Connecting-Minds} als Vorlage dienen können, erweitert werden soll.

Außerdem soll durch durch die Ergebnisse der Analyse die Forschungsfrage \emph{\say{Welche spezifischen Eigenschaften muss eine solche Umgebung aufweisen und welche Kommunikationsparameter werden dabei angesprochen?}} beantwortet werden können.

\section{Methodik}
Zur systematischen Untersuchung der artverwandten Spiele wurde ein mehrstufiges, eigens entwickeltes Analyseraster verwendet. Dieses gliedert sich in folgende Abschnitte:

\subsection{Visuelle Analyse}
Im visuellen Design wurde zunächst der Fokus auf sichtbare Rätsel- und Hinweiselemente der Spiele gelegt. Dabei wurden diese jeweils im Bild markiert. Die Markierungen dienten der späteren Auswertung, welche Aspekte dabei auffielen oder welche Arten von Rätseldesign genutzt wurden. Zusätzliche wurden so auch einzelnen Interaktionsflüsse erkannt und aufgereiht.

\subsection{Erstellung eines Diagramms zur Rätselstruktur}
Im zweiten Schritt nach der visuellen Analyse wurden für bestimmte Abschnitte oder das ganze Spiel UML-Ablauf Diagramme angelegt, die den Aufbau und die Verschachtlungen der Rätselelemente aufzeigen sollen. Hierbei wurde sich an der Arbeit von \cite{tim_schafer_grim_1996} orientiert.

\subsection{Deskriptive Übertragung}
Im dritten Schritt wurden Erkenntnisse zum Rätseldesign aus Schritt 1 (Visuelle Analyse) und Schritt 2 (Erstellung des Diagramms) zusammengefasst.

\subsection{Schlussfolgerung}
Im letzten Schritt wurden die Beobachtungen aus den vorangegangenen Schritten in Bezug auf die Konzeptentwicklung von \say{Connecting-Minds} gesetzt und Rückschlüsse in das eigene Konzept eingearbeitet.

\section{Ergebnisse der Analysen}
In den einzelnen Kapiteln werden nun die Analysen zu den Spielen \say{We were here}, \say{We were here too}, \say{The past within}, \say{Tiny Room Stories}, \say{Myrmidon}, \say{Outlanders} und \say{Keep Talking and Nobody Explodes} vorgestellt.

\subsection{We were here - Spielreihe}
Es wurde sich hier nur auf die ersten zwei Teile (\say{We were here} und \say{We were here too}) der Spielreihe beschränkt, da der Umfang alle Spiele zu spielen und zu analysieren zu wenig Ertrag ergeben würde und daher in der Breite der Spiele analysiert wurde.

\paragraph{Visuelle Analyse}
Abbildungen \ref{fig:wwh-visuals} in Anhang [Anhang verlinken] und \ref{fig:wwht-visuals} in Anhang [Anhang verlinken] geben einen Einblick in die Analyse der Spieler-Rollen der Spiele. Schnell fällt auf, dass die Rätselelemente der Spiele sehr häufig über Symbole oder Bilder gestaltet wurde, bei denen diese beschrieben werden, richtig ausgewählt und platziert werden müssen. Grundlegend sind die Rätsel in der Spielwelt eingebaut, das bedeutet, dass entweder Symbole an den Wänden zu sehen sind, oder Anweisungen in Textform in der Spielwelt integriert sind. Selten ist es so, dass durch auditives Storytelling oder Texte in Büchern Rätsel gelöst werden müssen. In \say{We were here} fällt zudem auf, dass der \say{Explorer} die eingebauten Rätsel lösen muss und der \say{Librarian} sich in einer Hub befindet, über welchen der zu allen Rätseln die richtige Lösung finden muss. In \say{We were here too} ist es ähnlich. Häufig befindet sich der \say{Peasant} in Räumen mit den Rätseln und der \say{Lord} in den Räumen mit den Lösungen. Allerdings wurde hier öfters auch der \say{Lord} durch das Lösen von Rätseln gefordert. Außerdem befindet er sich nicht in einem zentralen Hub.

\paragraph{Analyse des Rätseldesigns}
Abbildung \ref{fig:wwh-uml} und \ref{fig:wwht-uml} zeigen jeweils die Ablaufdiagramme von \say{We were here} und \say{We were here too}. Auffallend ist, dass in \say{We were here} die Rolle des \say{Librarian}s im Kontext des Rätseldesigns linear verläuft und nur selten verschachtelte Rätsel aufzufinden sind. Grundlegend ist es so, dass der \say{Explorer} auf die Rätsel und Hindernisse trifft, und der \say{Librarian} dafür zuständig ist, die richtigen Antworten zu liefern. Es findet eine sehr eindimensionale Verkettung zwischen den Rollen statt.
In \say{We were here too} lassen sich im Vergleich zum Vorgänger-Teil öfters verkettetes Rätseldesign finden, zum Beispiel in Raum 2 des Spiels. Dies ist jedoch auch in Ausnahmen der Fall, es wiederholt sich ein ähnlicher Aufbau wie im Vorgänger.

\begin{figure}[ht]
\centering
\includegraphics[width=0.8\linewidth]{content/pictures/WeWereHereUML.png}
\caption{Rätseldesign von We were here (Quelle: eigene Darstellung), vollständig in Anhang: }
\label{fig:wwh-uml}
\end{figure}

\begin{figure}[ht]
\centering
\includegraphics[width=0.8\linewidth]{content/pictures/WeWereHereTooUML.png}
\caption{Rätseldesign von We were here too (uelle: eigene Darstellung), vollständig in Anhang: }
\label{fig:wwht-uml}
\end{figure}

\paragraph{Schlussfolgerung}
Der zentrale Aspekt der Kommunikationsaufforderung ist in beiden Spielen erkennbar und bildet eine gemeinsame Grundlage. In \say{We were here} fällt jedoch auf, dass das Rätseldesign häufig monoton und eindimensional wirkt - \say{Explorer} stößt auf Rätsel, beschreibt gegebene und/oder gesuchte Gegenstände, \say{Librarian} geht in bestimmten Raum und beschreibt gesuchte Gegenstände, \say{Explorer} löst Rätsel. Es existieren zwar einzelne Ausnahmen, in denen der \say{Librarian} aktiv werden muss um dem \say{Explorer} weiterzuhelfen - \say{Librarian} muss das richtige Ventil vom Rohr öffnen. Doch diese wechselseitige Abhängigkeit bleibt die Ausnahme. Gerade diese Form der Verzweigung ist für \say{Connecting-Minds} essenziell und sollte eine tragende Rolle Spielen. Auf dieser Grundidee lässt sich aufbauen, auch wenn nicht jedes Rätsel in seiner konkreten Ausgestaltung überzeugt, bieten bestimmte Ansätze dennoch wertvolle Anregungen - insbesondere im Hinblick auf die Möglichkeit, abwechslungsreiche, variantenreichere und stärker auf Kooperation ausgelegte Rätsel für \say{Connecting-Minds} zu entwickeln.

Deutlich weiter geht das zweite Spiel, \say{We were here too}, in dem die gegenseitige Verflechtung zwischen den beiden Rollen wesentlich häufiger zum Tragen kommt. In vielen Fällen ist es der \say{Peasant}, der nicht nur seinen eigenen Weg zum nächsten Raum freischaltet, sondern zugleich auch den vom \say{Lord}. In zwei Fällen ist das Prinzip sogar umgekehrt gestaltet: Der \say{Lord} ermöglicht dem \say{Peasant} den Zugang zu neuen Bereichen. In einem der beiden Fällen muss der \say{Peasant} eine Wendeltreppe hinaufrennen, unter der sich der Boden langsam einzieht. Er kommt jedoch nur bis zu einer Schwerwand, die über einen Würfel geöffnet werden kann. Er muss dem \say{Lord} den Aufschnitt des Würfels beschreiben, welcher den richtigen Würfel auswählen muss und in die Zielablage ablegen muss. Durch die Verschachtlung wird der Grundgedanke der Kommunikation und Zusammenarbeit deutlich verstärkt hervorgehoben und erzeugt ausgeglichenen Spaß in den Anwendungen. Ein weiterer Zusatz des Rätseldesigns ist das aufeinander Aufbauen von Rätseln. Dieses Element wird im Rahmen dieser Schlussfolgerung als \say{Mehrstufigkeit} bezeichnet und ist zum Beispiel in Raum 2 zu finden, bei dem aneinander gekettet, verschiedene Rätsel gelöst werden müssen.

Diese strukturelle Ausrichtung eignet sich grundsätzlich gut als gestalterische Vorlage, sofern sie sich sinnvoll in \say{Connecting-Minds} übertragen lässt. Gleichzeitig zielt \say{Connecting-Minds} auf eine noch tiefere und kontinuierliche Abhängigkeit zwischen den beiden Spielerrollen. Beide Rollen sollen nicht nur gelegentlich, sondern regelmäßig und systematisch aufeinander angewiesen sein - die Zusammenarbeit wird damit zur unverzichtbaren Grundlage des Spielfortschritts. Aus der engen Verflechtung ergibt sich, dass Kooperation nicht nur hilfreich, sondern spielentscheidend wird.

Konzepte wie mehrstufige Rätsel oder Einsatz von zeitlichen Begrenzungen (Timer) sind in diesem Zusammenhang ebenfalls interessante Überlegungen. Während Mehrstufigkeit sich als zentrales Element für die Rätselgestaltung anbietet, sollte der Timer gezielt und situationsabhängig eingesetzt werden, da sein Einfluss stark vom jeweiligen Spannungsbogen und dem beabsichtigten Spielerlebnis abhängt.
\chapter{Konzeption und Aufbau des Prototyps}\label{sec:concept}
Nachdem in der Literaturrecherche (Kapitel \textit{\nameref{sec:related-works}}) zentrale funktionale und gestalterische Anforderungen identifiziert und in Kapitel \textit{\nameref{sec:analysis}} vergleichbare Spiele hinsichtlich ihres Game- und Rätseldesigns analysiert wurden, kann \say{Connecting-Minds} nun auf Basis dieser Erkenntnisse vollständig konzipiert werden. Ergänzend fließen die von \cite{krekhov_puzzles_2021} entwickelte Taxonomie für analoge und digitale Escape-Room-Spiele in den Gestaltungsprozess mit ein.

Die Konzeption des Spiels folgt einem systematisch-methodischen Vorgehen. Zunächst werden die übergeordneten Designvision sowie die grundlegenden Zielsetzungen erläutert. Darauf aufbauend dient das \ac{MDA}-Framework (vgl. \citealp{hunicke_mda_2004}) als zentrales Analyse- und Strukturierungsinstrument, um die angestrebte Spielerfahrung gezielt gestalten zu können. In diesem Rahmen werden die grundlegenden Spielmechaniken, die Rollenverteilung sowie die angestrebten dynamischen Prozesse beschrieben. Im Anschluss folgen die Darstellung der technischen Verbundenheit der Anwendungen, das Konzept für das Tutorial, Überlegungen zum Dialog- und Sounddesign sowie eine abschließende Reflexion über die Ideen und Ansätze, die im finalen Prototyp keine Berücksichtigung mehr gefunden haben.

\section{Designziele und Zielgruppe}
Connecting-Minds verfolgt das Ziel, kooperative Kommunikation unter asymmetrischen Perspektiven in einem Escape-Room-ähnlichen Szenario zu fördern. Das Spiel basiert auf der Zusammenarbeit zweier Rollen, Player und Watcher, die gemeinsam Rätsel lösen und Hindernisse in der Spielwelt überwinden müssen. Beide Rollen verfügen über unterschiedliche Wahrnehmungen und Interaktionsmöglichkeiten innerhalb der Spielwelt, die sich gegenseitig ergänzen und auf Kooperation angewiesen sind.

Das Spiel lässt sich dem Genre der kooperativen Adventure-Spiele zuordnen. Außerdem soll es sich dabei, wie im ersten Konzept des Vorabprojekts, um ein \ac{Sci-Fi} Abenteuer handeln.

Im Zentrum des spielerischen Erlebnisses steht die gezielte Verteilung asymmetrischer Informationen, um eine Balance zwischen Orientierung und Vertrauen zu schaffen. Der gemeinsame Fortschritt bildet dabei den zentralen Motivationsfaktor.

Die Zielgruppe entspricht derjenigen, die bereits in der vorangegangenen Konzeptionsphase im Rahmen des Moduls Interaktionsdesign im \ac{UCD}-Prozess entwickelt wurde. Im Fokus stehen drei exemplarische Personae:

\begin{itemize}
    \item \textbf{Steve Works}, 19 Jahre alt, ist Studienganganfänger im Fach Medieninformatik. Neben seinem akademischen Interesse sucht er gezielt nach sozialer Interaktion. Spieleabende und gemeinschaftliche Aktivitäten betrachtet er als Möglichkeit, Kontakte zu knüpfen und den Studienalltag aktiv zu gestalten.
    \item \textbf{Uwe Kaufmann}, 64 Jahre alt, ist erfahrener Projektleiter. Er steht vor der Aufgabe, ein neues Team zusammenzustellen und sieht in \say{Connecting-Minds} eine Gelegenheit, Teambuilding und Motivation zu fördern, um eine effektive Zusammenarbeit zu etablieren.
    \item \textbf{Anja Gayms}, 31 Jahre alt, arbeitet als introvertierte Zahnarzthelferin. Sie sucht im Spiel sowohl eine kognitive Herausforderung als auch eine Gelegenheit, bestehende Freundschaften zu vertiefen.
\end{itemize}

Die vollständige Ausarbeitung der Personae befinden sich im Anhang \ref{sec:append_concept_personae}: \nameref{sec:append_concept_personae}.

\section{Narratives und funktionales Grundgerüst}
Das Spiel basiert auf einem asymmetrischen Zwei-Rollen-Prinzip, bei dem zwei Spieler unterschiedliche Rollen einnehmen und gemeinsam innerhalb einer geteilten Spielwelt agieren. Diese Welt ist in mehrere räumlich und funktional voneinander abgegrenzte Abschnitte gegliedert. Der Fortschritt im Spiel wird durch kooperatives Handeln und das gemeinsame Lösen von Rätsel ermöglicht. Dabei ist die wechselseitige Abhängigkeit beider Rollen wesentlich für das Vorankommen.

Die narrative Struktur des Spiels entfaltet sich durch eine Kombination aus textbasierten Hinweisen, Umweltinformationen und der räumlichen Gestaltung. Das Storytelling ist dabei stark an das sog. \say{Environmental Storytelling} angelehnt, bei dem die Umgebung selbst narrative Funktionen übernimmt.

Die zugrundeliegende Hintergrundgeschichte lautet wie folgt:

Der Protagonist des Spiels nimmt an einer experimentellen Simulation innerhalb seines Forschungsinstituts teil. Diese Simulation verläuft jedoch nicht wie geplant. Durch eine unerwartete Anomalie während des Prozesses wird das Selbst des Protagonisten gespalten. Zurück bleibt der physische Körper in der realen Welt, während das Bewusstsein in das digitale Netz der Forschungseinrichtung übertragen wird. Beide Entitäten, Körper und Geist, existieren fortan getrennt, können jedoch auf bislang unerklärliche Weise miteinander kommunizieren. Ziel beider Instanzen ist es, die Ursache der Anomalie zu ergründen, den oder die Verantwortlichen ausfindig zu machen und schließlich die eigene Wiedervereinigung herbeizuführen.

Die Spielwelt bildet diesen narrativen Rahmen architektonisch und funktional ab. Beginnend in einem alten, unterirdischen Gewölbe des Forschungsinstituts, in das der leibliche Körper nach dem fehlgeschlagenen Experiment gebracht wurde, arbeiten sich die beiden Rollen durch verschiedene Abteilungen der Einrichtung. Dabei sammeln sie Hinweise auf die Hintergründe des Vorfalls und identifizieren mögliche Antagonisten. Im weiteren Verlauf öffnet sich die Spielwelt sukzessive. Sie führt zunächst durch unterschiedliche Gebäudeteile der Forschungseinrichtung, anschließend in Außenareale sowie in die privaten Wohnräume von Personen, die in den Vorfall verwickelt sein könnten. Die Erweiterung der Spielwelt entsteht dabei stets im direkten Zusammenhang mit dem narrativen Fortschritt.


\section{Spielkonzeption mithilfe des MDA-Frameworks}

Aus den bisherigen Recherchen zu asymmetrischen kooperativen Spielen und der zugrunde liegenden Spielerkommunikation geht hervor, dass zwischen den Spielerrollen eine funktionale oder perspektivische Abhängigkeit bestehen muss. Diese Abhängigkeit sollte jedoch nicht zu stark ausgeprägt sein, da ansonsten der Spielfluss beeinträchtigt und Frustration bei den Spielenden hervorgerufen werden könnte. Eine gelungene Balance zwischen Abhängigkeit und Eigenständigkeit der Rollen ist somit essenziell für ein kooperatives und motivierendes Spielerlebnis.

Die Analyse verwendeter Spielkonzepte verdeutlicht, dass bestimmte Inhalte und Funktionalitäten sinnvoll integriert werden können, Gleichzeitig muss jedoch vermieden werden, dass die Rollen zu stark voneinander entkoppelt agieren, da dies die kooperative Interaktion minimieren würde.

Die Spielreihe We were here zeigt, dass Rätselelemente so gestaltet sein müssen, dass sie durch die jeweils andere Spielpartei beschrieben und nachvollziehbar erklärt werden können. Zusätzlich zeigt sich, dass eine höhere Interaktionsdynamik zwischen den Anwendungen der beiden Rollen dazu beiträgt, eine engere Verzahnung von Spielerfahrung und Spielmechanik zu erreichen.

Das Spiel Tiny Room Stories demonstriert, wie sich kleinere Rätselelemente und das schrittweise Freischalten von Hindernissen zu einem übergeordneten Ziel zusammenfügen lassen. Darüber hinaus dient er als Inspirationsquelle hinsichtlich der Steuerungsmechanik und der intuitiven Benutzerführung.

The Past Within macht deutlich, dass ein zu hohes kognitives Anforderungsniveau einzelner Anwendungen die kooperative Kommunikation negativ beeinflussen kann. In solchen Fällen neigen Spieler dazu, sich ausschließlich auf ihre eigene Anwendung zu konzentrieren, wodurch die Sensibilität für kooperative Momente, als Zeitpunkte, an denen die Unterstützung durch die andere Rolle notwendig wäre, verloren geht.

Myrmidon hingegen zeigt, wie stark voneinander abhängige Anwendungsbereiche grundsätzlich gestaltet werden können. Allerdings leidet in diesem Fall das Spielerlebnis unter einer unausgewogenen Rollenverteilung. Die Spielerrolle des Animators nimmt vorwiegend eine unterstützende Funktion für die andere Rolle (die Stop-Motion-Puppe) ein und hat dadurch nur eingeschränkt eigene Spielanteile.

Ein ähnliches Ungleichgewicht lässt sich bei Keep Talking and Nobody Explodes beobachten. Auch hier übernimmt die Expertenrolle hauptsächlich eine beratende Funktion für den Bombenentschärfer, ohne selbst unmittelbar in das Spielgeschehen eingebunden zu sein.

\subsection{Mechanics}
Die mechanischen Elemente des Spiels lassen sich in drei Kategorien einteilen: Die Spielerrolle des Players, die des Watchers sowie allgemeingültige Weltregeln, die für beide Rollen relevant sind.

\paragraph{Player}

Der Player steuert seinen Avatar in der Spielwelt entweder über eine klassische Maussteuerung im Point-and-Click-Stil oder über Touch-Inputs. Die Kameraperspektive kann über das Mausrad bzw. Zoom-Gesten angepasst werden und zwischen einer standardmäßigen isometrischen Ansicht und einer First-Person-Perspektive wechseln.

Das \ac{UI} des Players umfasst eine Toolbar, über welch Interaktionen mit Weltobjekten ausgelöst werden, etwa das Aufnehmen oder Platzieren von Gegenständen. Zusätzlich ist es dem Player möglich, bestimmte Objekte zu tragen und an den Watcher zu übermitteln. Über die First-Person-Ansicht können Gegenstände präzise in der Spielwelt platziert werden.

Neue Objekte in der Spielwelt werden durch physische Annäherung des Avatars freigeschaltet, sobald eine Interaktion möglich ist.

\paragraph{Watcher}

Der Watcher interagiert über eine \ac{AR}-Anwendung mit der Spielwelt, die im physischen Raum vor ihm verankert ist. Er kann sich frei um die virtuelle Szene bewegen und erhält eine übergeordnete Perspektive auf die Raumstruktur sowie auf platzierte oder gesammelte Objekte.

Die Hauptaufgabe des Watchers besteht in der Verwaltung des Objektinventars, dem Platzieren und Entfernen interaktiver Gegenstände sowie dem gezielten weiterleiten von Objekten an den Player. Entfernte Gegenstände wandern zurück in ein Inventar, auf das ausschließlich der Watcher Zugriff hat. Über Touch-Inputs kann er Objekte an beliebige Stellen in der Spielwelt positionieren.

Im späteren Verlauf erhält der Watcher zudem die Möglichkeit, Gegenstände zu skalieren oder zu rotieren.

\paragraph{Weltregeln}

Leichte Gegenstände können vom Player aufgenommen, getragen und entweder über die Interaktionsleiste oder in der First-Person-Ansicht platziert werden. Schwere Objekte hingegen können ausschließlich vom Watcher positioniert, skaliert und rotiert werden, während der Player mit ihnen lediglich interagieren, sie aber nicht tragen kann. 

Sobald ein Gegenstand entdeckt oder in der Spielwelt platziert wurde, kann er vom Watcher (wieder) entfernt und dem Inventar hinzugefügt werden.

Darüber hinaus existieren Hinweise in Form von Texten oder Bildern, die der Player in der Spielwelt entdecken und zur weiteren Analyse an den Watcher weiterleiten kann. Diese Hinweise ergänzen das räumliche eingebettete Rätseldesign der Umgebung und fördern die kooperative Interaktion zwischen den beiden Rollen.

\subsection{Dynamics}

Die aus den Spielmechaniken resultierenden Dynamiken beruhen auf der asymmetrischen Verteilung von Perspektiven und Informationen zwischen den beiden Spielerrollen. Während des Player primär aus der Spielwelt heraus handelt, nimmt der Watcher eine übergeordnete, räumlich flexible Perspektive ein. Diese Asymmetrie bedingt, dass beide Rollen jeweils unterschiedliche Informationen erhalten und diese eigenständig, aber koordiniert interpretieren müssen.

Die Lösung von Herausforderungen erfordert somit die Kombination und wechselseitige Abstimmung beider Spielerfähigkeiten. Nur durch die koordinierte Nutzung der jeweils verfügbaren Informationen und Interaktionsmöglichkeiten entsteht ein funktionierendes kooperatives Zusammenspiel, dass das Fortschreiten im Spiel ermöglicht.

Besonders hervorzuheben ist die räumliche Dynamik, die durch den Einsatz der \ac{AR}-Technologie entsteht. Die Spielwelt wird in den physischen Raum des Watchers projiziert, wodurch eine neuartige räumliche Orientierung und Interaktionsform entsteht, die über klassische Bildschirmdarstellung hinausgeht. Diese physisch-virtuelle Verschmelzung unterstützt nicht nur die Immersion, sondern verstärkt auch die Notwendigkeit einer engen Abstimmung zwischen den Spielerrollen.

\subsection{Aesthetics}

Das Spiel adressiert die ästhetischen Dimensionen \textit{Challange}, \textit{Fellowship} und \textit{Expression} (vgl. \citealp[S. 3]{hunicke_mda_2004}). Im Vordergrund steht die Förderung von logischem Denken, sowie die Anregung intensiver Kommunikation und Koordination zwischen den Spielteilnehmern. Die asymmetrische Rollenverteilung verlangt ein hohes Maß an gegenseitigem Verständnis und Abstimmung, wodurch ein starkes Gefühl der Zusammenarbeit und des gemeinsamen Fortschritts (\textit{Fellowship}) entsteht.

Darüber hinaus eröffnet das Spiel Möglichkeiten zur \textit{Expression}, indem es die Spieler dazu einlädt, individuelle Kommunikations- und Problemlösungsstrategien zu entwickeln. Über den spielerischen Austausch hinaus kann so auch ein besseres Verständnis der eignene Stärken und bevorzugten Arbeitsweisen entstehen.

Das narrative Element fungiert als untergeordnete, aber zentrale Stütze für die Sinnhaftigkeit der gestellten Herausforderungen. Es verleiht den Spielmechaniken einen kohärenten Rahmen und motiviert die Spieler durch eingebettete Kontexte zur Auseinandersetzung mit den Aufgaben.

\section{Spielabläufe}

Die Spielabläufe lassen sich in zwei unterschiedliche Ebenen unterteilen. Zum einen existiert ein übergeordneter Ablauf, der den strukturellen Rahmen des gesamten Spiels definiert. Zum anderen verfügt jeder einzelne Spielabschnitt über einen spezifischen, in sich geschlossenen Ablauf. Diese werden in den folgenden Kapiteln vorgestellt.

\subsection{Ablauf des Spiels}

Die Spielabläufe der Anwendungen für Player und Watcher folgen in weiten Teilen einem identischen Schema, unterscheiden sich jedoch in einem zentralen Punkt. Dieser Unterschied wird im Folgenden erläutert.

Die Abbildung Player im Anhang \ref{sec:append_gameloop}: \nameref{sec:append_gameloop} zeigt das Aktivitätsdiagramm der Player-Anwendung. Nach dem Laden des Spiels erschient zunächst das Startmenü, über das verschiedene Funktionen zugänglich sind. Der Player kann die Einstellungen öffnen, eine neue Session starten, einer bestehenden Session beitreten oder das Spiel beenden. In den Einstellungen lassen sich unter anderem die Belegungen der Eingabeflächen (Tastatur bzw. Touch) sowie Audio- und Kameraoptionen anpassen.

Beim Start einer neuen Session wird die Prolog-Szene geladen, die ein kurzes Tutorial enthält. Dieses vermittelt grundlegende Spielregeln sowie die Steuerung. Die Spieler erhalten dabei Informationen über Interaktionsmöglichkeiten mit Gegenständen in der Spielwelt. Nähert sich der Avatar einem interaktiven Objekt, erscheint ein Tooltip, mit dem interagiert werden kann. Das interaktive Objekt kann ein Computerterminal sein oder etwas zum Tragen.

Das Tutorial kann bei Bedarf übersprungen werden, etwa wenn das Spiel bereits zuvor gespielt wurde. Innerhalb der Spielszene ist jederzeit der Zugriff auf ein Pausemenü möglich, das Optionen zur Änderungen der Einstellungen, zum Verlassen der Session oder zur Rückkehr ins Spiel bietet. Wird die Session verlassen, kehrt der Player zum Startbildschirm zurück und kann entweder einer existierenden Session erneut beitreten oder eine neue starten.

Nach Abschluss des Tutorials wird die darauffolgende Szene geladen und innerhalb der aktiven Session gespeichert. Wenn der Player die Session verlässt und später erneut beitritt, wird der Fortschritt geladen und fortgesetzt. Dieser Spielzyklus wiederholt sich bis zum letzten Abschnitt, an dessen Ende die Spielgeschichte abgeschlossen ist. Nach dem Abspann gelangt der Player zurück ins Hauptmenü, von wo aus neue Sessions gestartet werden können. Ein erneutes Beitreten zur beendeten Session führt automatisch zum Endbildschirm, von dem aus entweder das Spiel beendet oder zum Hauptmenü zurückgekehrt werden kann.

Die Anwendung des Watchers, von der das Aktivitätsdiagramm in der Abbildung Watcher im Anhang \ref{sec:append_gameloop}: \nameref{sec:append_gameloop} abgebildet wird, unterscheidet sich in einem wesentlichen Punkt. Nach dem Laden des Spiels ist es nicht möglich, selbst eine Session zu initiieren, stattdessen kann nur einer bestehenden Session beigetreten werden. Dies unterstreicht die konzeptionelle Rolle des Watchers als unterstützende Instanz im Spielgeschehen, nicht als gleichwertig agierender Avatar.

Im Prolog erhält der Watcher spezifische Instruktionen zu seinen Funktionen und Benutzeroberflächen. Über des Menü \say{Platzieren} kann er Objekte in der Spielwelt positionieren und diese über das Tooltip auch wieder entfernen. Zudem ist es ihm über das Menü \say{Preview} möglich, Gegenstände an den Player zu senden, die dieser anschließend trägt. Gleiches gilt für bereits entdeckte Objekte. Der Watcher kann sie ebenfalls entfernen, sobald sie vom Player entdeckt oder aufgenommen wurden. Entdeckte und platzierte Gegenstände bleiben für beide Spielerrollen in der Spielwelt sichtbar.

Abgesehen von den beschriebenen Unterschieden im Session-Management und der Aufgabenverteilung im Prolog, entspricht der restliche Spielablauf des Watchers dem der Player Anwendung.

\subsection{Ablauf des Levels}

Analog zum allgemeinen Spielablauf weist auch der Ablauf auf Ebene der einzelnen Level Unterschiede zwischen den Anwendungen der beiden Spielerrollen auf, die im Folgenden näher erläutert werden. 

Die Abbildung Player im Anhang \ref{sec:append_levelloop}: \nameref{sec:append_levelloop} visualisiert das Aktivitätsdiagramm des Players innerhalb einer Spielszene. Nach dem Laden der Szene befindet sich der Avatar des Players an einem bestimmten Ort in der Spielwelt. Da dieser vom Watcher aus dessen Perspektive heraus nicht direkt gesehen werden kann, ist der Player zunächst gefordert, seine Position verbal zu beschreiben. Parallel dazu oder im Anschluss beginnt er, die Umgebung zu erkunden und stößt dabei auf erste Hindernisse oder Rätsel, die er dem Watcher schildert. 

In der Spielwelt sind potenzielle Lösungselemente für die Herausforderungen verteilt, die vom Player entdeckt und beschrieben werden müssen. Sobald beide Rollen genügend Informationen gesammelt haben, beginnt die eigentliche Lösungsphase. Dabei kann es erforderlich sein, dass der Watcher spezifische Gegenstände in der erweiterten Spielwelt, die der Player nicht sieht, platziert, um einen Fortschritt zu ermöglichen. Alternativ muss der Watcher dem Player Objekte zusenden, die dieser wiederum korrekt in der Spielumgebung platziert. Nach erfolgreicher Bewältigung eines Rätsels oder Hindernisses wird der Zugang zu neuen Räumen freigeschaltet und die gemeinsame Erkundung setzt sich fort.

Das entsprechende Aktivitätsdiagramm des Watchers wird in der Abbildung Watcher im \ref{sec:append_levelloop}: \nameref{sec:append_levelloop} dargestellt. Der Ablauf unterscheidet sich in einzelnen Punkten von jenen des Players. Zunächst muss der Watcher durch die Beschreibung des Players dessen Standort rekonstruieren. Bereits während dieses Prozesses, sowie im weiteren Verlauf, kann er Hindernisse und Rätsel in seiner \ac{AR}-Ansicht identifizieren und deren Eigenschaften mit dem Player teilen. Dabei kann er auch unabhängig Überlegungen zu den erhaltenen Informationen anstellen, etwa zur Bedeutung oder Funktion der beobachteten Elemente.

Im Unterschied zum Player ist der Watcher verantwortlich für das Platzieren oder Weiterleiten relevanter Gegenstände zur Interaktion mit der Spielwelt. Erst durch diese koordinierte Zusammenarbeit beider Rollen können Rätsel vollständig gelöst und neue Bereiche zugänglich gemacht werden. Mit jedem neu freigeschalteten Raum wiederholt sich der beschriebene Ablauf zyklisch, wobei stets eine wechselseitige Kommunikation und Aufgabenverteilung erforderlich bleibt.


\section{Relation der Anwendungen}

In der bisherigen Konzeption wurden voranging die verschiedenen Rollen innerhalb des Spiels beschrieben, jedoch nicht spezifiziert, wie viele Nutzer jeweils eine Rolle pro Spielsitzung zugewiesen werden können. Ursprünglich sah die Planung vor, dass pro Session stets ein Player teilnehmen muss, während die Anzahl der Watcher theoretisch variabel und beliebig groß sein kann. Daraus ergab sich zunächst die Regel $1\ldots n$ \quad wobei $n \geq 1$.

Im Rahmen einer früheren Bachelorarbeit, die einen vergleichbaren Prototyp entwickelte, wurde im Evaluationskapitel jedoch kritisch angemerkt, dass eine Mehrfachbesetzung der Rolle des sog. \say{Smartphone-Nutzers} (Navigator) sich nachteilig auf das Spielerlebnis auswirkt (vgl. \citealp[S. 34]{lotz_konzeption_2021}). Da die Rolle des Watchers funktional stark mit jener des Navigators vergleichbar ist, insbesondere im Hinblick auf die Aufgaben der Orientierung, Anleitung und Unterstützung, erscheint eine Einschränkung auf eine Person pro Rolle sinnvoll.

Diese Annahme wird durch die Ergebnisse einer aktuellen Studie von \cite{bautista_isaza_understanding_2024} gestützt, in der der Einfluss unterschiedlicher Gruppengröße auf das Engagement und die wahrgenommene Arbeitsbelastung in einem Handheld-\ac{MR}- und \ac{VR}-Szenario untersucht wurde. Die Autoren kamen zu dem Ergebnis, dass kleinere Gruppen zu einem signifikant höheren Engagement führen (vgl. \citealp[S. 197:22]{bautista_isaza_understanding_2024}).

Basierend auf diesen empirischen Erkenntnissen sowie den Beobachtungen aus vergleichbaren Projekten wurde das ursprüngliche Konzept entsprechend angepasst. Die nun gültige Regel für die Spielzusammensetzung lautet: $1\ldots1$. Das bedeutet, dass pro Session genau ein Player gemeinsam mit genau einem Watcher spielt. Dieses Setup erlaubt eine klare Rollenzuweisung, fördert die Kommunikation zwischen den beiden Teilnehmern und schafft die Grundlage für eine kooperative Spielerfahrung.

\section{Konzeption des Tutorials}

Für den vorliegenden Prototyp wurde ein kompaktes Tutorial entwickelt, das sowohl dem Watcher als auch dem Player einen Einstieg in die grundlegenden Mechaniken und Funktionen des Spielkonzepts ermöglicht. Aufgrund des begrenzten Rahmens wurde keine vollständig selbsterklärenden Einführungselemente in die Anwendung integriert. Stattdessen übernahm der Versuchsleiter die Aufgabe, Steuerungselemente und Funktionalitäten manuell zu erläutern. Die im Prototyp enthaltenen Rätselkonzepte basieren bereits auf einer überarbeiteten Fassung, welche die Rückmeldung aus den bevorstehenden Probandentests vorwegnehmend berücksichtigt.

Für das Tutorial wurden spezifische Lernziele definiert, die auf die jeweiligen Rollen von Watcher und Player zugeschnitten sind.

Der Watcher soll im Verlauf des Tutorials mit den grundlegenden Steuerungsmechanismen der Anwendung vertraut gemacht werden. Dazu gehören Drag-. Zoom- und Yaw-Bewegungen sowie Touch-Interaktionen. Darüber hinaus soll der Watcher lernen, die aktuelle Position des Player-Avatars zu lokalisieren, das nächste Ziel zu identifizieren, Hinweise und Rätsel in der Spielwelt zu erkennen, sowie schwere Gegenstände zu platzieren, leichte Gegenstände an den Player zu senden (Preview-Funktion) und platzierte Objekte wieder zu entfernen. Ein weiteres Lernziel besteht darin, Unterschiede in der Spielwelt zu erkennen, um kontextbezogene Informationen interpretieren zu können.

Auch für den Player wurden zielgerichtete Lernziele formuliert. Dazu zählt die Bedienung der Steuerung (Drag, Zoom, Touch), die Lokalisierung und Beschreibung der eigenen Position, das Erkennen und Kommunizieren von Hinweisen und Rätseln sowie der Umgang mit interaktiven Weltobjekten. Zudem soll der Player lernen, leichte Gegenstände aufzunehmen und zu tragen (Preview-Funktion). Er soll neue Gegenstände finden, die zur Lösung von Rätseln relevant sind, sowohl schwere Gegenstände, die durch den Watcher platziert werden müssen.

Das Tutorial gliedert sich in drei Abschnitte, zu denen jeweils einzelne Räume in der Spielumgebung zugeordnet sind. In diesen Räumen werden die zuvor beschriebenen Lernziele gezielt aufgegriffen und durch entsprechende Aufgabestellungen eingeübt. Im Folgenden werden die einzelnen Tutorial-Abschnitte und ihre jeweiligen Lerninhalte detailliert vorgestellt.

% Folgende Lernziele wurden für das Tutorial konzipiert:
% \paragraph{Watcher}
% \begin{itemize}
%     \item Steuerung in der Anwendung (Drag, Zoom, Yaw und Touch)
%     \item Auffinden der richtigen Position, an der sich der Avatar des Players befindet 
%     \item Identifizierung des Ziels, wohin er den Player navigieren muss
%     \item Identifizierung von Hinweisen und Rätseln
%     \item Platzieren von schweren Gegenständen
%     \item Previewen von leichten Gegenständen
%     \item Entfernen von platzierten Gegenständen
%     \item Identifizierung von Unterschieden in der Spielwelt
% \end{itemize}

% \paragraph{Player}

% \begin{itemize}
%     \item Steuerung in der Anwendung (Drag, Zoom, Touch)
%     \item Identifizierung und Beschreibung der Lokalität
%     \item Identifizierung von Hinweisen und Rätseln
%     \item Tragen von Objekten (Previewen)
%     \item Interaktion mit interagierbaren Weltobjekten
%     \item Entdecken von neuen schweren und leichten Gegenständen
% \end{itemize}

% Das Tutorial wurde in verschiedene Abschnitte unterteilt, welche sich in einzelne Räume gliedern. Die einzelnen Abschnitte mit ihren Lernzielen und ihre Räumen werden nun vorgestellt.
\subsection{Abschnitt 1: Der Start}
Zunächst werden die Lernziele dieses Abschnittes dargelegt, welche die konzeptionelle Grundlage für dessen Ausgestaltung bilden.


\paragraph{Lernaspekte und Konzeption dieses Abschnittes}

Der Watcher erhält zunächst eine kurze Einführung in die grundlegenden Steuerungsmechaniken seiner Anwendung. Dadurch ist er in der Lage, sich eigenständig in der Spielwelt zu orientieren und zu navigieren. Seine zentrale Aufgabe besteht darin, die Bewegungen des Players nachzuverfolgen, wofür er dessen aktuelle Position kennen muss. Aus diesem Grund ist die Spielwelt so gestaltet, dass der Player seine Startposition zwingend beschreiben muss. Die Beschreibung ermöglicht es dem Watcher, den konkreten Standort zu identifizieren. Die unterschiedlichen Startpositionen im Raum unterstützen diesen Prozess. Abbildung \ref{fig:sketch-starterrooms} zeigt eine Skizze des Startgebietes.

\begin{figure}[ht]
\centering
\includegraphics[width=1\linewidth]{content/pictures/Startplaces_Sketch.png}
\caption{Sketchzeichnung der Starträume (Quelle: eigene Darstellung)}
\label{fig:sketch-starterrooms}
\end{figure}

Der Watcher muss darüber hinaus erkennen können, welches Ziel in der Spielwelt als nächstes angesteuert werden soll. Um dies zu ermöglichen, muss die Gestaltung der Spielumgebung so ausfallen, dass entweder ein einzelnes oder mehrere klar identifizierbare Ziele vorhanden sind, auf die sich die Spieler gemeinsam zubewegen können. Für den ersten Abschnitt des Tutorials bietet es sich an, mit einem eindeutigen Ziel zu arbeiten, zu dem alle möglichen Wege führen. Dieses Ziel ist in der Skizze von Abbildung \ref{fig:sketch-starterrooms} deutlich erkennbar.

Zudem sollen die grundlegenden Mechaniken und Funktionen, die mit der Spielerrolle des Watchers einhergehen, erlernt werden. Dazu zählen insbesondere das Platzieren, Previewen und Entfernen von Gegenständen. Das Platzieren kann eingeleitet werden, indem der Watcher zu Beginn des Spiels bereits einen Gegenstand, bspw. eine Säule, im Inventar besitzt, für den lediglich ein geeigneter Platz in der Spielwelt gefunden werden muss. Ein weiteres Lernziel wird durch ein Hindernis umgesetzt, das dem Player nur noch ein passender Gegenstand, etwa eine Fackel, fehlt, um dieses zu überwinden. Diese Konstellation fordert den Watcher dazu auf, dem Player den benötigten Gegenstand aus dem Inventar bereitzustellen.

Das entfernen bereits platzierter Objekte kann wieder dadurch eingeführt werden, dass sich der Player im Zielbereich, die Eingangshalle (der Bereich, der an das Ziel angrenzt) aus Abbildung \ref{fig:sketch-starterrooms}, befindet und dort bspw. eine Tür zu einem neuen Abschnitt öffnen muss. Hierfür werden erneut jene Gegenstände benötigt, die bereits in den zwei vorherigen Rätseln verwendet wurden. Daraus ergibt sich die Notwendigkeit, zuvor platzierte Objekte aus der Spielwelt zu entfernen und diese an neuen Stellen erneut einzusetzen.


Darüber hinaus erlernt der Watcher die Spielwelt zu interpretieren. Dabei existieren gezielt eingebaute Unterschiede in der Darstellung der Spielwelt zwischen der Anwendung des Players und der des Watchers. Diese Inkonsistenzen sollen gezielt die Kommunikation zwischen beiden Rollen anregen und fördern. Dieser Aspekt wurde sich in der vorangegangenen Ausarbeitung im Feedback gewünscht. Konkret wird dies durch das Fehlen bestimmter Türen in den Startbereichen des Players oder an den Durchgängen zum Eingangsportal des nächstes Abschnitts realisiert.

Analog zum Watcher erhält auch der Player eine kurze Einführung in die grundlegenden Steuerungsmechaniken seiner Anwendung. Während sich die meisten Rätsel und Hindernisse innerhalb der Anwendungen des Players befinden, sind einige Interaktionen auch auf Seiten des Watchers verortet. Der Player muss daher in der Lage sein, relevante Rätsel, Hindernisse und Hinweise zu identifizieren und korrekt anzuwenden. Um den Einstieg in die Spielmechanik zu erleichtern, wird in den Starträumen des Players eine kleine Notiz platziert, die einen ersten Hinweis auf die Funktionsweise der Watcher-Mechaniken sowie auf das Lösen des ersten gemeinsamen Hindernisses gibt.

\begin{figure}[ht]
\centering
\includegraphics[width=1\linewidth]{content/pictures/Startroom_Sketch.png}
\caption{Sketchzeichnung des Hinweises für das erste Hindernis (Quelle: eigene Darstellung)}
\label{fig:sketch-startriddle}
\end{figure}

Abbildung \ref{fig:sketch-startriddle} zeigt eine erste konzeptionelle Überlegung, bei der eine im Startraum platzierte Notiz den Player darauf hinweist, dass ein schwerer Gegenstand auf eine Druckplatte gestellt werden muss, um die Tür zum angrenzenden Flur zu öffnen. Dieses Rätsel dient als Einführung in die Spielmechanik des indirekten Türöffnen und betont zugleich die Notwendigkeit der Kooperation zwischen Player und Watcher.

Darüber hinaus soll dieser erste Abschnitt das Tragen und Platzieren von Gegenständen vermitteln. Zu diesem Zweck kann ein weiteres Hindernis eingebaut werden, das bspw. durch das Einsetzen eines Objektes, wie einer Fackel in eine entsprechende Halterung, überwunden werden muss. In diesem Fall ist der Player auf die Unterstützung des Watchers angewiesen, der den benötigten Gegenstand auswählt und übermittelt. Der Player wiederum muss den Gegenstand korrekt einsetzen, um das Hindernis zu lösen.

Als letztes Lernziel in diesem Abschnitt wird das Freischalten neuer Bereiche eingeführt. Dieses Ziel betrifft sowohl den Player als auch den Watcher, da beide gemeinsam Bedingungen erfüllen müssen, um Zugang zu weiteren Abschnitten der Spielwelt zu erhalten.

\paragraph{Beschreibung des Abschnittes}

\begin{figure}[ht]
\centering
\includegraphics[width=1\linewidth]{content/pictures/Abschnitt_Concept_00.png}
\caption{Konzept Abschnitt 1 (Quelle: eigene Darstellung)}
\label{fig:section_00_concept}
\end{figure}

Abbildung \ref{fig:section_00_concept} zeigt eine erste Konzeptzeichnung des Einstiegsabschnitts. Auf der linken Seite ist ein sechseckiger Raum dargestellt, in dem der Spieleravatar des Players zu Beginn der Spielsequenz platziert wird. Dieser Startbereich existiert, wie auf der rechten Seite der Abbildung zu erkennen ist, in drei Varianten. Der Player muss seinem Watcher mitteilen, in welchem der drei Räume er sich befindet. Dies erfordert eine gezielte Beschreibung der Umgebung und stellt somit eine erste kommunikative Aufgabe dar.

Die Unterscheidung der Räume erfolgt über visuelle Merkmale in deren Gestaltung. Wie in Abbildung \ref{fig:corridors} dargestellt, besitzt der erste Raum (erste Reihe, linkes Bild) einen zentral angebrachten Kronleuchter, der zweite Raum (zweite Reihe, linkes Bild) ist durch einen großflächigen Teppich gekennzeichnet, während sich im dritten Raum (dritte Reihe, linkes Bild) eine Bank zwischen zwei Innensäulen befindet. Diese unterschiedlichen Merkmale dienen als Referenzpunkte für die verbale Orientierung und fördern die Koordination zwischen beiden Rollen zu Spielbeginn.

\begin{figure}[ht]
\centering
\includegraphics[width=1\linewidth]{content/pictures/Room_00-Room_02-Corridor_00-Corridor_02.png}
\caption{Korridor 1 bis Korridor 3 (Quelle: eigene Darstellung)}
\label{fig:corridors}
\end{figure}

Wie auf der rechten Darstellung der Konzeptzeichnung in Abbildung \ref{fig:section_00_concept} ersichtlich, führen die drei Starträume jeweils über einen Flur in eine zentrale Eingangshalle. Diese architektonische Verbindung ist in Abbildung \ref{fig:section_00} visualisiert. An der gegenüberliegenden Wand der Flure befindet sich eine verschlossene Tür, die vom Player, in Zusammenarbeit mit dem Watcher, geöffnet werden muss, um den Zugang zum nächsten Spielabschnitt zu ermöglichen.

\begin{figure}[ht]
\centering
\includegraphics[width=1\linewidth]{content/pictures/Abschnitt_00.PNG}
\caption{Abschnitt 1 (Quelle: eigene Darstellung)}
\label{fig:section_00}
\end{figure}

Zu Beginn erhält der Watcher eine Übersicht über den gesamten ersten Spielabschnitt, wie in Abbildung \ref{fig:section_00} dargestellt. Im Gegensatz dazu ist das Sichtfeld des Players zu Beginn auf einen der drei Starträume (vgl. Abbildung \ref{fig:corridors}) sowie den angrenzenden Flur beschränkt. Die beiden übrigen Starträume sowie deren zugehörige Flure bleiben für den Player dauerhaft unzugänglich, auch nachdem er die Eingangshalle zum nächsten Abschnitt betreten hat.

\subsection{Abschnitt 2: Der Sicherheitsraum}

Analog zum ersten Abschnitt werden zunächst die zugrunde liegenden Lernziele sowie die konzeptionellen Überlegungen erläutert, bevor im Anschluss der Aufbau des Abschnitts im Detail vorgestellt wird.

\paragraph{Lernaspekte und Konzeption dieses Abschnittes}

Im zweiten Abschnitt lernt der Watcher, das ihm neue Räume angezeigt werden können (die dem Player verborgen bleiben), sobald diese vom Player freigeschaltet wurden. Der Player kann bspw. einen Mechanismus aktivieren, durch den ein bislang verborgener Bereich sichtbar wird, in welchem der Watcher anschließend ein Rätsel lösen muss.

Darüber hinaus erkennt der Watcher neu entdeckte Objekte, die der Player in der Spielwelt gefunden hat, als platzierte Gegenstände. Diese Objekte kann der Watcher bei Bedarf wieder entfernen. Solche Gegenstände können bspw. zur Lösung weiterer Aufgaben genutzt werden.

In diesem Zusammenhang erlernt der Player das gezielte Entdecken von Gegenständen, etwas durch das Vorbeigehen in unmittelbarer Nähe. Technisch kann dies so umgesetzt werden, dass beim Annähern ein Tooltip eingeblendet wird und der Gegenstand dadurch auch für den Watcher sichtbar wird.

\paragraph{Beschreibung des Abschnittes}

\begin{figure}[ht]
\centering
\includegraphics[width=1\linewidth]{content/pictures/Abschnitt_Concept_01.PNG}
\caption{Konzept Abschnitt 2 (Quelle: eigene Darstellung)}
\label{fig:section_01_concept}
\end{figure}

Abbildung \ref{fig:section_01_concept} zeigt eine erste Konzeptzeichnung des zweiten Abschnitts, der als Sicherheitsraum konzipiert ist. Dieser Abschnitt fungiert als verbindendes Element zwischen dem zuvor beschriebenen Startraum und dem nachfolgendem Bereich in Abschnitt drei. Der Player betritt den Sicherheitsraum über den oberen Zugang, der von der Eingangshalle aus erreichbar ist.

Der Raum verfügt auf der rechten Seite über eine Tür, die zu einem Innenhof bzw. einem Außenbereich führt. Über den unteren Ausgang kann ein weiterer Abschnitt betreten werden. Zentral für die Funktion dieses Raumes als Sicherheitsbereich ist ein Überwachungsterminal, das sich in der unteren rechten Ecke befindet.

Um die im Sicherheitsraum tätigen Personen vor Einblicken oder Störungen durch andere Mitarbeiter oder Spielcharaktere abzuschirmen, wurde zwischen dem Terminal und der rechten Tür eine Trennwand vorgesehen.

\begin{figure}[ht]
\centering
\includegraphics[width=1\linewidth]{content/pictures/Abschnitt_01 - Player.png}
\caption{Abschnitt 2 aus Sicht des Players (Quelle: eigene Darstellung)}
\label{fig:section_01_player}
\end{figure}

\begin{figure}[ht]
\centering
\includegraphics[width=1\linewidth]{content/pictures/Abschnitt_01 - Watcher.png}
\caption{Abschnitt 2 aus Sicht des Watchers (Quelle: eigene Darstellung)}
\label{fig:section_01_watcher}
\end{figure}

Die Abbildungen \ref{fig:section_01_player} und \ref{fig:section_01_watcher} zeigen den Sicherheitsraum aus den jeweiligen Perspektiven von Player und Watcher. Die gezielten Unterschiede in der visuellen Gestaltung der beiden Szenen zielen darauf ab, eine intensivere verbale Abstimmung zwischen den Spielerrollen zu fördern. Beide Versionen des Raumes enthalten einen gelben Sicherungskasten, zu dem gelbe Leitungen führen und von dem sie auch wieder wegführen.

In der Player-Ansicht (vgl. Abbildung \ref{fig:section_01_player}) befindet sich der Sicherungskasten im rechten Bild links neben dem PC an der Rückwand zum Außenbereich. In der Watcher-Ansicht (vgl. Abbildung \ref{fig:section_01_watcher}) ist er hingegen rechts neben der Tür zum Außenbereich positioniert. Auf der Rückseite des Sicherungskasten befindet sich ein Stromgenerator, der nur in der Watcher-Ansicht sichtbar ist, dort ist er links neben der Tür zum Außenbereich zu erkennen. In der Anwendung des Players fehlt dieser Stromgenerator vollständig.

Dieser asymmetrische Informationszugang ist zentral für das im Raum enthaltene Rätsel. Nur der Watcher kann den Außenbereich mit dem Generator sehen und muss mithilfe der vom Player bereitgestellten Informationen eigenständig zur Lösung gelangen. Die Raumgestaltung unterstützt somit gezielt eine wechselseitige Abhängigkeit beider Rollen in der Kommunikation und Interaktion.

\subsection{Abschnitt 3: Das Büro}

Der dritte Abschnitt fungiert im Rahmen des Tutorials nicht mehr als expliziter Lernbereich, sondern dient vielmehr als Anwendungsumgebung für die zuvor erlernten Mechaniken. Als szenisches Setting wurde ein Bürotrakt innerhalb eines größeren Bürokomplexes gewählt, welcher zugleich Teil der Spielumgebung in der Haupthandlung ist. Der zuvor durchlaufene Abschnitt hatte die Funktion eines Kontrollraums und stellt die Verbindung zwischen dem Anfangsbereich und dem Bürokomplex her.

\begin{figure}[ht]
\centering
\includegraphics[width=0.6\linewidth]{content/pictures/Abschnitt_02_Concept.png}
\caption{Konzept Abschnitt 3 (Quelle: eigene Darstellung)}
\label{fig:section_02_concept}
\end{figure}

Abbildung \ref{fig:section_02_concept} zeigt erste konzeptionelle Überlegungen zur räumlichen Gestaltung innerhalb des Bürokomplexes, die in Abbildung \ref{fig:section_02} weiter ausgearbeitet und ergänzt wurden.

\begin{figure}[ht]
\centering
\includegraphics[width=1\linewidth]{content/pictures/Abschnitt_02.png}
\caption{Abschnitt 3 (Quelle: eigene Darstellung)}
\label{fig:section_02}
\end{figure}

Der Bürokomplex im letzten Abschnitt des Tutorials setzt sich aus mehreren Räumlichkeiten zusammen, in Abbildung \ref{fig:section_02} zu sehen. Einem Korridor (oben links in der Abbildung), einer Küche (unten links), einem Tagungsraum (oben mittig), einem kleinen WC (unten mittig), sowie zwei Büros (oben und unten rechts). Auch in diesem Bereich bestehen Unterschiede in der Wahrnehmung zwischen der Anwendung des Players und der des Watchers. Beide sehen jeweils nur eines der beiden Büros, die zueinander gespiegelt angelegt sind. Jedes dieser Büros beinhaltet ein separates Rätsel, sowie Hinweise darauf, wie nach dessen Lösung fortgefahren werden kann.

In der Player-Ansicht befinden sich im Tagungsraum zudem mehrere Stühle, die zunächst entdeckt werden müssen, bevor sie dem Watcher als Objekte zur Verfügung stehen. Die funktionale Einbindung dieser Räume und ihr Beitrag zum Gesamtpuzzle werden im folgenden Kapitel \emph{\nameref{sec:riddles}} näher erläutert.

\subsection{Rätseldesign}\label{sec:riddles}

Der Aufbau der Rätsel in den Abschnitten 1 und 2 folgt einer linearen Struktur, da sie primär der Einführung in die grundlegenden Spielmechaniken dienen. Die zugehörigen Hinweise sind in der Regel räumlich nah am jeweiligen Rätsel platziert. Entweder direkt in die Umgebung oder in Form von Notizen integriert.

\paragraph{Abschnitt 1}

\begin{figure}[ht]
\centering
\includegraphics[width=1\linewidth]{content/pictures/Rätseldesign - Abschnitt00 - Rätsel00.png}
\caption{Aufbau der Rätsel von Abschnitt 1, Teil 1 (Quelle: eigene Darstellung)}
\label{fig:riddle-design-section00-00}
\end{figure}

\begin{figure}[ht]
\centering
\includegraphics[width=1\linewidth]{content/pictures/Rätseldesign - Abschnitt00 - Rätsel01.png}
\caption{Aufbau der Rätsel von Abschnitt 1, Teil 2 (Quelle: eigene Darstellung)}
\label{fig:riddle-design-section00-01}
\end{figure}

\begin{figure}[ht]
\centering
\includegraphics[width=1\linewidth]{content/pictures/Rätseldesign - Abschnitt00 - Rätsel02.png}
\caption{Aufbau der Rätsel von Abschnitt 1, Teil 3 (Quelle: eigene Darstellung)}
\label{fig:riddle-design-section00-02}
\end{figure}

Zu Beginn befindet sich der Player innerhalb eines Verlieses. Eine verrostete Eisentür blockiert den Weg aus diesem Raum (vgl. Abbildung \ref{fig:riddle-design-section00-00}, linkes Bild). Auf einer Notiz (vgl. Abbildung \ref{fig:riddle-design-section00-00}, mittleres Bild) erhält der Player den Hinweis, einen Gegenstand auf eine Druckplatte zu werfen (vgl. Abbildung \ref{fig:riddle-design-section00-00}, rechtes Bild). Gemeint ist damit das Platzieren eines schweren Gegenstandes, die verfügbare Säule die sich zu Spielbeginn im Inventar des Watchers befindet.

Nach dem Verlassen des Verlieses trifft der Player auf eine weitere verrostete Eisentür, die ebenfalls geöffnet werden muss (vgl. Abbildung \ref{fig:riddle-design-section00-01}, linkes Bild). An der linken Wandseite der Tür befindet sich eine Fackelhalterung mit einer eingesetzt Fackel, die sowohl vom Player als auch vom Watcher wahrgenommen wird (vgl. Abbildung \ref{fig:riddle-design-section00-01}, mittlere Bild). Auf der gegenüberliegenden Wandseite erkennen beide eine leere Fackelhalterung (vgl. Abbildung \ref{fig:riddle-design-section00-01}, rechts Bild). Aus dieser Konstellation ergibt sich die Schlussfolgerung, dass der Watcher dem Player eine Fackel senden muss, damit dieser sie in die leere Haltung einsetzt.

In der darauffolgenden Eingangshalle treffen beide Spieler auf eine verschlossene Sicherheitstür, die zum Abschluss des Abschnitts geöffnet werden muss (vgl. Abbildung \ref{fig:riddle-design-section00-02}, linkes Bild). Links von der Tür befinden sich eine Säule, sowie eine Fackelhalterung mit eingesetzter Fackel (vgl. Abbildung \ref{fig:riddle-design-section00-02}). Auf der rechten Seite ist eine weitere, leere Fackelhalterung zu sehen (vgl. Abbildung \ref{fig:riddle-design-section00-02}, rechts Bild). Die zuvor benutzte Fackel muss nun durch den Player in diese leere Halterung eingesetzt werden.

Darüber hinaus muss der Watcher die im ersten Rätsel verwendeten Säule gemäß der beiliegenden Beschreibung (\say{The column is required to match a pattern or to serve as a counterweight}) auf der gegenüberliegenden Seite der Tür so positionieren, dass beide Säulen symmetrisch zur Tür ausgerichtet sind (vgl. Abbildung \ref{fig:riddle-design-section00-02}, linkes Bild).

\paragraph{Abschnitt 2}

Abschnitt zwei setzt das lineare Rätseldesign aus dem ersten Abschnitt fort, wodurch die Spieler weiterhin schrittweise an die zugrunde liegenden Mechaniken herangeführt werden.

\begin{figure}[ht]
\centering
\includegraphics[width=1\linewidth]{content/pictures/Rätseldesign - Abschnitt01 - Rätsel00.png}
\caption{Aufbau der Rätsel von Abschnitt 2, Teil 1 (Quelle: eigene Darstellung)}
\label{fig:riddle-design-section01-00}
\end{figure}

\begin{figure}[ht]
\centering
\includegraphics[width=1\linewidth]{content/pictures/Rätseldesign - Abschnitt01 - Rätsel01.png}
\caption{Aufbau der Rätsel von Abschnitt 2, Teil 2 (Quelle: eigene Darstellung)}
\label{fig:riddle-design-section01-01}
\end{figure}

Nachdem der Player den Sicherheitsraum betreten hat, trifft er auf zwei Türen. Eine befindet sich links vom Eingang, die andere liegt direkt gegenüber. Letztere markiert den Ausgang aus diesem Abschnitt, ist jedoch zu Beginn nicht passierbar, da der zugehörige Bewegungsmelder deaktiviert ist und zunächst aktiviert werden muss. Das dafür vorgesehen Terminal befindet sich links neben dem Ausgang (vgl. Abbildung \ref{fig:riddle-design-section01-00}), ist allerdings ohne Strom.

Sowohl der Player als auch der Watcher erhalten Hinweise darauf, wie die Stromversorgung für das Terminal wiederhergestellt werden kann. Innerhalb des Sicherheitsraums befindet sich ein Stromgenerator (vgl. Abbildung \ref{fig:riddle-design-section01-01}, Bild zweite Zeile Mitte), der an einer bestimmten Stelle positioniert werden muss. In der Ansicht des Watchers ist rechts neben der zweiten Tür im Sicherheitsraum, also der Tür, die sich vom Eingang aus links befindet, eine Sicherung zu sehen. Auf deren gegenüberliegender Seite befindet sich ein kleiner Innenhof, in dem ein weiterer Stromgenerator neben der Tür steht (vgl. Abbildung \ref{fig:riddle-design-section01-01}, Bilder erste und zweite Zeile rechts).

Der Player wiederum findet in seiner Version des Raumes einen Plan auf einer Pinnwand (vgl. Abbildung \ref{fig:riddle-design-section01-01}, Bild erste Reihe Mitte), der die Stromversorgung der Watcher-Anwendung darstellt. Ein zentraler Unterschied zwischen beiden Versionen besteht darin, dass sich die Sicherung in der Sicht des Players nicht neben der Tür, sondern links neben dem Terminal befindet (vgl. Abbildung \ref{fig:riddle-design-section01-01}, Bild zweite reihe links). Daraus ergibt sich, dass der Generator, ausgehend von der Ansicht des Watchers, auf der gegenüberliegenden Seite, also links neben dem bestehenden Stromkasten (vgl. Abbildung \ref{fig:riddle-design-section01-01}, Bild zweite Reihe rechts), platziert werden muss.

Sobald der Generator korrekt positioniert ist, lasst sich der Bewegungsmelder am Terminal aktivieren. Dadurch öffnet sich die Tür und die Spieler können Abschnitt drei betreten.

\paragraph{Abschnitt 3}

Abschnitt drei stellt den komplexesten Teil des Tutorials dar. Sowohl der Player als auch der Watcher erhalten in unterschiedlichen, nach und nach freigeschalteten Räumen Hinweise oder Gegenstände, die sie für das Lösen späterer Rätsel benötigen.

Das im Anhang  \ref{sec:append_riddles_part_3}: \nameref{sec:append_riddles_part_3} dargestellte Rätseldesign visualisiert den Aufbau der Rätsel-Struktur in Abschnitt drei. Ergänzt wird dieses Diagramm durch die Abbildungen \ref{fig:riddle-design-section02-00} bis \ref{fig:riddle-design-section02-04}, welche zentrale Elemente des Designs im Detail veranschaulichen.

\begin{figure}[ht]
\centering
\includegraphics[width=1\linewidth]{content/pictures/Rätseldesign - Abschnitt02 - Rätsel00.png}
\caption{Aufbau der Rätsel von Abschnitt 3, Teil 1 (Quelle: eigene Darstellung)}
\label{fig:riddle-design-section02-00}
\end{figure}

\begin{figure}[ht]
\centering
\includegraphics[width=1\linewidth]{content/pictures/Rätseldesign - Abschnitt02 - Rätsel01.png}
\caption{Aufbau der Rätsel von Abschnitt 3, Teil 2 (Quelle: eigene Darstellung)}
\label{fig:riddle-design-section02-0l}
\end{figure}

\begin{figure}[ht]
\centering
\includegraphics[width=1\linewidth]{content/pictures/Rätseldesign - Abschnitt02 - Rätsel02.png}
\caption{Aufbau der Rätsel von Abschnitt 3, Teil 3 (Quelle: eigene Darstellung)}
\label{fig:riddle-design-section02-02}
\end{figure}

\begin{figure}[ht]
\centering
\includegraphics[width=1\linewidth]{content/pictures/Rätseldesign - Abschnitt02 - Rätsel03.png}
\caption{Aufbau der Rätsel von Abschnitt 3, Teil 4 (Quelle: eigene Darstellung)}
\label{fig:riddle-design-section02-03}
\end{figure}

\begin{figure}[ht]
\centering
\includegraphics[width=1\linewidth]{content/pictures/Rätseldesign - Abschnitt02 - Rätsel04.png}
\caption{Aufbau der Rätsel von Abschnitt 3, Teil 5 (Quelle: eigene Darstellung)}
\label{fig:riddle-design-section02-04}
\end{figure}

Nachdem der Player den neuen Korridor betreten hat, trifft er auf mehrere verschlossene Türen (vgl. Abbildung \ref{fig:riddle-design-section02-00}, Bild erste Reihe links). Zu Beginn muss der Player den Druckvorgang auslösen, der in der Anwendung des Watchers über einen Laptop im Flur sicht- und ausführbar ist (vgl. Abbildung \ref{fig:riddle-design-section02-00}, Bild zweite Reihe links). Sobald der Druck gestartet wurde, erhält der Player eine Notiz mit einem Zahlencode, der zum Öffnen der Küchentür in der Player-Anwendung benötigt wird (vgl. Abbildung \ref{fig:riddle-design-section02-00}, Bild zweite Reihe Mitte).

Da der Watcher bereits zu diesem Zeitpunkt Zugang zur Küche hat, besteht seine Aufgabe darin, den Player zu dieser zu navigieren. In der Küche findet der Player einen Projektor sowie ein Buch, die in späteren Rätseln von Bedeutung sind (vgl. Abbildung \ref{fig:riddle-design-section02-00}, Bild zweite Reihe rechts). 

Nachdem der Projektor vom Player entdeckt wurde und dieser auch in der Anwendung des Watchers erscheint, wird das Konferenzzimmer zugänglich (vgl. Abbildung \ref{fig:riddle-design-section02-0l}, Bild erste Reihe Mitte und Bild erste Reihe Rechts). Sowohl Player als auch Watcher sehen in diesem Raum mehrere Stühle. Die Anzahl der sichtbaren Stühle unterscheidet sich dabei je nach Anwendung. In der Sicht des Watchers erscheinen zusätzliche Stühle erst dann, wenn der Player sie zuvor entdeckt hat  (vgl. Abbildung \ref{fig:riddle-design-section02-0l}, Bild erste Reihe rechts und zweite Reihe links).

Wird der Projektor vom Watcher auf den Konferenztisch platziert, erscheint eine Nachricht auf der Leinwand. Diese trägt den Text \say{The chamber has been opened}, eine Anspielung auf den Satz aus Harry Potter und die Kammer des Schreckens, und verweist auf das benachbarte Badezimmer, das als nächstes Ziel dient (vgl. Abbildung \ref{fig:riddle-design-section02-0l}, Bild zweite Reihe Mitte und zweite Reihe rechts).

Im Badezimmer erhält der Player einen Hinweis darauf, wie der Watcher die Stühle im Konferenzzimmer anordnen muss. An den Türen der Toilettenkabinen befinden sich drei verschiedene Symbole, die jeweils für eine bestimmte Anordnung der Stühle stehen. Diese Symbole wurden aus dem Spiel We were here too entnommen (vgl. \citealp{total_mayhem_games_we_2018}). Die entsprechenden Anordnungen sind auf Postern dargestellt, die an der Wand des Badezimmers hängen (vgl. Abbildung \ref{fig:riddle-design-section02-02}, Bild erste Reihe rechts und zweite Reihe links).

Welches der drei Symbole für das aktuelle Rätsel relevant ist, kann der Watcher in der Küche erkennen. Es befindet sich auf der Rückwand der Küche zum Flur hin, unmittelbar auf der rechten Seite nach dem Betreten des Raumes (vgl. Abbildung \ref{fig:riddle-design-section02-02}, Bild erste Reihe Mitte). Wenn der Watcher daraufhin die Stühle im Konferenzzimmer entsprechend der richtigen Anordnung platziert, erscheint eine neue Nachricht auf der Leinwand mit dem Text \say{The chairs haven fallen}. Dieser Spruch orientiert sich an dem Lateinischen Ausdruck \say{Alea iacta est}. Diese informiert den Player darüber, dass die Stühle korrekt positioniert wurden und sich nun ein weiterer Raum geöffnet hat (vgl. Abbildung \ref{fig:riddle-design-section02-02}, Bild zweite Reihe rechts).

Am linken Ende des Korridors wird nun ein kleines Büro zugänglich, zu dem sowohl Player als auch Watcher Zutritt erhalten (vgl. Abbildung \ref{fig:riddle-design-section02-03}, Bild erste Reihe links). Dieses Büro existiert in zwei verschiedenen Versionen. Der Player betritt einen Raum auf der einen Seite des Flurs, während sich für den Watcher die Tür auf der gegenüberliegenden Seite öffnet. Obwohl die beiden Büroräume nahezu identisch aufgebaut sind, fehlen dem Player bestimmte Gegenstände, die ergänzt werden müssen.

Beide Büroräume bestehen aus zwei kleinen Unterräumen. Der erste Raum fungiert als kleine Bibliothek, in der sich ein Bücherregal befindet (vgl. Abbildung \ref{fig:riddle-design-section02-03}, Bild erste Reihe Mitte). In der Anwendung des Watchers sind bereits alle Bücher vorhanden, auch jene, die in der Umgebung des Players fehlen (vgl. Abbildungen \ref{fig:riddle-design-section02-03}, Bild zweite Reihe rechts und erste Reihe rechts). Die fehlenden Bücher können in der Küche (vgl. Abbildung \ref{fig:riddle-design-section02-03}, Bild zweite Reihe links), sowie bei den Waschbecken im Badezimmer gefunden werden (vgl. Abbildung \ref{fig:riddle-design-section02-03}, Bild zweite Reihe Mitte).

Werden die Bücher schließlich in der richtigen Reihenfolge in das Regal eingesetzt, das graue Buch links,  das hellblau Buch rechts, öffnet sich die weiße Tür auf der linken Seite, die zum hinteren Bereich des Büros führt (vgl. Abbildung \ref{fig:riddle-design-section02-03}, Bild erste Reihe Mitte).

Der hintere Bereich des Büros bildet den Arbeitsplatz innerhalb des Raums. Dort befinden sich ein Schreibtisch mit einem Computer und mehrere Monitoren. An den seitlichen Wänden sind Bücherregale angebracht, die jedoch für die Lösung des letzten Rätsels keine Relevanz besitzen. Betritt der Player den Raum, kann er auf einem Monitor die Notiz \say{Exit marks the exit. Time shows the time. K.} lesen (vgl. Abbildung \ref{fig:riddle-design-section02-04}, Bild links). Die Notiz enthält einen Hinweis auf das Passwort für den Ausgang und verweist zugleich auf dessen Position innerhalb der Spielwelt.

Zeitgleich wird der hintere Bereich des Büros auch für den Watcher sichtbar. In dieser Version des Raumes ist die Notiz nicht zu sehen; stattdessen befindet sich dort eine Uhr, deren angezeigte Uhrzeit das Passwort für den Ausgang darstellt (vgl. Abbildung \ref{fig:riddle-design-section02-04}, Bild Mitte). Der Ausgang selbst ist durch ein \say{Exit}-Schild gekennzeichnet und befindet sich am rechten Ende des Korridors, unmittelbar neben der Tür zum Konferenzzimmer (vgl. Abbildung \ref{fig:riddle-design-section02-04}, Bild rechts und Abbildung \ref{fig:riddle-design-section02-0l}, Bild erste Reihe Mitte).

Gibt der Player das korrekte Passwort ein, ist das Tutorial abgeschlossen, Infolgedessen können sowohl Player als auch Watcher den entsprechenden Abschnitt des Bürokomplexes verlassen.

\section{Dialoge}

Das grundlegende Spielkonzept sieht einen Dialog vor, der den Spielern Hintergrundinformationen vermittelt und den aktuellen Spielfortschritt erläutert. Im Unterschied zu vielen anderen Spielen ist der Dialog jedoch nicht identisch für beide Spielanwendungen gestaltet. Stattdessen ist vorgesehen, dass die Dialogtexte so aufgeteilt werden, dass sich die Spieler diese gegenseitig vorlesen müssen. Ziel dieses Ansatzes ist es, die wechselseitige Kommunikation zu fördern und die Zusammenarbeit zwischen den Spielteilnehmern gezielt zu stärken.

\section{Sounddesign}

Das Sounddesign soll den visuellen Eindruck der jeweiligen Anwendung durch gezieltes Feedback unterstützen. Sowohl die Bewegung des Playeravatars als auch die Interaktionen des Watchers erzeugen dabei entsprechende akustische Rückmeldungen. Zusätzlich erhalten sowohl Player als auch Watcher über spezifische Klänge ein Feedback, wenn Rätsel erfolgreich gelöst oder neue Wege freigeschaltet wurden.

Die konkrete Ausgestaltung des Sounddesigns wird in den folgenden Unterkategorien näher beschrieben. Grundsätzlich ist das Sounddesign jedoch dezent im Hintergrund gehalten, um die verbale Kommunikation zwischen den Spielern nicht zu beeinträchtigen.

\subsection{Hintergrundmusik}

Jedes Szenario verfügt über eine eigene atmosphärische Hintergrundmusik, die zur akustischen Untermalung der Spielwelt beiträgt. Diese wirkt sich auch auf die klangliche Gestaltung der Geräusche aus, die durch den Avatars des Players sowie durch die Aktionen des Watchers erzeugt werden. So sind beispielsweise Hall-Effekte im leeren Bürokomplex oder ein gedämpfter Klang im Verlies zu hören, die zur jeweiligen räumlichen Atmosphäre beitragen.

Auch das Haupt- und Pausemenü sind mit einer atmosphärischen Klangkulisse unterlegt. Diese dient der Beruhigung und Orientierung, steht jedoch zugleich in thematischem Bezug zum Setting des Spiels.

\subsection{Umgebungsgeräusche}

Jedes dynamische oder licht-emittierende Weltobjekt erzeugt bei Interaktion, Platzierung oder Entfernung ein entsprechendes Geräusch, dessen akustische Eigenschaft, insbesondere der Hall, an die jeweilige Umgehung angepasst sind. Schwere Gegenstände, wie z.B. eine Säule, verursachen beim Abstellen oder Entfernen ein charakteristisches Kratzgeräusch auf dem Boden, das das physikalische Gewicht und die Materialität des Gegenstands auditiv vermittelt.


\subsection{Interaktionsgeräusche}

Jede Interaktion des Players kann mit einem spezifischen Geräusch verknüpft sein. So führt das Tragen einer Fackel in der Kleidung zu einem dezenten Rascheln, wehrend das Einsetzen der Fackel in eine Halterung ein charakteristisches Klacken von Holz auf Metall erzeugt. Auch der Watcher erhält für seine Interaktion innerhalb des Menüs ein auditives Feedback. Sowohl die Auswahl als auch das Verschieben von Gegenständen im Menü werden akustisch begleitet. Bei der Positionsbestimmung eines Gegenstandes ist ein Kratzen oder Schleifen zu hören, das die Bewegung des Gegenstands auditiv nachvollziehbar macht.


\section{Weitere nicht berücksichtigte Überlegungen}

In diesem Kapitel werden Konzeptideen vorgestellt, die im Verlauf der Entwicklung verworfen und nicht in das finale Spielkonzept integriert wurden.

Zu Beginn war vorgesehen, dass in einer Spielsitzung ein Player mit mehreren Watchern gemeinsam interagieren kann. Während der Abschlusspräsentation des vorangegangenen Projekts wurde jedoch hinterfragt, ob sich mehrere Watcher in ihrer Interaktion nicht gegenseitig behindern würden. Als Reaktion auf diese Problematik wurde zunächst erwogen, durch gezielte Events einzelne Watcher temporäre zu beschäftigen, bspw. durch Minispiele, wie sie aus Mobile-Spielen wie \say{Duskwood} oder \say{Sentence} bekannt sind (vgl. \citealp{everbyte_duskwood_2019,jaunt_sentence_2019}). Diese Lösung wurde jedoch verworfen, da sie den Fokus vom eigentlichen Spielkonzept und der Kommunikation der Spieler abgelenkt hätte.

Aus der ursprünglichen Idee ergab sich ein weiteres Konzept, das eine Teamstruktur innerhalb der Spielwelt vorsah. Dabei sollten zwei Gruppen agieren. Ein Team hätte Rätsel in die Spielwelt integriert, während das andere Team sie zu lösen gehabt hätte. Dieses Modell hätte jedoch die Entwicklung einer konsistenten narrativen Struktur erschwert, weshalb auch diese Variante nicht weiterverfolgt wurde.

Im Kontext der angestrebten Gleichwertigkeit zwischen den beiden Spielerrollen wurden weitere Ideen zur Verbesserung der Aufgabenverteilung entwickelt. Eine dieser Überlegungen betraf die Einführung eines Crafting-Systems. Der Player hätte in der Spielwelt verarbeitbare Gegenstände gefunden, die der Watcher in seinem Menü zu leichten oder schweren Gegenständen hätte kombinieren können. Dieser würden zum Lösen und beseitigen der Hindernisse gebraucht werden. Aufgrund des hohen gestalterischen und zeitlichen Aufwands zur Implementierung sowie der begrenzten Relevanz für das zentrale Spielziel wurde diese Idee ebenfalls nicht umgesetzt.

Eine weitere verworfene Idee bezog sich auf ein Notizsystem, das sich an der Kernmechanik des Spiels \say{Shadow of Doubt} orientiert (vgl. \citealp{colepowered_games_shadows_2023}). Der Watcher hätte dabei Notizen, etwa in Form von Screenshots oder Texten, auf einer virtuellen Notiztafel anheften können, basierend auf Beobachtungen oder Informationen, die vom Player gesammelt wurden. Dieses System sollte die Strukturierung der Rätsellogik unterstützen und die Planung der nächsten Spielschritte erleichtern. Auch dieses Feature wurde nicht realisiert, da dem Watcher im späteren Verlauf ohnehin erweiterte Funktionen zur Verfügung stehen (z. B. das Drehen oder Vergrößern von Gegenständen) und das Lösen der Rätsel meist unmittelbar mit dem Platzieren von Gegenständen verknüpft ist. Ein komplexeres Rätseldesign hätte überdies einen erheblichen Mehraufwand in der Entwicklung bedeutet. 
\chapter{Umsetzung des Prototyps}
Nachdem in Kapitel \ref{sec:concept} die Konzeption der Anwendung vorgestellt wurde kann, kann nun die Umsetzung der Anwendungen erfolgen.

Zunächst wird dabei auf die Ausgangssituation aus dem vorangegangenem Projekt vorgestellt. Im Anschluss erfolgt eine Auflistung der verwendeten Technologien. Danach erfolgt die Vorstellung der Anwendungen, sowie die Herausforderungen und Probleme in der Umsetzung

\section{Ausgangssituation}
In der vorausgehenden Arbeit wurde ein Art \ac{MVP} der Spielidee umgesetzt welche aus 3 Hauptanwendungen bestand. Es gibt eine Unity-Anwendung, in welcher das Spielgeschehen des \say{Players} umgesetzt wurde. Das Spielgeschehen der \say{Watcher}-Anwendung wurde über eine Vue3-Webseite realisiert. Für die Kommunikation der beiden Anwendungen untereinander wurde auf der Basis eines \say{Express.js} Node-Servers ein WebSocket-Server entwickelt. Der Node-WebSocket-Server kommuniziert zusätzlich mit einer MongoDB Datenbank, in welcher die Fortschritte der einzelnen Sessions gespeichert werden.

Die Anwendungen des \say{Players} und des \say{Watchers} sind in dieser Konstellation jeweils Anzeigende und auf die Eingaben des Nutzers reagierende Komponenten im gesamten System. Sie geben eine Rückmeldung an den Server, der die Daten zur Laufzeit abspeichert und persistent in einer Datenbank speichern kann.

\subsection{Aufbau der Ausgangssituation}

Im Softwaredesign wird dabei von einem \ac{MVC} Design-Pattern gesprochen (vgl. \cite{GlossarWiki:Reenskaug:1979a}). 
Das Model definiert, welche Daten die App enthalten soll. Ändert sich der Zustand dieser Daten, informiert das Modell die einzelnen Views, damit die Ansicht der Daten entsprechend aktualisiert werden können. Außerdem wird manchmal auch der Controller über Änderungen informiert. Die View definiert wie die Daten angezeigt werden sollen. Der Controller verarbeitet die reinkommenden Änderungen aus den Views, die die Nutzer getätigt haben, und gibt diese an das Model oder die Views direkt weiter (vgl. \cite{noauthor_mvc_2023}); (vgl. Abbildung \ref{fig:mvc-diagramm}).

\begin{figure}[ht]
\centering
\includegraphics[width=1\linewidth]{content/pictures/mvc-architecture.png}
\caption{\ac{MVC} Beispiel-Diagramm \cite{noauthor_mvc_2023}}
\label{fig:mvc-diagramm}
\end{figure}

Die MongoDB Datenbank und Klassen innerhalb des WebSocket-Servers nehmen die Rolle des Model ein, die einzelnen WebSocket-Nachricht Endpunkte übernehmen die Aufgaben des Controllers und die Anwendung des \say{Watchers} und des \say{Players} sind die Views der Architektur.

\subsection{Beitreten einer Session}

\begin{figure}[ht]
\centering
\includegraphics[width=1\linewidth]{content/pictures/Login_Login_by_ID.png}
\caption{Startbildschirme der Player und Watcher Anwendung (Quelle: eigene Darstellung)}
\label{fig:old-logins}
\end{figure}

Abbildung \ref{fig:old-logins} zeigt den bereits im alten Prototyp entwickelten Startbildschirm über welchen der Player eine neue Session starten (linkes Bild, links oben) und der Watcher dieser beitreten kann (linkes Bild, rechts unten). Sobald der Player eine Session erstellt hat, erhält er vom WebSocket-Server eine Rückmeldung mit der erstellten Session-ID (rechtes Bild, links oben) welches er dem Watcher mitteilen muss, damit dieser ihr beitreten kann (rechts Bild, rechts unten).

\subsection{Einführung in die Anwendungen}

\begin{figure}[ht]
\centering
\includegraphics[width=1\linewidth]{content/pictures/Introduction.png}
\caption{Einführung in die Anwendung des Players und Watchers (Quelle: eigene Darstellung)}
\label{fig:old-introductions}
\end{figure}

Abbildung \ref{fig:old-introductions} zeigt die Einführung der beiden Anwendungen in die Spielwelt. Zum Start erhielt der Watcher einzelne Tooltips, mit Erklärungen zu den Grundfunktionen seiner Anwendung (erstes bis drittes Bild in der ersten Zeile und linkes Bild in der zweiten Zeile; jeweils in weiß umrandet im rechten Bildelement). Seine Anwendung enthalten zwei Dropdown-Menüs, über die Gegenstände und Positionen ausgewählt werden können, (erste Zeile, Bilder links und und in der Mitte) eine Liste mit allen platzierten Gegenstände (erste Zeile rechts Bild), die jeweils einzeln entfernt werden können und eine Top-Down-Ansicht der Spielwelt, in der sich der Player befindet (zweite Zeile, linkes Bild). 

Der Player steuert seinen Avatar über eine Touch-druck auf die Spielwelt (zweite Reihe mittlere Bild, linkes Bildelement). Außerdem kann er über vertikale Swipes die Höher der Kamera zum Avatar verändern und dadurch in einem gewissen Rahmen die Ansicht verändern.

Um in der Spielwelt an das Ziel zu gelangen, müssen der Player und Watcher zusammenarbeiten und Hindernisse in der Spielwelt beseitigen. Im linken Bild in der zweiten Zeile im weiß umrandeten wird ein solches Rätselelement dargestellt, welches gelöst werden muss.
\subsection{Lösen von Rätseln}
Wie wurden im alten Prototyp die einzelnen Rätsel gelöst und durch wen erfolgte dies?

\begin{figure}[ht]
\centering
\includegraphics[width=1\linewidth]{content/pictures/HowToSolve.png}
\caption{Vorgang des Lösens von Rätseln (Quelle: eigene Darstellung)}
\label{fig:old-solving-riddle}
\end{figure}

Der Watcher war dafür verantwortlich, dass Gegenstände auf ihre richtigen Zielpositionen platziert wurden. Sobald der Player auf eine Absperrung in der Spielwelt stieß, musste er beschreiben was er sah um dem Watcher einen Hinweis darauf zu geben, welche Gegenstände platziert werden müssen und auf welche Positionen diese gehören. Zunächst wählt der Watcher über das Gegenstände-Dropdown einen entsprechenden Gegenstand aus (linkes Bild, rechtes Bildelement). Anschließend wählt er eine vorgegebene Position aus, die im derzeitig aktiven Abschnitt der Spielwelt hinzugekommen ist (mittlere Bild, rechts Bildelement). Über den Button \say{Gegenstand platzieren} (linker Button) wird der Gegenstand in die Spielwelt des Players platziert. Sowohl der Player als auch der Watcher erhalten vom System eine Benachrichtigung, dass der Gegenstand platziert wurde (rechts Bild, beide weißen Umrandungen). Auf der rechten Seite der Watcher-Anwendung erscheint zur selben Zeit wie die Benachrichtigung der platzierte Gegenstand in der Liste der platzierten Gegenstände (rechtes Bild, rechtes Bildelement). 

\subsection{Freischalten von Gegenständen und Positionen}
Sobald ein Rätsel durch das Platzieren von Gegenständen gelöst wurde, erhielten Player und Watcher die Information, dass neue Gegenstände freigeschaltet wurden (vgl. Abbildung \ref{fig:old-unlock-system}, erste Reihe linkes Bild, eingekreist in weiß). 

\begin{figure}[ht]
\centering
\includegraphics[width=1\linewidth]{content/pictures/UnlockMore.png}
\caption{Freischalten neuer Gegenstände (Quelle: eigene Darstellung)}
\label{fig:old-unlock-system}
\end{figure}

Sobald das Lösen des Rätsel das Hindernis beseitigt und einen neuen Bereich zugänglich macht, geht der Bilder-Slider in der Anwendung des Watchers auf das aktuelle Bild. Der Slider zeigt dem Watcher alle verfügbaren Abschnitte der Spielwelt (vgl. erste Reihe, rechtes Bild, in weiß umrandet in Abbildung \ref{fig:old-unlock-system}). Durch die neuen Spielabschnitte werden nicht nur neue Gegenstände freigeschaltet, sondern auch zusätzliche vordefinierte Positionen, auf denen die Gegenstände platziert werden können. Diese werden dem Watcher in den Gegenstand- und Positions-Dropdown aufgelistet (vgl. Abbildung \ref{fig:old-unlock-system}, beide Bilder in der zweiten Reihe in weiß umrandet). 

\subsection{Aspekte zum Überarbeiten}
\paragraph{Backfaces}
In den zuvor betrachteten Abbildungen \ref{fig:old-logins} bis \ref{fig:old-unlock-system} fällt auf, dass die Wände der Spielwelt, in der sich der Player befindet (jeweils immer das linke Bildelement), \say{eigenartig} aussehen. Das liegt daran, dass die einzelnen Raum-Elemente dafür gedacht sind, dass durch die Kameraperspektive der Spieler im inneren des Raumes zu sein sollte. Man müsste für diesem Raumaufbau eine First-Person Kameraansicht wählen.

\begin{figure}[ht]
\centering
\includegraphics[width=1\linewidth]{content/pictures/Backfaces.png}
\caption{Fehlende Rückwand-Oberflächen in den Raummodellen (Quelle: eigene Darstellung), (Modell von \cite{alasl_autolevel_nodate})}
\label{fig:missing-backfaces}
\end{figure}

Abbildung \ref{fig:missing-backfaces} zeigt fehlende Rückseiten-Oberflächen an. Die blauen Oberflächen sind die Oberflächen, die durch ihre Ausrichtung der Normalvektoren von der Game-Engine gerendert werden können. Die roten Oberflächen sind die, die von der Game-Engine nicht berücksichtigt werden. Hierbei kann es zwei unterschiedliche Ansätze geben, fehlende Rückseiten hinzuzufügen. Innerhalb der Material-Konfiguration der einzelnen Materialien kann bi-direktionales Rendering aktiviert werden oder das Modell muss pro Richtung auf die auf das Modell geschaut wird jeweils eine Oberfläche haben, die auch in diese Richtung zeigt.

Außerdem müsste für den Anwendungszweck eines Spiel mit einer Top-Down Ansicht auch eine Decke der Räume entfernt werden, da diese für den Spieler nicht sichtbar sein sollte und sonst stören würde.

% backfaces

\paragraph{Steuerung des Avatars}
Im vorangegangenem Prototyp wurde wenig Aufmerksamkeit auf die Gestaltung der Spielsteuerung über Touch-Inputs gelegt. Die Standard-Steuerung innerhalb des verwendeten Unity-Assets von \cite{alasl_autolevel_nodate} enthielt zum Start des Projekts nur die Steuerung über Maus und Tastatur. Über die Tastatur konnte sich über die Spielwelt hinweg bewegt werden, ähnlich wie es aus \ac{RTS}-Spielen bekannt ist. Über die linke Maustaste konnte in die Spielwelt geklickt werden, wodurch sich der Spieler-Avatar an diese Stelle selbständig bewegte. Eine Steuerung über Touch-Eingaben an einem Touch-Monitor oder Fernseher musste erst eingebaut werden. Das Asset von \cite{alasl_autolevel_nodate} verwendet das Kamera-System der Cinemachine (vgl. \cite{noauthor_about_nodate}) wodurch einige vorgefertigte Steuerungen der Kamera durch die Maus bereits implementiert sind. 

Es fehlte also die Touchsteuerung. Diese wurde allerdings nicht nach den Funktionalitäten eines Touchscreens konzipiert und umgesetzt, wie es bspw. in der Arbeit von \cite[S. 64ff]{reinhard_augmented_2022} aufgezeigt wurde, sondern über eigene Ideen, die nicht den Komfort der Maussteuerung wiedergeben konnte. Durch existierende und nicht sichtbare Oberflächen in der Spielwelt, wurde ebenfalls die Erfolgschance den Avatar durch einen Touch-Klick in die Spielwelt zu bewegen vermindert, wodurch ebenfalls Spielkomfort verloren ging.

% steuerung beim Player

\paragraph{Interaktion mit der Spielwelt}
Sowohl die Anwendung des Players als auch die des Watchers besitzen wenig, bis keine Interaktionen mit der Spielwelt. Bei der Anwendung des Players besteht lediglich das Laufen innerhalb der Spielwelt als Interaktion. Der Watcher hingegen besaß keine Funktionalität im Bezug auf die Spielwelt. Lediglich die Menüs, über die die Daten der Session bearbeitet werden konnten. Für einen zukünftigen Prototyp müssen mehr Kernfunktionen in die Anwendungen eingebaut und Interaktionen mit der Spielwelt geschaffen werden.

% interaktion mit der spielwelt beim watcher
% feedback von den probandentests aufzählen

\paragraph{Sonstiges Feedback aus den Probandentests}
\begin{figure}[ht]
\centering
\includegraphics[width=1\linewidth]{content/pictures/Handlungsempfehlungen.PNG}
\caption{Handlungsempfehlungen des alten Prototyps (Quelle: eigene Darstellung aus der Abschlusspräsentation), (ganze Präsentation in Anhang \ref{}, S. 33)}
\label{fig:recommended-action}
\end{figure}

Abbildung \ref{fig:recommended-action} zeigt die Handlungsempfehlungen für den alten Prototyp aus der vorangegangenen Arbeit. Die Abschnitte der \say{Player Interaktion}, \say{Player Ansicht}, \say{Kamerasteuerung} und \say{Watcher Ansicht} wurden bereits angesprochen. Die Abschnitte \say{Environment} und \say{Watcher Tutorial} fehlen noch, wobei \say{Watcher Tutorial} in der Weiterentwicklung des Prototyps in der Implementierung keine Beachtung geschenkt wurde, da der Umfang ein eingebautes Tutorial für \ac{UI}-Elemente zu konzipieren und umzusetzen zu aufwändig wäre. Es müsste außerdem auch für die Anwendung des Player gälten.

Daher wird nun noch der Fokus auf das Environment gelegt. Die Handlungsempfehlungen beziehen sich in der Darstellung auf das bestehende System des Prototyps. Abstrahiert betrachtet ermöglicht eine grundlegende Weiterentwicklung der Anwendungen und des Environment neue Möglichkeiten wie Rätsel innerhalb der Spielwelt konzipiert und umgesetzt werden können. Dadurch kann es frei wählbare Positionen auf der Spielkarte geben, die nicht mehr vordefiniert sein müssen, oder dass bestehende Gegenstände auf irgend eine weise verändert werden müssen. Diese Aspekte könnten in einer neu Entwicklung des Environments anders aufgebaut werden.

\section{Verwendete Technologien}
In der folgenden Aufzählung werden alle externen Assets und Packages vorgestellt, welche in der Entwicklung des Prototyps verwendet wurden. Manche der Assets werden im Registry des Projekts nicht aufgezählt, da diese in einem Test-Projekt importiert und für den Gebrauch dieses Projekt in jeweilige Submodule editiert importiert wurden. 

\subsection{Unity Editor Version 2022.3.45f1}
Der Prototyp wurde mit der 2022.3.45f1 \ac{LTS}-Version umgesetzt, da diese zum Start der Masterarbeit die aktuelle 2022 \ac{LTS}-Version war. Zwischenzeitlich wurde auch die 2023.2.20f1 (vgl. \cite{noauthor_unity_nodate}) ausprobiert. Da diese allerdings weder eine \ac{LTS}-Version noch zusätzlich im Package der Cinemachine ein Major-Update enthalten war und die Cinemachine einige Probleme mit sich führte, wurde eine stabile 20222-Version gewählt.

\subsection{Blender 4.3.2}
Blender dient als Bearbeitungstool für die 3D-Modelle aus den hinzugenommenen Unity Assets. Die aktuelle Version zu Projektstart war die 4.3.2 Version- welche über den gesamten Bearbeitungszeitraum verwendet wurde. Außerdem bietet Unity einen Blender Direktimport an, wodurch keine \ac{FBX} oder \ac{OBJ} Dateien aus den Blenderdateien exportiert und in Unity importiert werden mussten.

\subsection{NuGetForUnity 4.1.1}
NuGetForUnity ist ein für Unity entwickelter NuGet-Client (vgl. \cite{noauthor_nuget_nodate}) über den zusätzliche funktionale Pakete für Unity installiert werden kann (vgl. \cite{mccarthy_glitchenzonugetforunity_2025}). Er war nötig, um ein nicht über Unity erworbenes Package in Unity nutzen zu können. Die aktuelle Version zum Bearbeitungsstart war die 4.1.1 Version, welche bis zum Ende genutzt wurde.

\subsection{Newtonsoft.Json 13.03}
Das erste Package, welches über NuGetForUnity installiert wurde ist Newtonsoft.Json, durch welches über den WebSocket übertragene JSON-Daten leicht deserialisiert und serialisiert werden können (vgl. \cite{newtonsoft_jsonnet_nodate}). Die aktuelle Version im NuGet-Paketverwaltungstool war die 13.03.

\subsection{WebSocketSharp-netstandard 1.0.1}
Derzeit gibt es einige Netzwerk-Integrationen für Unity, bspw. Mirror (vgl. \cite{noauthor_mirror_nodate}) oder Netcode for GameObjects (vgl. \cite{noauthor_about_2025}). Durch beiden Packages wäre jedoch die Netzwerk-Topologie starrer und die Entwicklung nach eigener Vorstellung wäre ebenfalls eingeschränkter. Daher wurde über NuGetForUnity das WebSocketSharp-netstandard Paket installiert, welches eine direkte Kommunikation mit einem WebSocket-Server ermöglicht (vgl. \cite{pingman_tools_pingmantoolswebsocket-sharp_2025}), da bislang existierende Integration mit Unity nicht mehr unterstützt werden. Die derzeit installierte und zum Projektstart installierbare Version ist die 1.0.1.

\subsection{AI Navigation 1.1.5}
Da sich der Avatar des Players innerhalb der Spielwelt per Touch-Click auf den Bildschirm an die angeklickte stelle bewegen soll, muss ein \ac{AI}-Agent eingebaut werden, der das Avatar Spielobjekt an die gewünschte Stelle bewegt. In Unity kann dafür das NavMesh System verwendet werden, welches ein Pathfinding-System implementiert, wodurch ein automatisiertes Bewegen eines Agents durch eine Zielposition integriert werden kann. Das Paket in Unity heißt dafür \ac{AI}-Navigation. Die aktuelle Version zum Start des Projekts ist die 1.1.5.

\subsection{Cinemachine 2.10.1}


\subsection{Universal RP 14.0.11}

\subsection{Unity UI 1.0.0}

\subsection{TextMeshPro 3.0.9}

\subsection{Input System 1.7.0}

\subsection{FBX Exporter 4.2.1}

\subsection{Unity Assets}

\begin{itemize}  
    \item Astronaut Model von \cite{quaternius_astronaut_nodate}
    \item Chair Pack - 3D Low Poly Office Furniture 1.0 \cite{fast_mesh_chair_nodate}
    \item low poly WD | 3D Props 1.1 \cite{squid_low_nodate}
    \item Adventure Game Environment Pack | URP | 3D Sci-Fi 1.0 \cite{unity_technologies_adventure_nodate-1}
    \item Adventure - Sample Game | Tutorials 3.0 \cite{unity_technologies_adventure_nodate}
    \item Bedroom / Interior - Low Poly assets | 3D Interior 1.1.6 \cite{fries_and_seagull_bedroom_nodate}
    \item Big Furniture Pack | 3D Furniture 1.3 \cite{vertex_studio_big_nodate}
    \item Chair pack - 3D Low Poly Office Furniture - Created with FastMesh Asset | 3D Furniture 1.0 \cite{fast_mesh_chair_nodate}
    \item Fantasy Cemetery \& Necropolis Pack Lite: 3D Assets for RPG and Adventure Games | 3D Fantasy 1.2 \cite{emaceart_fantasy_nodate}
    \item Free Wood Door Pack | 3D Interior 1.0 \cite{biostart_free_nodate}
    \item Kitchen Appliance - Low Poly | 3D Electronics 1.02 \cite{alstra_infinite_kitchen_nodate}
    \item Lowpoly Dungeon Assets | 3D Dungeons 1.0 \cite{kunniki_lowpoly_nodate}
    \item Low Poly Dungeon Generator | 3D Dungeons 1.0 \cite{mysticforge_low_nodate}
    \item Low Poly Dungeons Lite | 3D Dungeons 1.11 \cite{justcreate_low_nodate-1}
    \item Low Poly Simple Medieval Props | 3D Props 1.0 \cite{justcreate_low_nodate}
    \item LowPoly Server Room Props | 3D Environments 1.0 \cite{ipoly3d_lowpoly_nodate}
    \item Melon's Low Poly Office | 3D Interior 1.1.1 \cite{mistyczny_arbuz_melons_nodate}
    \item Office Pack - Free | 3D Interior 1.02 \cite{nappin_office_nodate}
    \item Ultimate Low Poly Dungeon | 3D Dungeons 2.0 \cite{broken_vector_ultimate_nodate}
\end{itemize}

\subsection{Docker}
\cite{noauthor_docker_2025}

\subsection{MongoDB - Docker Image}
\cite{noauthor_mongo_nodate}
\cite{mongodb_mongodbnode-mongodb-native_2025}

\subsection{Express.js Webserver 5.1.0}
\cite{noauthor_express_nodate}

\subsection{Node Version 20.18.1}
\cite{noauthor_nodejs_nodate}

\subsection{User Interface Inspirationen}

\begin{itemize}
    \item Vorlage für SciFi \ac{UI}s von \cite{pchvector_free_nodate}
    \item Vorlage für Low-Poly Feder von \cite{masud_download_nodate}
    [TODO: Hier fehlen noch die Dumps von CHat, die nachgemalt wurden]
\end{itemize}

\section{Aufbau des Prototyps}



\section{Herausforderungen in der Umsetzung}
% erste Ansätze im Levelbuilding erwähnen

% hier kommt das placing also die positionen in der spielwelt

% dann drauf bezogen das mit den slots, dass die ne position haben später auch fürs setzen wichtig
% das mit den collidern, was man auch anders umsetzen kann noch, das noch erwähnen






% bei umsetzung müssen die 3d und ar anwendung vorgestellt werden vom watcher
% hier muss ein bezg auf das paper mit den steuerungen für touch erwähnt werden


% die anwendung vom playewr

\section{Probleme in der Umsetzung}

% AR

% Gerätefindung im selben Netzwerk
\chapter{Evaluation des Prototyps sowie der Wirkung des Prototyps auf das Kommunikationsverhalten der Probanden}

Dieser Abschnitt behandelt das Testen der umgesetzten Anwendung sowie das Messen des Effektes, den die Anwendung haben soll. Er zeigt auf, an welchen Elementen des Prototyps Weiterentwicklungen von Nöten sind und gibt erste Ergebnisse darauf, wie seine Wirkung ist.

% Die beiden Studienschwerpunkte 


\section{Methodik}
% Die Evaluationsphase zielt primär darauf ab in der Kommunikationsforschung von asymmetrischen-Multiplayer Spielen neue Erkenntnisse zu liefern. Der gewählte Ablauf der primären Forschungsstudie erfolgt innerhalb eines kleines Experimentes, bei dem die Probanden zunächst nicht den Zweck der Studie erfahren. DIe Ergebnisse dieses Experiments werden auf quantitative Weise erfasst.
% % Die gewählte Methodik der Bewertung des Einflusses auf die Probanden ist dabei diejenige, die bei den zugrunde Legenden Arbeiten für ihre quantitativen Ergebnisse verwendet wurde. 

% Während der Kommunikationseinfluss im Vordergrund steht, wird ebenfalls auch die entwickelte Anwendung geprüft. Bei der hierfür gewählten Forschungsmethode wurden quantitative als auch qualitative Daten gesammelt. Diese dienen dazu einen Kenntnis stand darüber zu erhalten, an welchen Aspekten die Anwendung weiterentwickelt werden muss.
% Die Evaluation der Anwendung erfolgt über standardisierte Fragebögen, die in vergleichbaren Studien ebenfalls auf diese Weise verwendet wurden. Zusätzlich wurden über einen anderen Fragebogen qualitative Ergebnisse eingeholt.

Dieser Abschnitt beschreibt die methodische Herangehensweise zur Evaluation der entwickelten Anwendung im Kontext der Kommunikationsforschung bei asymmetrischen-Multiplayer Spielen. Ziel war es, sowohl die kommunikative Wirkung der Anwendung als auch ihre funktionale und gestalterische Tauglichkeit zu Untersuchung. Die gewählte Vorgehensweise kombiniert qualitative und quantitative Methoden innerhalb eines experimentellen Studiensettings, um ein möglichst umfassendes Bild der Nutzung und der Interaktionen zu gewinnen.

\subsection{Forschungsdesign}
Zur Untersuchung der Forschungsfragen \say{Welche Verbesserungen in der Kommunikation zwischen den Anwendern können durch ein asymmetrisches Multiplayer-Spiel mit zwei verschiedenen Spielerklassen beobachtet werden?} und \say{Wie stehen die Nutzer zu einem spielerischen Ansatz und zur Verbesserung der Kommunikation, insbesondere auch im Umgang mit Fremden?} wurde ein praxisorientiertes, experimentelles Forschungsdesign gewählt. Im Zentrum steht das asymmetrische-Multiplayer-Szenario, in dem jeweils zwei Personen unterschiedliche Rollen mit ungleich verteilten Informationen übernehmen. Diese Konstellation ermöglicht es, die Wirkung der Anwendung auf kooperative Kommunikationsprozesse zu analysieren.

Das Design sieht in der Erhebung sowohl quantitative als auch qualitative vor. 

\subsection{Erhebungsinstrumente}
Die Datenerhebung dieser Studie erfolgte mittels standardisierte Fragebögen zu den Themen System Usability, Immersion, Spiel Erfahrung Motivation und dem Workload. Außerdem wurden über standardisierte Fragebögen die Entwicklung des affektiven Status, der Beziehung zueinander und Leadership. Außerdem wurden neu erstelle Fragebögen zum Thema Umgang mit fremden Menschen und Demographie verwendet.

Zusätzlich wurden Videoaufnahmen von der Versuchsdurchführung gemacht, die zur Auswertung der Kommunikationsentwicklung notwendig sind.

\subsection{Stichprobe}
[hier quellen finden zu stichproberngröße bei statistischen auswertungen, expiermenten bei denen 2 oder mehrere personen mitmachen und zu qualitativer forschung]

\subsection{Durchführung der Studie}
Zunächst wurden Probanden über Einladungsnachrichten in diversen Chatgruppen, sowie mehrere Rundmails gesucht und eingeladen. Sie konnten sich in einer Terminkalender-Anwendung in einen für den Probandentest verfügbaren Zeitslot eintragen.

Beim Probandentest wurden die Teilnehmer zunächst begrüßt und mussten eine Einverständniserklärung unterzeichnen, da sie während des Versuchsdurchlaufes gefilmt wurden. Den Probanden wurde zunächst nicht erklärt um was es in dem Versuchsdurchlauf im Detail geht. Sie wussten zunächst nur, dass sie einen asymmetrischen Multiplayer spielen dürfen, bei denen zu zweit Rätsel gelöst werden müssen und jeder Spielteilnehmer eine andere Rolle im Spiel einnehmen wird. Das gezielte Verschweigen des Sinns der Nutzerstudie wurde gemacht, damit die Teilnehmer nicht voreingenommen in die Aufgaben des Versuchsaufbaus gehen. Dies würde sonst dazu führen, dass die Ergebnisse verfälscht werden. Die Art der Anleitung orientierte sich dabei an die des Milgram Experiments 1963 (vgl. \cite{milgram_behavioral_1963}) bei dem den Probanden ebenfalls nicht der eigentliche Zweck des Versuches erklärt wurde, damit das Ergebnis nicht verfälscht wird. Sie haben lediglich Anweisungen zu den Aufgaben, die sie machen mussten erhalten. 

Zu Beginn des Versuchsdurchlaufs füllten die Probanden den demografischen Fragebogen aus, bei dem Alter, Geschlecht und Vorerfahrungen mit Spielen, Multiplayer-Spielen und Touchsteuerungen aus. Anschließend folgten die standardisierten Fragebögen zur gegenseitigen Beziehung, zum Spielertyp und den aktuellen affektiven Status aus.

Der Kern der Versuchsdurchführung wird in 3 Komponenten geteilt. Die Hauptkomponente ist das Spielen des Prototyps, welcher zwischen einem Vortest und Nachtest gespielt wurde.

Die Probanden mussten nun zunächst den Vortest absolvieren. 

\begin{figure}[ht]
\centering
\includegraphics[width=1\linewidth]{content/pictures/MazeScape.jpg}
\caption{Verpackung des Spiels MazeScape (vgl. \cite{noauthor_mazescape_nodate})}
\label{fig:mazescape}
\end{figure}

\begin{figure}[ht]
\centering
\includegraphics[width=1\linewidth]{content/pictures/MazeScape_Level02.jpg}
\caption{Level 02 aus dem Spiel MazeScape (vgl. \cite{noauthor_mazescape_nodate})}
\label{fig:mazescape_level-02}
\end{figure}

% Der Vortest umfasste das Spielen eines Levels aus dem Spiel MazeScape (vgl. Abbildung \ref{fig:mazescape}). Die Regeln wurden für den Anwendungszweck abgeändert. Die Probanden mussten zu zweit die 
Der Vortest umfasste das Spielen des zweiten Levels (vgl. Abbildung \ref{fig:mazescape_level-02}) aus dem Spiel \say{MazeScape} (vgl. Abbildung \ref{fig:mazescape}). Die grundlegenden Regeln des Spiels bezüglich der Bewegung durch die Spielwelt wurden für diesen Anwendungszweck übernommen. Die weiteren Regeln den Spiels zu bestimmten Gegenständen, Punkten oder Gegnern wurden aus dem Grund der fehlenden Relevanz nicht berücksichtigt. Die Probanden hatten zehn Minuten Zeit um gemeinsam vom Startpunkt im Level an das Ziel zu kommen. Es war dabei nicht wichtig rechtzeitig ans Ziel zu kommen.

Dieser Vortest wurde gemacht, damit ein Grundwert der bestehenden Kommunikation zwischen den Probanden zu bestimmen. Dieser sollte sich nach dem Nachtest verbessert haben.

Nach erfolgreich oder nicht erfolgreichem Abschluss des Vortests durften die Probanden den Prototyp von Connecting-Minds testen. Sie erhielten zunächst eine Einführung zur Steuerung der jeweiligen Anwendung. Anschließend durften sie starten. Sie hatten hierfür 40 Minuten Zeit. Ziel war es, die Rätsel zu lösen, allerdings war es nicht wichtig, ob sie das Tutorial erfolgreich in der gegebenen Zeit abschließen.

Die Zuordnung welcher Proband welche Spielrolle einnimmt, wurde nach der Begrüßung der Probanden vollzogen. Für die ersten Fragebögen war es wichtig zu wissen, wer welche Rolle im Spiel einnimmt. Daher erfolgt zum Ausfüllen der Einverständniserklärung die Einteilung von den Probanden ausgehend.

Nach absolvieren des Prototyps mussten die Probanden die Fragebögen zu den Themen System Usability, Immersion, Spiel Erfahrung Motivation und dem Workload des jeweiligen Probanden ausfüllen.

Im Anschluss folgte nun der Nachtest. Hierfür mussten die Probanden erneut ein Level aus dem Spiel MazeScape spielen. Sie spielten dabei das Level 3 (vgl. Abbildung \ref{fig:mazescape_level-03}). Wie im Vortest wurden die gleichen Regeln zur Bewegung durch die Spielwelt für den Versuch angewandt. Alle anderen Regeln wurden ebenfalls nicht berücksichtigt. Die Schwierigkeit, die sich in diesem Level für die Probanden ergeben hat, ist dass sie dass Ziel zunächst zusammen suchen müssen ehe sie einen Weg durch das Labyrinth finden können.

\begin{figure}[ht]
\centering
\includegraphics[width=1\linewidth]{content/pictures/MazeScape_Level03.jpg}
\caption{Level 03 aus dem Spiel MazeScape (vgl. \cite{noauthor_mazescape_nodate})}
\label{fig:mazescape_level-02}
\end{figure}

Wie im Vortest hatten die Probanden hierfür ebenfalls zehn Minuten Zeit. Außerdem war es ebenso nicht wichtig das Ziel zu erreichen.

Nach Abschluss wurden die letzten Fragebögen von den Probanden ausgefüllt. Sie behandelten Vergleichswerte im Bezug auf die gegenseitige Beziehung und dem affektiven Status. Außerdem mussten sie Fragebögen zum Thema Leadership und dem Umgang mit Menschen in diesem Versuchsaufbau ausfüllen. Zuletzt hatten sie noch ein Freitextfeld, bei dem ihr qualitatives Feedback zur Anwendung gegeben werden konnte.

Nachdem alle Fragebögen ausgefüllt wurden, wurde die Aufnahme beendet und es wurde sich bei den Probanden für die Teilnahme bedankt. Sie durften sich noch etwas von den bereitgestellten Snacks nehmen und wurden im Anschluss entlassen.

Die gesamte Versuchsdurchführung dauerte etwa 75 Minuten pro Probanden-paar.

\section{Ergebnisse}

% \section{Zielsetzung}

% \section{Planung und Durchführung}

% \section{Auswertung der Tests}


\chapter{Diskussion der Ergebnisse} 
Ziel dieser Arbeit war es das Konzept und die Umsetzung der vorausgegangenen Arbeit zu überarbeiten und für die Anwendung des Watchers zwei unterschiedliche Versionen anzulegen. Eine der beiden Versionen sollte dabei im \ac{AR} gespielt werden können, die andere Version soll dabei eine \ac{RTS}-ähnliche Anwendung für das Smartphone werden. Anhand der beiden Anwendungsformen sollen Unterschiede im Kommunikationsverhalten der Probanden (hauptsächlich der Watcher und die Reaktionen der Player darauf) beobachtet und ausgewertet werden. Außerdem sollte anhand der Anwendungen evaluiert werden, welche den Watchern das größte Maß an Immersion bietet um dieses Empfinden in Bezug auf die Player-Anwendung anzugleichen. 
Durch die nicht erfolgreiche Umsetzung der \ac{AR}-Integration konnte die entstandene Zeit sinnvoll in die umfassende Forschung von verwandten Arbeiten und Auswertungsmethodiken investiert werden. Dies half ungemein einen Überblick über die bestehende Forschung zu gewinnen und mögliche Formen der Auswertung zu bestimmen.
%  Ziel der Arbeit nochmal vorstellen; darunter fallen auch die Einzelstudien die damit noch einhergegangen wären (a/b test) -> aufzeigen, dass durch die eingrenzung und dem nicht möglichen umsetzen einer funktionierenden ar anwendung intensiver mit der forschung beschäftigt werden konnte

Ein wesentlicher Bestandteil dieser Arbeit war die Konzeption und Entwicklung eines asymmetrischen Multiplayers, bei dem zwei verschiedene Spielerrollen miteinander an ein gemeinsames Ziel gelangen. Um diese Rollen bestmöglich zu gestalten, musste zunächst recherchiert werden, in welcher Form sich die Rollen unterschieden müssen. Durch die umfassende Recherche konnte dabei schnell das Thema \say{Interdependence} (in Deutsch: Abhängigkeiten) gefunden und als Basis des Konzeptes festgelegt werden. Aufgrund dessen konnten die Aufgaben der Spielerrollen und die Rätsel bestmöglich konzipiert und umgesetzt werden.

% wesentlicher bestandteil der anwednung waren die interdependencen zwischen den anwendungen, sowie die netzwerkinfrasdtruktur

Das Konzept setzte voraus, dass drei verschiedene Anwendungen entwickelt werden mussten. Die Entwicklung der Netzwerkinfrastruktur konnte dabei auf der Serverinfrastruktur der vorangegangenen Arbeit aufsetzen, sodass nur noch Inhalte für den neuen Prototyp entwickelt werden mussten. Unter Inhalte sind Endpunkte für die WebSocket Nachrichten und die entsprechenden Express Endpunkte für die Datenbank Kommunikation gemeint. Das bestehende Konstrukt konnte dabei einen enormen zusätzlichen Aufwand verhindern. Anders sah es bei den Anwendungen des Players und Watchers aus. Diese mussten von Grund auf neu Entwickelt werden. Dabei entstanden neue Herausforderungen die bewältigt werden mussten. Zunächst mussten die Kernelemente der Netzwerkkommunikation für beide Anwendungen eingerichtet werden, außerdem brauchten beide Anwendungen Zugang zu allen Komponenten und \ac{3D}-Objekten die verwendet werden sollten. Der die dabei entstandenen koordinativen Herausforderungen sind für die Art einer solchen Arbeit zu anspruchsvoll gewesen. Außerdem musste ein passendes und sinnvolles Szenario für den Gesamtverlauf der Handlung entworfen werden, welcher zuvor nicht existent war. Rückblickend hätte zunächst der Ansatz verfolgt werden sollen, das bestehende Fundament weiter zu entwickeln, anstatt von Grund auf neue Wege zu gehen. Auf diese Weise hätte in der anfänglichen Umsetzung der Spielwelt Zeit eingespart werden können, welche für die Ausgestaltung der Anwendungen sinnvoller investiert worden wäre.
% die fehldene Grundlage die anwendungen zu entwickelten verursachten einen hohen aufwand, da in 2/3 der anwendungen von 0 gestartet werden musste und sich die welt erst überlegt werden musste; glücklicherwiese konnte bei der entwicklung des netzwerkes auf einer sehr gut und breit ausgelegten grundlage aufgebaut werden und quasi nur inhalte entwickelt werden mussten

Zu Beginn der Umsetzung wurden zunächst die Kernfunktionen der Anwendungen entwickelt. Im Anschluss erfolgte, unterstützt durch die Ergebnisse der Spielanalysen, die Umsetzungen der Szenarien aus dem entwickelten Tutorial. Dieser Ansatz zeigte sich als Fehler, da die \ac{AR}-Funktionalitäten nicht mehr rechtzeitig fehlerfrei umgesetzt werden konnten, bevor die geplanten Nutzertests stattfinden sollten. Bevor überhaupt mit der Umsetzung begonnen wurde, hätte bereits eine Recherche stattfinden sollen, welche Version der ARFoundation fehlerfrei funktioniert, um darum die Versionen des Unity-Editors und der anderen Packages auszuwählen. Außerdem hätte die etablierte Game-Loop des Prototyps auch besser an die Gegebenheiten der \ac{AR}-Funktionalitäten angepasst werden können. Dabei geht es primär um das Platzieren der Spielwelt und wie sich die enthaltenen Komponenten in der gesamten Game-Loop verhalten. Über den in der Entwicklung gewählte Weg, war die Flexibilität des Systems nicht mehr wie gewünscht gegeben und konnte auch aus der Sorge eines Funktionsverlusts der genutzten Erweiterungen kein Major-Downgrade der ARFoundation-Version vollzogen werden.

% der Fokus erst die kernmechanik zu entwickeln und dann sich um die entwicklung der ar anwendung zu kümmern zeigte sich als ein Fehler, da so die ar anwendung mangels fehlerhafter version nicht funktionsfähig war; man hätte während der entwicklung des systems bzw vor der entwicklung darauf achten sollen, welche version für die ar integrtation die richtige gewesen wäre

Die Anwendungen des Players und Watchers sollten eine gute Performance haben. Dies war ein zentrales Nebenziel in der Entwicklung. Um eine gute Performance zu erhalten wurden sogenannte Low-Poly Modelle verwendet, welche der Engine für die Darstellung einen geringen Rechenaufwand bietet. Zudem wurden Lightmaps angelegt, welche das Berechnen von dynamischen Lichtquellen in Echtzeit auf ein Mindestmaß reduziert hat. Außerdem boten die Lightmaps ein gutes Stimmbild ab, welches auch den Probanden positiv in Erinnerung blieb. Zusätzlich dazu wurde das Occlusion-Culling von Unity aktiviert, wodurch nur für die Kamera sichtbare Objekte berechnet werden. Dies fördert das Benutzererlebnis durch eine gute Performance. Ob das Occlusion-Culling in den entsprechenden Builds aktiv war und einen Einfluss hatte, konnte rückblickend leider nicht überprüft werden. Allerdings gab es während der Probandentests keine Performance-Probleme wodurch sich der Einsatz gelohnt hat.

% Außerdemn lag ein Fokus auf der Perfomrnace der Anwendungen wodurch mit vorgeneriteren Lightmaps und dem occlusion culling gearbeit wurde; Die Leightmapsm zeigten sich in den fertigen Anwendungen als zuiemlcih stimmig und erzeugten ein gutes Feedbaxck der Teilnehmer, ob das occlusion culling aktiv war konnte leider nicht besttigt werden, allerdings war die insgesammte performance der Anwendungenm, was framerate angeht, sehr gut 

Die Analyse der artverwandten Spielen und Arbeiten gaben einen guten Überblick über bestehende Funktionalitäten und Mechaniken und gaben in der Frühphase der Konzeption gute Differenzierungsmöglichkeiten um Alleinstellungsmerkmale etablieren zu können. Gleiches gilt auch in der theoretischen Arbeit des Rätseldesigns, welches insgesamt einen guten Einblick in die Konzeption von Rätseln gibt. 

% Durch die Analyse von bestenenden Spielen konnten frühzeitig differenzioerungen in den mechaniken festgelegt werden und man sich relativ schnell in der konzeptionsphase um die entwicklung eigener rätsel konzentrieren konnte

Video- und Computerspiele werden durch nahezu alle Altersgruppen hinweg gespielt. Daher wäre es nur logisch gewesen nicht nur Studierende der ehemaligen Fakultät Digitale Medien (jetzt Fakultät 1 und 4) für die Probandentests einzuladen, sondern auch Mitarbeitenden/ Dozierende der Fakultäten. Außerdem gäbe es auch die Option Selbständige oder Angestellte des Impact-Hubs in Stuttgart zu den Probandentests einzuladen. Aufgrund des großen Aufbaus des Spiels und der umfassenden Forschungsstudie wurde sich nur auf den Aufbau an der Fakultät konzentriert. Des weiteren hätte bei anderen Fakultäten oder anderen Standorten um weitere Probanden bemüht werden können. Aufgrund der aufwändigen und Planung und Organisation wurden diese Möglichkeiten leider außer acht gelassen. Zumal es in Schwenningen offenbar Pflichtteilnahmen für Probandentests gibt, welche für den Zweck dieser Studie gerne zugunsten genommen worden wäre. So wurde leider die Chance nicht genutzt mehr als nur sieben Versuchsdurchläufe durchführen zu können.

% man traf zunächst die Wahl nur innerhalb der Hochschule nach Probanden zu fragen, primär studierende; für den Zweck der Studie hätten auch weitere personengruppen wie Dozenten und Mitarbeiter aquiriert werrden können; außerdem hätten noch weitere Probanden im Co-Working Space meiner Fiorma nach mehr Probanden gefragt werden können; so wurden es nur 14 Probanden bei 7 Versuchsdurchläufen

Der Aufbau des Probandentests mit der ausgeliehenen Hardware hätte durch ein zentrales Hosting der Datenbank und des Express Server deutlich reduziert werden können. Die einzelnen Anwendungen hätten in der Entwicklung am eigenen Arbeitsplatz vorbereitet werden können und dann als fertige Anwendung zum Testraum gebracht werden können. Da sich zwischenzeitlich nicht mehr (aus Gründen des Umfangs und der begrenzten Zeit) um ein Hosting bemüht wurde, wurde die Netzwerkinfrastruktur über eigene Hardware umgesetzt. Die Herausforderung dabei ist die, dass in den einzelnen Anwendungen jeweils die IP-Adresse des Gerätes angegeben werden muss, auf dem der Express-Server ausgeführt wird. Anfangs wurde ein TP-Link Router ausgeliehen, für den zunächst die Projekte auf den ausgeliehenen Geräten importiert und gebaut werden mussten. Ein Aufwand der durch einen eigenen Router gemindert werden könnte. Außerdem gab es im Verlauf der Vorbereitungen der Probandentests ach Herausforderungen bezüglich der Koordination der Räume und der Hardware, welche allesamt mit einem positiven Ergebnis gelöst werden konnten.

% Für den Zweck der Probandentests hätte auch ein geplantes hosting untersatützen können, da es den Aufbau verkleinert hätte und die Tests insgesamt angenehmer gestaltert hätte; allerdings wäre man so weniger flexibel, falls Probleme aufgestoßen wären

% Bei der Durchführung der Tests gabs einige orgamnisatorische Herausforderung wie Raumreservierungs überschneidungen oder ausgeliehende Hardware die nach Absprache plötzlich doch nicht mehr zur Verfügung standen, diese konnten aber geklärt werden


% insgesamt zeigte sich trotz des sehr großen Umfangs der Arbeit eine ordentliche Entwicklung eines stylistischen Prototyps mit ansprechender FOrschungsmethode und Durchführung, auch wemn die Ergebnisse der Studie nicht die gewünschten Ergebnisse liefern konnten.


\chapter{Fazit}

Ziel dieser Masterarbeit war es, mit Connecting-Minds einen Spiel-basierten Prototypen zu entwickeln, der als Versuchsumgebung zur Untersuchung kommunikativer Prozesse in asymmetrischen Multiplayer-Szenarien dient. Die zentrale Forschungsfrage lautet, inwiefern durch ein solches Spielkonzept die Kommunikation zwischen zwei Personen, insbesondere im Umgang mit zunächst fremden Personen, verbessert werden kann.

Die Analyse der quantitativen Daten zeigt, dass der Prototyp Potenzial besitzt, um soziale Nähe zwischen Spielpartnern zu fördern. Zwar konnten aufgrund der geringen Stichprobengröße keine signifikanten Veränderungen im Kommunikationsverhalten nachgewiesen werden, jedoch deuten einzelne Effekte, wie etwa die mittleren Korrelationen bei Gesprächsinitiativen oder der Pausenzeiten, auf erste positive Tendenzen hin. Darüber hinaus wurde das Spielkonzept von den Probanden überwiegend positiv bewertet, insbesondere in Bezug auf auf seine Originalität und motivierende Wirkung.

Gleichzeitig wurden auch Schwächen offensichtlich. Vor allem die Bedienbarkeit der Watcher-Anwendung wurde vielfach kritisch bewertet. Eine klarere Trennung der Gestensteuerung (Zoom vs. Rotation) sowie eine konsistente UI-Gestaltung stellen zentrale Ansatzpunkte für zukünftige Iterationen dar. Auch konzeptionell wurden einige Rätsel als nicht vollständig durchdacht empfunden und sollten überarbeitet werden, um das das interdependente Zusammenspiel der beiden Rollen stärker herauszuarbeiten.

\section{Ausblick}\label{sec:prospect}

Diese Arbeit legt die Grundlage für weitere Forschungen im Bereich kooperativer Kommunikation durch asymmetrische Multiplayer Adventure-Spiele.

Aufbauend auf den gewonnenen Erkenntnissen sollten zukünftige Studien eine größere Stichprobenzahl einbeziehen, um eine größere Anzahl statistisch signifikanten Aussagen treffen zu können. Darüber hinaus wäre es sinnvoll, Zwischenmessungen zu integrieren, um Entwicklungsverläufe differenzierter erfassen und interpretieren zu können. Ergänzend könnten noch weitere Erhebungsinstrumente, etwa Fragebögen zur wahrgenommenen Interdependenz oder der qualitativen Usability, eingesetzt werden, um den Prototyp gezielter weiterentwickeln zu können. Auch das bestehende Rätsel- und Interface-Design bietet Potenzial für iterative Weiterentwicklungen, insbesondere im Hinblick auf Benutzerfreundlichkeit, Rollenbalance und die spielmechanische Unterstützung kooperativer Zusammenarbeit.

Das Konzept Connecting-Minds bietet nicht nur eine spannende Möglichkeit das Kommunikationsverhalten von Dyaden zu verbessern, sondern auch einen allgemeinen Spielspaß mit einer neuartigen Spielmechanik. Wenn es gelingt, die gebrauchstauglichen Aspekte zu verbessern und die Konzeption final umzusetzen, steht dem Spiel nichts im Wege, es der breiten Öffentlichkeit zugänglich zu machen.
% \chapter{Ausblick}\label{sec:prospect} 


% Schalgwortverzeichnis (Index)
%\printindex

% Literaturverzeichnis
\singlespacing
\bibliographystyle{agsm}
\bibliography{references}

% Eidesstattliche Erklärung
\chapter*{Versicherung über redliches wissenschaftliches Arbeiten\markboth{Versicherung über redliches wissenschaftliches Arbeiten}{}}
% Eintrag in das Inhaltsverzeichnis 
\addcontentsline{toc}{chapter}{Versicherung über redliches wissenschaftliches Arbeiten}

Hiermit versichere ich, Nick Philipp Häcker, dass ich die vorliegende Arbeit selbstständig verfasst und erstellt habe. Ich versichere, dass ich nur zugelassene Hilfsmittel und keine anderen als die angegebenen Quellen und Hilfsmittel benutzt habe. Ferner versichere ich, dass ich alle wörtlich oder sinngemäß übernommenen Stellen in der Arbeit gemäß gängiger wissenschaftlicher Zitierregeln korrekt zitiert und als solche gekennzeichnet habe. Darüber hinaus versichere ich, dass alle verwendeten Hilfsmittel, wie KI-basierte Chatbots (bspw. ChatGPT), Übersetzungs- (bspw. Deepl), Paraphrasier- (bspw. Quillbot) oder Programmier-Applikationen (bspw. Github Copilot) vollumfänglich deklariert und ihre Verwendung an den entsprechenden Stellen angegeben und gekennzeichnet habe.\newline

Ich bin mir bewusst, dass die Nutzung maschinell generierter Texte keine Garantie für die Qualität von Inhalten und Text gewährleistet. Ich versichere, dass ich mich textgenerierender KI-Tools lediglich als Hilfsmittel bedient habe und in der vorliegenden Arbeit mein gestalterischer Einfluss überwiegt. Ich verantworte die Übernahme jeglicher von mir verwendeter maschinell generierter Textpassagen vollumfänglich selbst. \newline

Auch versichere ich, die „Satzung der Hochschule Furtwangen (HFU) zur Sicherung guter wissenschaftlicher Praxis“ vom 27. Oktober 2022 zur Kenntnis genommen zu haben und mich an den dortigen Ausführungen zu orientieren.
Mir ist bewusst, dass meine Arbeit auf die Benutzung nicht zugelassener Hilfsmittel oder Plagiate überprüft werden kann. Auch habe ich zur Kenntnis genommen, dass ein Verstoß gegen § 10 bzw. § 11 Absatz 4 und 5 der Allgemeinen Teile der HFU-SPOen zu einer Bewertung der betroffenen Arbeit mit der Note 5 oder mit «nicht ausreichend» und/oder zum Ausschluss von der Erbringung aller weiteren Prüfungsleistungen führen kann.


Ort, Datum		  \hspace{5cm}                  Unterschrift


\vspace*{1.5cm} \par
\line(1,0){200} \par
\docOrt, \docAbgabedatum ~~\docVorname~\docNachname


%Zurücksetzen \chaptermark
\let\chaptermark\oldchaptermark

% Hier können Anhaenge angefuegt werden
\begin{appendices}
\chapter{Anhang}

\section{Analyse}

\subsection{We Were Here \& We Were Here Too}\label{sec:append_anylsis_wwh_wwht}

\subsubsection{Visuelle Analyse}\label{sec:append_anylsis_wwh_wwht_visual}
Im beiliegenden Datenträger in den Verzeichnissen $Analyse/WeWereHere/Visual$ und $Analyse/WeWereHereToo/Visual$.

\subsubsection{Rätseldesign nach \cite{tim_schafer_grim_1996}}\label{sec:append_riddles_wwh_wwht}
im beiliegenden Datenträger in den Verzeichnissen $Analyse/WeWereHere$ und $Analyse/WeWereHereToo$.

\subsection{Tiny Room Stories: Town Mystery}\label{sec:append_anylsis_trstm}

\subsubsection{Visuelle Analyse}\label{sec:append_analysis_trstm}
Im beiliegenden Datenträger im Verzeichnis $Analyse/TinyRoomStoriesTownMystery$.

\subsubsection{Rätseldesign nach \cite{tim_schafer_grim_1996}}\label{sec:append_riddles_trstm}
Ablaufdiagramm-Diagramm in Abbildung \ref{fig:trs-uml} auf der nächsten Seite; im Detail im beiliegenden Datenträger im Verzeichnis $Analyse/TinyRoomStoriesTownMystery$.

\newpage

\begin{figure}[ht]
\centering
\includegraphics[width=1\linewidth]{content/pictures/TinyRoomStoriesUML.png}
\caption{Rätseldesign von Tiny Room Stories (Quelle: eigene Darstellung)}
\label{fig:trs-uml}
\end{figure}

\clearpage

\subsection{Myrmidon}\label{sec:append_anylsis_m}

\subsubsection{Visuelle Analyse}\label{sec:append_anylsis_m_visual}
Im beiliegenden Datenträger im Verzeichnis $Analyse/Myrmidon/Visual$.

\subsubsection{Rätseldesign nach \cite{tim_schafer_grim_1996}}\label{sec:append_riddles_m}
UML-Diagramm in Abbildung \ref{fig:m-uml} auf der nächsten Seite; im Detail im beiliegenden Datenträger im Verzeichnis $Analyse/Myrmidon$.

\newpage

\begin{figure}[ht]
\centering
\includegraphics[width=1\linewidth]{content/pictures/RätseldesignMyrmidon.png}
\caption{Rätseldesign von Myrmidon (Quelle: eigene Darstellung)}
\label{fig:m-uml}
\end{figure}

\clearpage

\section{Konzeption}

\subsection{Personae}\label{sec:append_concept_personae}

\subsubsection{Uwe Kaufmann}

\includepdf[pages=-]{content/attachments/personae/Uwe Kaufmann.pdf}


\clearpage

\subsubsection{Steve Works}

\includepdf[pages=-]{content/attachments/personae/Steve Works.pdf}

\clearpage

\subsubsection{Anja Games}

\includepdf[pages=-]{content/attachments/personae/Anja Gayms.pdf}

\clearpage

\subsection{Spielabläufe}

\subsubsection{Ablauf des Spiels}\label{sec:append_gameloop}
Aktivitätsdiagramme von Player und Watcher im beiliegenden Datenträger im Verzeichnis $Konzeption/Spielabläufe/GameLoop$.

\subsubsection{Ablauf des Levels}\label{sec:append_levelloop}
Aktivitätsdiagramme von Player und Watcher im beiliegenden Datenträger im Verzeichnis $Konzeption/Spielabläufe/LevelLoop$.

\newpage

\subsection{Rätseldesign}

\subsubsection{Abschnitt 3 nach \cite{tim_schafer_grim_1996}}\label{sec:append_riddles_part_3}
UML-Diagramm in Abbildung \ref{fig:riddle-design-section02-uml} auf der nächsten Seite; im Detail im beiliegenden Datenträger im Verzeichnis $Konzeption/Rätseldeisgn$.

\begin{figure}[ht]
\centering
\includegraphics[width=0.7\linewidth]{content/pictures/Rätseldesign_Section02.drawio.png}
\caption{Rätseldesign von Abschnitt 3 (Quelle: eigene Darstellung)}
\label{fig:riddle-design-section02-uml}
\end{figure}

\clearpage

\section{Umsetzung}

\subsection{Ausgangslage}

\subsubsection{Abschlusspräsentation}\label{sec:append_realisation_ausgangslage_presentation}
Im beiliegenden Datenträger im Verzeichnis $Umsetzung/Ausgangslage/Abschlusspräsentation$.

\subsection{Verwendete Technologien}

\subsubsection{Vorlage Rucksack}\label{sec:append_realisation_vorlage_rucksack}

Prompt: \say{Ein Pfeil zeigt seitlich auf einen Rucksack, die Spitze des Pfeils ist am besten von über dem Rucksack auf den Rucksack gerichtet, in schwarzweiß und einfachen Linien gezeichnet.}

\begin{figure}[ht]
\centering
\includegraphics[width=1\linewidth]{content/attachments/vorlagen/Vorlage_Rucksack.png}
\caption{Vorlage für die Zeichnung des Rucksacks (Quelle: ChatGPT)}
\label{fig:vorlage_rucksack}
\end{figure}
\clearpage

\subsubsection{Vorlage Handzeichnung}\label{sec:append_realisation_vorlage_handzeichnung}
Prompt und Chat nicht mehr Verfügbar durch Änderung von Copilot in Bing
\begin{figure}[ht]
\centering
\includegraphics[width=1\linewidth]{content/attachments/vorlagen/Vorlage_Hand.jpeg}
\caption{Vorlage für die Zeichnung der Hand (Quelle: Copilot)}
\label{fig:vorlage_hand}
\end{figure}

\subsection{Questsystem}

\subsubsection{UML-Diagramm}\label{sec:append_realisation_uml_quest}
Im beiliegenden Datenträger im Verzeichnis $Umsetzung/Questsystem$. 

\subsection{Pathsystem}

\subsubsection{UML-Diagramm}\label{sec:append_realisation_uml_path}
Im beiliegenden Datenträger im Verzeichnis $Umsetzung/Pathsystem$. 

\subsection{Herausforderungen in der Umsetzung}

\subsubsection{Strukturgebende Maßstäbe}\label{sec:append_realisation_maßstaebe}
Basis-Breite von Wänden: 1 Meter -> vergrößerbar im 0.25 Meter Maßstab
Basis-Höhe von Wänden: 3 Meter -> verkleinerbar um die Hälfte auf 1.5 Meter
Basis-Tiefe von Wänden: 0.125 Meter

\section{Evaluation}

\subsection{Teststudie}

\subsubsection{Notizen}\label{sec:append_evaluation_pre_study_notes}
Im beiliegenden Datenträger im Verzeichnis $Evaluation/PreStudy/Notizen$.

\subsection{Forschungsstudie}

\subsubsection{Einverständniserklärung}\label{sec:append_study_consent}

\includepdf[pages=-]{content/attachments/study/Einverständniserklärung.pdf}

\subsubsection{System Usability Scale (SUS)}\label{sec:append_study_sus}

\includepdf[pages=-]{content/attachments/questtionaires/SUSCHAPT.pdf}

\subsubsection{Game Experience Questionnaire (GEQ)}\label{sec:append_study_xp}

\includepdf[pages=-]{content/attachments/questtionaires/Game_Experience_Questionnaire_English.pdf}

\subsubsection{Intrinsic Motivation Inventory (IMI)}\label{sec:append_study_imi}

Genutzt wurde der abschnitt \say{Interest-Enjoyment}
\includepdf[pages=-]{content/attachments/questtionaires/IMI_German.pdf}

\subsubsection{NASA Task Load Index (NASA-TLX)}\label{sec:append_study_tlx}

Genutzt wurde eine Skala von $0$ bis $10$
\includepdf[pages=-]{content/attachments/questtionaires/TLXScale.pdf}


\subsubsection{Self Assessment Manikin (SAM)}\label{sec:append_study_sam}

\begin{figure}[ht]
\centering
\includegraphics[width=1\linewidth]{content/attachments/questtionaires/Self-Assessment-Manikin-SAM-for-valence-arousal-and-dominance-The-five.png}
\caption{Fragebogen zum Self Assessment Manikin (SAM) (Quelle: \citealp{soares_affective_2013})}
\label{fig:append_sam}
\end{figure}

\clearpage

\subsubsection{Inclusion of the Other in the Self (IOS)}\label{sec:append_study_ios}


\begin{figure}[ht]
\centering
\includegraphics[width=1\linewidth]{content/attachments/questtionaires/IOS.png}
\caption{Fragebogen zur Inclusion of the Other in the Self (IOS) (Quelle: \citealp{gachter_measuring_2015})}
\label{fig:append_ios}
\end{figure}



\subsubsection{Spielertypen nach Bartle}\label{sec:append_study_bartle}


\begin{figure}[ht]
\centering
\includegraphics[width=1\linewidth]{content/attachments/questtionaires/bartle-spielertypenbrell-1024x512.png}
\caption{Spielertypen nach Bartle (Quelle: \citealp{bartle_hearts_1996})}
\label{fig:append_bartle}
\end{figure}

\clearpage

\subsubsection{Fragebogen zum Thema Leadership}\label{sec:append_study_leader}

\begin{figure}[ht]
\centering
\includegraphics[width=1\linewidth]{content/attachments/questtionaires/Leadership.PNG}
\caption{Fragebogen zum Thema Leadership (Quelle: \citealp{emmerich_game_2016})}
\label{fig:append_leadership}
\end{figure}

\clearpage

\subsubsection{Questionnaire of Cognitive and Affective Empathy (QCAE)}

\begin{figure}[ht]
\centering
\includegraphics[width=1\linewidth]{content/attachments/questtionaires/QCAE.PNG}
\caption{Fragebogen des Questionnaire of Cognitive and Affective Empathy (QCAE) (Quelle: \citealp{noauthor_questionnaire_nodate})}
\label{fig:append_QCAE}
\end{figure}

\subsubsection{Demografische Informationen}\label{sec:append_study_demografic}

\includepdf[pages=-]{content/attachments/questtionaires/Demografische-Informationen.pdf}

\subsubsection{Kommunikationsprotokolle}\label{sec:append_study_protocols}
Im beiliegenden Datenträger im Verzeichnis $Evaluation/Probandentests/Protokolle$. 

% \begin{figure}[ht]
% \centering
% \includegraphics[width=0.8\linewidth]{content/pictures/WeWereHereUML.png}
% \caption{Rätseldesign von We were here (Quelle: eigene Darstellung)}
% \label{fig:wwh-uml}
% \end{figure}

% \newpage

% \begin{figure}[ht]
% \centering
% \includegraphics[width=0.8\linewidth]{content/pictures/WeWereHereTooUML.png}
% \caption{Rätseldesign von We were here too (Quelle: eigene Darstellung)}
% \label{fig:wwht-uml}
% \end{figure}
% \newpage
% % \appendixsection{Monatsbericht Februar}Thesis Präsentation als PDF}{content/attachments/Praesentation.pdf}
% \includepdf[pages=-]{content/attachments/reportPraesentation.pdf}
% \appendixsection{Monatsbericht Januar}{content/attachments/report.pdfGamedesign Dokument Gamedesign Workshop}{content/attachments/gdd.pdf}
% \includepdf[pages=-]{content/attachments/gdd}
\end{appendices}
\end{document}      